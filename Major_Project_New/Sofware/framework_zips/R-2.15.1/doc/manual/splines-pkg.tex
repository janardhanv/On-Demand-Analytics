
\chapter{The \texttt{splines} package}
\HeaderA{splines-package}{Regression Spline Functions and Classes}{splines.Rdash.package}
\aliasA{splines}{splines-package}{splines}
\keyword{package}{splines-package}
\keyword{models}{splines-package}
%
\begin{Description}\relax
Regression spline functions and classes.
\end{Description}
%
\begin{Details}\relax
This package provides functions for working with regression
splines using the B-spline basis, \code{\LinkA{bs}{bs}}, and the
natural cubic spline basis, \code{\LinkA{ns}{ns}}.  

For a complete list of functions, use \code{library(help="splines")}.

\end{Details}
%
\begin{Author}\relax
Douglas M. Bates \email{bates@stat.wisc.edu} and William N. Venables
\email{Bill.Venables@csiro.au}

Maintainer: R Core Team \email{R-core@r-project.org}
\end{Author}
\HeaderA{asVector}{Coerce an Object to a Vector}{asVector}
\keyword{models}{asVector}
%
\begin{Description}\relax
This is a generic function.  Methods for this function coerce objects
of given classes to vectors.
\end{Description}
%
\begin{Usage}
\begin{verbatim}
asVector(object)
\end{verbatim}
\end{Usage}
%
\begin{Arguments}
\begin{ldescription}
\item[\code{object}] An object.
\end{ldescription}
\end{Arguments}
%
\begin{Details}\relax
Methods for vector coercion in new classes must be created for the
\code{asVector} generic instead of \code{as.vector}.  The
\code{as.vector} function is internal and not easily extended.
Currently the only class with an \code{asVector} method is the
\code{xyVector} class.
\end{Details}
%
\begin{Value}
a vector
\end{Value}
%
\begin{Author}\relax
Douglas Bates and Bill Venables
\end{Author}
%
\begin{SeeAlso}\relax
\code{\LinkA{xyVector}{xyVector}}
\end{SeeAlso}
%
\begin{Examples}
\begin{ExampleCode}
require(stats)
ispl <- interpSpline( weight ~ height,  women )
pred <- predict(ispl)
class(pred)
utils::str(pred)
asVector(pred)
\end{ExampleCode}
\end{Examples}
\HeaderA{backSpline}{Monotone Inverse Spline}{backSpline}
\keyword{models}{backSpline}
%
\begin{Description}\relax
Create a monotone inverse of a monotone natural spline.
\end{Description}
%
\begin{Usage}
\begin{verbatim}
backSpline(object)
\end{verbatim}
\end{Usage}
%
\begin{Arguments}
\begin{ldescription}
\item[\code{object}] an object that inherits from class \code{nbSpline} or
\code{npolySpline}.  That is, the object must represent a natural
interpolation spline but it can be either in the B-spline
representation or the piecewise polynomial one.  The spline is
checked to see if it represents a monotone function.

\end{ldescription}
\end{Arguments}
%
\begin{Value}
An object of class \code{polySpline} that contains the piecewise
polynomial representation of a function that has the appropriate
values and derivatives at the knot positions to be an inverse of the
spline represented by \code{object}.  Technically this object is not a
spline because the second derivative is not constrained to be
continuous at the knot positions.  However, it is often a much better
approximation to the inverse than fitting an interpolation spline to
the y/x pairs.
\end{Value}
%
\begin{Author}\relax
Douglas Bates and Bill Venables
\end{Author}
%
\begin{SeeAlso}\relax
\code{\LinkA{interpSpline}{interpSpline}}
\end{SeeAlso}
%
\begin{Examples}
\begin{ExampleCode}
require(graphics)
ispl <- interpSpline( women$height, women$weight )
bspl <- backSpline( ispl )
plot( bspl )                   # plots over the range of the knots
points( women$weight, women$height )
\end{ExampleCode}
\end{Examples}
\HeaderA{bs}{B-Spline Basis for Polynomial Splines}{bs}
\keyword{smooth}{bs}
%
\begin{Description}\relax
Generate the B-spline basis matrix for a polynomial spline.
\end{Description}
%
\begin{Usage}
\begin{verbatim}
bs(x, df = NULL, knots = NULL, degree = 3, intercept = FALSE,
   Boundary.knots = range(x))
\end{verbatim}
\end{Usage}
%
\begin{Arguments}
\begin{ldescription}
\item[\code{x}] the predictor variable.  Missing values are allowed.
\item[\code{df}] degrees of freedom; one can specify \code{df} rather than
\code{knots}; \code{bs()} then chooses \code{df-degree} (minus one
if there is an intercept) knots at suitable quantiles of \code{x}
(which will ignore missing values).
\item[\code{knots}] the \emph{internal} breakpoints that define the
spline.  The default is \code{NULL}, which results in a basis for
ordinary polynomial regression.  Typical values are the mean or
median for one knot, quantiles for more knots.  See also
\code{Boundary.knots}.
\item[\code{degree}] degree of the piecewise polynomial---default is \code{3} for
cubic splines.
\item[\code{intercept}] if \code{TRUE}, an intercept is included in the
basis; default is \code{FALSE}.
\item[\code{Boundary.knots}] boundary points at which to anchor the B-spline
basis (default the range of the data). If both \code{knots} and
\code{Boundary.knots} are supplied, the basis parameters do not
depend on \code{x}. Data can extend beyond \code{Boundary.knots}.
\end{ldescription}
\end{Arguments}
%
\begin{Value}
A matrix of dimension \code{c(length(x), df)}, where either \code{df}
was supplied or if \code{knots} were supplied, \code{df =
  length(knots) + degree} plus one if there is an intercept.  Attributes
are returned that correspond to the arguments to \code{bs}, and
explicitly give the \code{knots}, \code{Boundary.knots} etc for use by
\code{predict.bs()}.

\code{bs()} is based on the function \code{\LinkA{spline.des}{spline.des}()}.
It generates a basis matrix for
representing the family of piecewise polynomials with the specified
interior knots and degree, evaluated at the values of \code{x}.  A
primary use is in modeling formulas to directly specify a piecewise
polynomial term in a model.
\end{Value}
%
\begin{References}\relax
Hastie, T. J. (1992)
Generalized additive models.
Chapter 7 of \emph{Statistical Models in S}
eds J. M. Chambers and T. J. Hastie, Wadsworth \& Brooks/Cole.
\end{References}
%
\begin{SeeAlso}\relax
\code{\LinkA{ns}{ns}}, \code{\LinkA{poly}{poly}}, \code{\LinkA{smooth.spline}{smooth.spline}},
\code{\LinkA{predict.bs}{predict.bs}}, \code{\LinkA{SafePrediction}{SafePrediction}}
\end{SeeAlso}
%
\begin{Examples}
\begin{ExampleCode}
require(stats); require(graphics)
bs(women$height, df = 5)
summary(fm1 <- lm(weight ~ bs(height, df = 5), data = women))

## example of safe prediction
plot(women, xlab = "Height (in)", ylab = "Weight (lb)")
ht <- seq(57, 73, length.out = 200)
lines(ht, predict(fm1, data.frame(height=ht)))
## Not run: 
## Consistency:
x <- c(1:3,5:6)
stopifnot(identical(bs(x), bs(x, df = 3)),
          !is.null(kk <- attr(bs(x), "knots")),# not true till 1.5.1
          length(kk) == 0)

## End(Not run)
\end{ExampleCode}
\end{Examples}
\HeaderA{interpSpline}{Create an Interpolation Spline}{interpSpline}
\keyword{models}{interpSpline}
%
\begin{Description}\relax
Create an interpolation spline, either from \code{x} and \code{y}
vectors, or from a formula/data.frame combination.
\end{Description}
%
\begin{Usage}
\begin{verbatim}
interpSpline(obj1, obj2, bSpline = FALSE, period = NULL,
             na.action = na.fail)
\end{verbatim}
\end{Usage}
%
\begin{Arguments}
\begin{ldescription}
\item[\code{obj1}] Either a numeric vector of \code{x} values or a formula.
\item[\code{obj2}] If \code{obj1} is numeric this should be a numeric vector
of the same length.  If \code{obj1} is a formula this can be an
optional data frame in which to evaluate the names in the formula.
\item[\code{bSpline}] If \code{TRUE} the b-spline representation is returned,
otherwise the piecewise polynomial representation is returned.
Defaults to \code{FALSE}.
\item[\code{period}] An optional positive numeric value giving a period for a
periodic interpolation spline.
\item[\code{na.action}] a optional function which indicates what should happen
when the data contain \code{NA}s.  The default action
(\code{na.omit}) is to omit any incomplete observations.  The
alternative action \code{na.fail} causes \code{interpSpline} to print
an error message and terminate if there are any incomplete
observations.
\end{ldescription}
\end{Arguments}
%
\begin{Value}
An object that inherits from class \code{spline}. The object can be in
the B-spline representation, in which case it will be of class
\code{nbSpline} for natural B-spline, or in the piecewise polynomial
representation, in which case it will be of class \code{npolySpline}.
\end{Value}
%
\begin{Author}\relax
Douglas Bates and Bill Venables
\end{Author}
%
\begin{SeeAlso}\relax
\code{\LinkA{splineKnots}{splineKnots}},
\code{\LinkA{splineOrder}{splineOrder}},
\code{\LinkA{periodicSpline}{periodicSpline}}.
\end{SeeAlso}
%
\begin{Examples}
\begin{ExampleCode}
require(graphics); require(stats)
ispl <- interpSpline( women$height, women$weight )
ispl2 <- interpSpline( weight ~ height,  women )
# ispl and ispl2 should be the same
plot( predict( ispl, seq( 55, 75, length.out = 51 ) ), type = "l" )
points( women$height, women$weight )
plot( ispl )    # plots over the range of the knots
points( women$height, women$weight )
splineKnots( ispl )
\end{ExampleCode}
\end{Examples}
\HeaderA{ns}{Generate a Basis Matrix for Natural Cubic Splines}{ns}
\keyword{smooth}{ns}
%
\begin{Description}\relax
Generate the B-spline basis matrix for a natural cubic spline.
\end{Description}
%
\begin{Usage}
\begin{verbatim}
ns(x, df = NULL, knots = NULL, intercept = FALSE,
   Boundary.knots = range(x))
\end{verbatim}
\end{Usage}
%
\begin{Arguments}
\begin{ldescription}
\item[\code{x}] the predictor variable.  Missing values are allowed.
\item[\code{df}] degrees of freedom. One can supply \code{df} rather than
knots; \code{ns()} then chooses \code{df - 1 - intercept} knots at
suitably chosen quantiles of \code{x} (which will ignore missing values).
\item[\code{knots}] breakpoints that define the spline.  The default is no
knots; together with the natural boundary conditions this results in
a basis for linear regression on \code{x}.  Typical values are the
mean or median for one knot, quantiles for more knots.  See also
\code{Boundary.knots}.
\item[\code{intercept}] if \code{TRUE}, an intercept is included in the
basis; default is \code{FALSE}.
\item[\code{Boundary.knots}] boundary points at which to impose the natural
boundary conditions and anchor the B-spline basis (default the range
of the data).  If both \code{knots} and \code{Boundary.knots} are
supplied, the basis parameters do not depend on \code{x}. Data can
extend beyond \code{Boundary.knots}
\end{ldescription}
\end{Arguments}
%
\begin{Value}
A matrix of dimension \code{length(x) * df} where either \code{df} was
supplied or if \code{knots} were supplied,
\code{df = length(knots) + 1 + intercept}.
Attributes are returned that correspond to the arguments to \code{ns},
and explicitly give the \code{knots}, \code{Boundary.knots} etc for
use by \code{predict.ns()}.

\code{ns()} is based on the function \code{\LinkA{spline.des}{spline.des}}.  It
generates a basis matrix for representing the family of
piecewise-cubic splines with the specified sequence of
interior knots, and the natural boundary conditions.  These enforce
the constraint that the function is linear beyond the boundary knots,
which can either be supplied, else default to the extremes of the
data.  A primary use is in modeling formula to directly specify a
natural spline term in a model.
\end{Value}
%
\begin{References}\relax
Hastie, T. J. (1992)
Generalized additive models.
Chapter 7 of \emph{Statistical Models in S}
eds J. M. Chambers and T. J. Hastie, Wadsworth \& Brooks/Cole.
\end{References}
%
\begin{SeeAlso}\relax
\code{\LinkA{bs}{bs}}, \code{\LinkA{predict.ns}{predict.ns}}, \code{\LinkA{SafePrediction}{SafePrediction}}
\end{SeeAlso}
%
\begin{Examples}
\begin{ExampleCode}
require(stats); require(graphics)
ns(women$height, df = 5)
summary(fm1 <- lm(weight ~ ns(height, df = 5), data = women))

## example of safe prediction
plot(women, xlab = "Height (in)", ylab = "Weight (lb)")
ht <- seq(57, 73, length.out = 200)
lines(ht, predict(fm1, data.frame(height=ht)))
\end{ExampleCode}
\end{Examples}
\HeaderA{periodicSpline}{Create a Periodic Interpolation Spline}{periodicSpline}
\keyword{models}{periodicSpline}
%
\begin{Description}\relax
Create a periodic interpolation spline, either from \code{x} and
\code{y} vectors, or from a formula/data.frame combination.
\end{Description}
%
\begin{Usage}
\begin{verbatim}
periodicSpline(obj1, obj2, knots, period = 2*pi, ord = 4)
\end{verbatim}
\end{Usage}
%
\begin{Arguments}
\begin{ldescription}
\item[\code{obj1}] either a numeric vector of \code{x} values or a formula.
\item[\code{obj2}] if \code{obj1} is numeric this should be a numeric vector
of the same length.  If \code{obj1} is a formula this can be an
optional data frame in which to evaluate the names in the formula.
\item[\code{knots}] optional numeric vector of knot positions.
\item[\code{period}] positive numeric value giving the period for the
periodic spline.  Defaults to \code{2 * pi}.
\item[\code{ord}] integer giving the order of the spline, at least 2.  Defaults
to 4.  See \code{\LinkA{splineOrder}{splineOrder}} for a definition of the order of
a spline.
\end{ldescription}
\end{Arguments}
%
\begin{Value}
An object that inherits from class \code{spline}.  The object can be in
the B-spline representation, in which case it will be a
\code{pbSpline} object, or in the piecewise polynomial representation
(a \code{ppolySpline} object).
\end{Value}
%
\begin{Author}\relax
Douglas Bates and Bill Venables
\end{Author}
%
\begin{SeeAlso}\relax
\code{\LinkA{splineKnots}{splineKnots}},
\code{\LinkA{interpSpline}{interpSpline}}
\end{SeeAlso}
%
\begin{Examples}
\begin{ExampleCode}
require(graphics); require(stats)
xx <- seq( -pi, pi, length.out = 16 )[-1]
yy <- sin( xx )
frm <- data.frame( xx, yy )
pispl <- periodicSpline( xx, yy, period = 2 * pi)
pispl
pispl2 <- periodicSpline( yy ~ xx, frm, period = 2 * pi )
stopifnot(all.equal(pispl, pispl2))# pispl and pispl2 are the same

plot( pispl )          # displays over one period
points( yy ~ xx, col = "brown")
plot( predict( pispl, seq(-3*pi, 3*pi, length.out = 101) ), type = "l" )
\end{ExampleCode}
\end{Examples}
\HeaderA{polySpline}{Piecewise Polynomial Spline Representation}{polySpline}
\aliasA{as.polySpline}{polySpline}{as.polySpline}
\keyword{models}{polySpline}
%
\begin{Description}\relax
Create the piecewise polynomial representation of a spline object.
\end{Description}
%
\begin{Usage}
\begin{verbatim}
polySpline(object, ...)
as.polySpline(object, ...)
\end{verbatim}
\end{Usage}
%
\begin{Arguments}
\begin{ldescription}
\item[\code{object}] An object that inherits from class \code{spline}.
\item[\code{...}] Optional additional arguments.  At present no additional
arguments are used.
\end{ldescription}
\end{Arguments}
%
\begin{Value}
An object that inherits from class \code{polySpline}.  This is the
piecewise polynomial representation of a univariate spline function.
It is defined by a set of distinct numeric values called knots.  The
spline function is a polynomial function between each successive pair
of knots.  At each interior knot the polynomial segments on each side
are constrained to have the same value of the function and some of its
derivatives.
\end{Value}
%
\begin{Author}\relax
Douglas Bates and Bill Venables
\end{Author}
%
\begin{SeeAlso}\relax
\code{\LinkA{interpSpline}{interpSpline}},
\code{\LinkA{periodicSpline}{periodicSpline}},
\code{\LinkA{splineKnots}{splineKnots}},
\code{\LinkA{splineOrder}{splineOrder}}
\end{SeeAlso}
%
\begin{Examples}
\begin{ExampleCode}
require(graphics)
ispl <- polySpline(interpSpline( weight ~ height,  women, bSpline = TRUE))
print( ispl )   # print the piecewise polynomial representation
plot( ispl )    # plots over the range of the knots
points( women$height, women$weight )
\end{ExampleCode}
\end{Examples}
\HeaderA{predict.bs}{Evaluate a Spline Basis}{predict.bs}
\aliasA{predict.ns}{predict.bs}{predict.ns}
\keyword{smooth}{predict.bs}
%
\begin{Description}\relax
Evaluate a predefined spline basis at given values.
\end{Description}
%
\begin{Usage}
\begin{verbatim}
## S3 method for class 'bs'
predict(object, newx, ...)

## S3 method for class 'ns'
predict(object, newx, ...)
\end{verbatim}
\end{Usage}
%
\begin{Arguments}
\begin{ldescription}
\item[\code{object}] the result of a call to \code{\LinkA{bs}{bs}} or
\code{\LinkA{ns}{ns}} having attributes describing \code{knots},
\code{degree}, etc.
\item[\code{newx}] the \code{x} values at which evaluations are required.
\item[\code{...}] Optional additional arguments.  At present no additional
arguments are used.
\end{ldescription}
\end{Arguments}
%
\begin{Value}
An object just like \code{object}, except evaluated at the new values
of \code{x}.

These are methods for the generic function \code{\LinkA{predict}{predict}} for
objects inheriting from classes \code{"bs"} or \code{"ns"}.  See
\code{\LinkA{predict}{predict}} for the general behavior of this function.
\end{Value}
%
\begin{SeeAlso}\relax
\code{\LinkA{bs}{bs}}, \code{\LinkA{ns}{ns}}, \code{\LinkA{poly}{poly}}.
\end{SeeAlso}
%
\begin{Examples}
\begin{ExampleCode}
require(stats)
basis <- ns(women$height, df = 5)
newX <- seq(58, 72, length.out = 51)
# evaluate the basis at the new data
predict(basis, newX)
\end{ExampleCode}
\end{Examples}
\HeaderA{predict.bSpline}{Evaluate a Spline at New Values of x}{predict.bSpline}
\aliasA{predict.nbSpline}{predict.bSpline}{predict.nbSpline}
\aliasA{predict.npolySpline}{predict.bSpline}{predict.npolySpline}
\aliasA{predict.pbSpline}{predict.bSpline}{predict.pbSpline}
\aliasA{predict.ppolySpline}{predict.bSpline}{predict.ppolySpline}
\keyword{models}{predict.bSpline}
%
\begin{Description}\relax
The \code{predict} methods for the classes that inherit from the
virtual classes \code{bSpline} and \code{polySpline} are used to
evaluate the spline or its derivatives.  The \code{plot} method for a
spline object first evaluates \code{predict} with the \code{x}
argument missing, then plots the resulting \code{xyVector} with
\code{type = "l"}.
\end{Description}
%
\begin{Usage}
\begin{verbatim}
## S3 method for class 'bSpline'
predict(object, x, nseg=50, deriv=0, ...)
## S3 method for class 'nbSpline'
predict(object, x, nseg=50, deriv=0, ...)
## S3 method for class 'pbSpline'
predict(object, x, nseg=50, deriv=0, ...)
## S3 method for class 'npolySpline'
predict(object, x, nseg=50, deriv=0, ...)
## S3 method for class 'ppolySpline'
predict(object, x, nseg=50, deriv=0, ...)
\end{verbatim}
\end{Usage}
%
\begin{Arguments}
\begin{ldescription}
\item[\code{object}] An object that inherits from the \code{bSpline} or the
\code{polySpline} class.
\item[\code{x}] A numeric vector of \code{x} values at which to evaluate the
spline.  If this argument is missing a suitable set of \code{x}
values is generated as a sequence of \code{nseq} segments spanning
the range of the knots.
\item[\code{nseg}] A positive integer giving the number of segments in a set
of equally-spaced \code{x} values spanning the range of the knots
in \code{object}.  This value is only used if \code{x} is missing.
\item[\code{deriv}] An integer between 0 and \code{splineOrder(object) - 1}
specifying the derivative to evaluate.
\item[\code{...}] further arguments passed to or from other methods.
\end{ldescription}
\end{Arguments}
%
\begin{Value}
an \code{xyVector} with components
\begin{ldescription}
\item[\code{x}] the supplied or inferred numeric vector of \code{x} values
\item[\code{y}] the value of the spline (or its \code{deriv}'th derivative)
at the \code{x} vector
\end{ldescription}
\end{Value}
%
\begin{Author}\relax
Douglas Bates and Bill Venables
\end{Author}
%
\begin{SeeAlso}\relax
\code{\LinkA{xyVector}{xyVector}},
\code{\LinkA{interpSpline}{interpSpline}},
\code{\LinkA{periodicSpline}{periodicSpline}}
\end{SeeAlso}
%
\begin{Examples}
\begin{ExampleCode}
require(graphics); require(stats)
ispl <- interpSpline( weight ~ height,  women )
opar <- par(mfrow = c(2, 2), las = 1)
plot(predict(ispl, nseg = 201),     # plots over the range of the knots
     main = "Original data with interpolating spline", type = "l",
     xlab = "height", ylab = "weight") 
points(women$height, women$weight, col = 4)
plot(predict(ispl, nseg = 201, deriv = 1),
     main = "First derivative of interpolating spline", type = "l",
     xlab = "height", ylab = "weight") 
plot(predict(ispl, nseg = 201, deriv = 2),
     main = "Second derivative of interpolating spline", type = "l",
     xlab = "height", ylab = "weight") 
plot(predict(ispl, nseg = 401, deriv = 3),
     main = "Third derivative of interpolating spline", type = "l",
     xlab = "height", ylab = "weight") 
par(opar)
\end{ExampleCode}
\end{Examples}
\HeaderA{splineDesign}{Design Matrix for B-splines}{splineDesign}
\aliasA{spline.des}{splineDesign}{spline.des}
\keyword{models}{splineDesign}
%
\begin{Description}\relax
Evaluate the design matrix for the B-splines defined by \code{knots}
at the values in \code{x}.
\end{Description}
%
\begin{Usage}
\begin{verbatim}
splineDesign(knots, x, ord = 4, derivs, outer.ok = FALSE, sparse = FALSE)
spline.des  (knots, x, ord = 4, derivs, outer.ok = FALSE, sparse = FALSE)
\end{verbatim}
\end{Usage}
%
\begin{Arguments}
\begin{ldescription}
\item[\code{knots}] a numeric vector of knot positions with non-decreasing
values.
\item[\code{x}] a numeric vector of values at which to evaluate the B-spline
functions or derivatives.  Unless \code{outer.ok} is true, the
values in \code{x} must be between \code{knots[ord]} and
\code{knots[ length(knots) + 1 - ord ]}.
\item[\code{ord}] a positive integer giving the order of the spline function.
This is the number of coefficients in each piecewise polynomial
segment, thus a cubic spline has order 4.  Defaults to 4.
\item[\code{derivs}] an integer vector of the same length as \code{x} and with
values between \code{0} and \code{ord - 1}.  The derivative of the
given order is evaluated at the \code{x} positions.  Defaults to a
vector of zeroes of the same length as \code{x}.
\item[\code{outer.ok}] logical indicating if \code{x} should be allowed
outside the \emph{inner} knots, see the \code{x} argument.
\item[\code{sparse}] logical indicating if the result should inherit from class
\code{\LinkA{sparseMatrix}{sparseMatrix.Rdash.class}} (package \pkg{Matrix}).
\end{ldescription}
\end{Arguments}
%
\begin{Value}
A matrix with \code{length( x )} rows and \code{length( knots ) - ord}
columns.  The i'th row of the matrix contains the coefficients of the
B-splines (or the indicated derivative of the B-splines) defined by
the \code{knot} vector and evaluated at the i'th value of \code{x}.
Each B-spline is defined by a set of \code{ord} successive knots so
the total number of B-splines is \code{length(knots)-ord}.
\end{Value}
%
\begin{Note}\relax
The older \code{spline.des} function takes the same arguments but
returns a list with several components including \code{knots},
\code{ord}, \code{derivs}, and \code{design}.  The \code{design}
component is the same as the value of the \code{splineDesign}
function.
\end{Note}
%
\begin{Author}\relax
Douglas Bates and Bill Venables
\end{Author}
%
\begin{Examples}
\begin{ExampleCode}
require(graphics)
splineDesign(knots = 1:10, x = 4:7)
## "visualize" band structure
Matrix::drop0(zapsmall(6*splineDesign(knots = 1:40, x = 4:37, sparse=TRUE)))

knots <- c(1,1.8,3:5,6.5,7,8.1,9.2,10)# 10 => 10-4 = 6 Basis splines
x <- seq(min(knots)-1, max(knots)+1, length.out=501)
bb <- splineDesign(knots, x=x, outer.ok = TRUE)

plot(range(x), c(0,1), type="n", xlab="x", ylab="",
     main= "B-splines - sum to 1 inside inner knots")
mtext(expression(B[j](x) *"  and "* sum(B[j](x), j==1, 6)), adj=0)
abline(v=knots, lty=3, col="light gray")
abline(v=knots[c(4,length(knots)-3)], lty=3, col="gray10")
lines(x, rowSums(bb), col="gray", lwd=2)
matlines(x, bb, ylim = c(0,1), lty=1)
\end{ExampleCode}
\end{Examples}
\HeaderA{splineKnots}{Knot Vector from a Spline}{splineKnots}
\keyword{models}{splineKnots}
%
\begin{Description}\relax
Return the knot vector corresponding to a spline object.
\end{Description}
%
\begin{Usage}
\begin{verbatim}
splineKnots(object)
\end{verbatim}
\end{Usage}
%
\begin{Arguments}
\begin{ldescription}
\item[\code{object}] an object that inherits from class \code{"spline"}.
\end{ldescription}
\end{Arguments}
%
\begin{Value}
A non-decreasing numeric vector of knot positions.
\end{Value}
%
\begin{Author}\relax
Douglas Bates and Bill Venables
\end{Author}
%
\begin{Examples}
\begin{ExampleCode}
ispl <- interpSpline( weight ~ height, women )
splineKnots( ispl )
\end{ExampleCode}
\end{Examples}
\HeaderA{splineOrder}{Determine the Order of a Spline}{splineOrder}
\keyword{models}{splineOrder}
%
\begin{Description}\relax
Return the order of a spline object.
\end{Description}
%
\begin{Usage}
\begin{verbatim}
splineOrder(object)
\end{verbatim}
\end{Usage}
%
\begin{Arguments}
\begin{ldescription}
\item[\code{object}] An object that inherits from class \code{"spline"}.
\end{ldescription}
\end{Arguments}
%
\begin{Details}\relax
The order of a spline is the number of coefficients in each piece of
the piecewise polynomial representation.  Thus a cubic spline has
order 4.
\end{Details}
%
\begin{Value}
A positive integer.
\end{Value}
%
\begin{Author}\relax
Douglas Bates and Bill Venables
\end{Author}
%
\begin{SeeAlso}\relax
\code{\LinkA{splineKnots}{splineKnots}},
\code{\LinkA{interpSpline}{interpSpline}},
\code{\LinkA{periodicSpline}{periodicSpline}}
\end{SeeAlso}
%
\begin{Examples}
\begin{ExampleCode}
splineOrder( interpSpline( weight ~ height, women ) )
\end{ExampleCode}
\end{Examples}
\HeaderA{xyVector}{Construct an \code{xyVector} Object}{xyVector}
\keyword{models}{xyVector}
%
\begin{Description}\relax
Create an object to represent a set of x-y pairs.  The resulting
object can be treated as a matrix or as a data frame or as a vector.
When treated as a vector it reduces to the \code{y} component only.

The result of functions such as \code{predict.spline} is returned as
an \code{xyVector} object so the x-values used to generate the
y-positions are retained, say for purposes of generating plots.
\end{Description}
%
\begin{Usage}
\begin{verbatim}
xyVector(x, y)
\end{verbatim}
\end{Usage}
%
\begin{Arguments}
\begin{ldescription}
\item[\code{x}] a numeric vector
\item[\code{y}] a numeric vector of the same length as \code{x}
\end{ldescription}
\end{Arguments}
%
\begin{Value}
An object of class \code{xyVector} with components
\begin{ldescription}
\item[\code{x}] a numeric vector
\item[\code{y}] a numeric vector of the same length as \code{x}
\end{ldescription}
\end{Value}
%
\begin{Author}\relax
Douglas Bates and Bill Venables
\end{Author}
%
\begin{Examples}
\begin{ExampleCode}
require(stats); require(graphics)
ispl <- interpSpline( weight ~ height, women )
weights <- predict( ispl, seq( 55, 75, length.out = 51 ))
class( weights )
plot( weights, type = "l", xlab = "height", ylab = "weight" )
points( women$height, women$weight )
weights
\end{ExampleCode}
\end{Examples}
\clearpage
