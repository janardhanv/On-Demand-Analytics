
\chapter{The \texttt{tools} package}
\HeaderA{tools-package}{Tools for Package Development}{tools.Rdash.package}
\aliasA{tools}{tools-package}{tools}
\keyword{package}{tools-package}
\keyword{programming}{tools-package}
\keyword{documentation}{tools-package}
%
\begin{Description}\relax
Tools for package development, administration and documentation.
\end{Description}
%
\begin{Details}\relax
This package contains tools for manipulating R packages and their
documentation.

For a complete list of functions, use \code{library(help="tools")}.

\end{Details}
%
\begin{Author}\relax
Kurt Hornik and Friedrich Leisch

Maintainer: R Core Team \email{R-core@r-project.org}
\end{Author}
\HeaderA{add\_datalist}{Add a \file{datalist} File to a Package}{add.Rul.datalist}
\keyword{utilities}{add\_datalist}
%
\begin{Description}\relax
The \code{\LinkA{data}{data}()} command with no arguments lists all the
datasets available via \code{data} in attached packages, and to do so a
per-package list is installed.  Creating that list at install time can
be slow for packages with huge datasets, and can be expedited by a
supplying \file{data/datalist} file.
\end{Description}
%
\begin{Usage}
\begin{verbatim}
add_datalist(pkgpath, force = FALSE)
\end{verbatim}
\end{Usage}
%
\begin{Arguments}
\begin{ldescription}
\item[\code{pkgpath}] The path to a (source) package.

\item[\code{force}] logical: can an existing \file{data/datalist} file be
over-written?
\end{ldescription}
\end{Arguments}
%
\begin{Details}\relax
\command{R CMD build} will call this function to add a data list to
packages with 1MB or more of data.

It is also helpful to give a \file{data/datalist} file in packages
whose datasets have many dependencies, including loading the packages
itself (and maybe others).
\end{Details}
%
\begin{SeeAlso}\relax
\code{\LinkA{data}{data}}.

The `Writing R Extensions' manual.
\end{SeeAlso}
\HeaderA{bibstyle}{Select or define a bibliography style.}{bibstyle}
\keyword{utilties}{bibstyle}
\keyword{documentation}{bibstyle}
%
\begin{Description}\relax
This function defines and registers styles for rendering \code{\LinkA{bibentry}{bibentry}} objects
into Rd format, for later conversion to text, HTML, etc.
\end{Description}
%
\begin{Usage}
\begin{verbatim}
bibstyle(style, envir, ..., .init = FALSE, .default = FALSE)
\end{verbatim}
\end{Usage}
%
\begin{Arguments}
\begin{ldescription}
\item[\code{style}] 
A character string naming the style.

\item[\code{envir}] 
(optional) An environment holding the functions to implement the style.

\item[\code{...}] 
Named arguments to add to the environment.

\item[\code{.init}] 
Whether to initialize the environment from the default style \code{"JSS"}.

\item[\code{.default}] 
Whether to set the specified style as the default style.

\end{ldescription}
\end{Arguments}
%
\begin{Details}\relax
Rendering of \code{\LinkA{bibentry}{bibentry}} objects may be done using routines
modelled after those used by BibTeX.  This function allows environments
to be created and manipulated to contain those routines.

There are two ways to create a new style environment.  The easiest
is to set \code{.init = TRUE}, in which case the environment will be
initialized with a copy of the default \code{"JSS"} environment.  (This style
is modelled after the \file{jss.bst} style used by the \emph{Journal 
of Statistical Software}.)  Alternatively, the \code{envir} argument 
can be used to specify a completely new style environment.

To simply retrieve an existing style, specify \code{style}
and no other arguments.  To modify an existing style, specify \code{style}
and some named entries via \code{...}.  (Modifying the default
\code{"JSS"} style is discouraged.)  Setting \code{style} to \code{NULL}
or leaving it missing will retrieve the default style, but modifications
will not be allowed.

At a minimum, the environment should contain routines to render each
of the 12 types of bibliographic entry supported by
\code{\LinkA{bibentry}{bibentry}} as well as a routine to produce a sort key to
sort the entries.  The former must be named \code{formatArticle},
\code{formatBook}, \code{formatInbook}, \code{formatIncollection},
\code{formatInProceedings}, \code{formatManual},
\code{formatMastersthesis}, \code{formatMisc}, \code{formatPhdthesis},
\code{formatProceedings}, \code{formatTechreport} and
\code{formatUnpublished}.  Each of these takes one argument, a single
\code{\LinkA{unclass}{unclass}}'ed entry from the \code{\LinkA{bibentry}{bibentry}} vector
passed to the renderer, and should produce a single element character
vector (possibly containing newlines).  The sort keys are produced
by a function named \code{sortKeys}.  It is passed the original 
\code{\LinkA{bibentry}{bibentry}} vector and should produce a sortable vector of 
the same length to define the sort order.
\end{Details}
%
\begin{Value}
The environment which has been selected or created.
\end{Value}
%
\begin{Author}\relax
Duncan Murdoch
\end{Author}
%
\begin{SeeAlso}\relax
\code{\LinkA{bibentry}{bibentry}}
\end{SeeAlso}
%
\begin{Examples}
\begin{ExampleCode}

refs <- 
c(bibentry(bibtype = "manual",
    title = "R: A Language and Environment for Statistical Computing",
    author = person("R Core Team"),
    organization = "R Foundation for Statistical Computing",
    address = "Vienna, Austria",
    year = 2010,
    isbn = "3-900051-07-0",
    url = "http://www.R-project.org"),  
  bibentry(bibtype = "article",
    author = c(person(c("George", "E", "P"), "Box"),
               person(c("David",  "R"),      "Cox")),
    year = 1964,
    title = "An Analysis of Transformations",
    journal = "Journal of the Royal Statistical Society, Series B",
    volume = 26,
    pages = "211-252"))
bibstyle("unsorted", sortKeys = function(refs) seq_along(refs),
       .init = TRUE)
print(refs, .bibstyle = "unsorted")    
\end{ExampleCode}
\end{Examples}
\HeaderA{buildVignettes}{List and Build Package Vignettes}{buildVignettes}
\aliasA{pkgVignettes}{buildVignettes}{pkgVignettes}
\keyword{utilities}{buildVignettes}
\keyword{documentation}{buildVignettes}
%
\begin{Description}\relax
Run \code{\LinkA{Sweave}{Sweave}} and \code{\LinkA{texi2dvi}{texi2dvi}} on all vignettes
of a package.
\end{Description}
%
\begin{Usage}
\begin{verbatim}
buildVignettes(package, dir, lib.loc = NULL, quiet = TRUE, clean = TRUE)

pkgVignettes(package, dir, lib.loc = NULL)
\end{verbatim}
\end{Usage}
%
\begin{Arguments}
\begin{ldescription}
\item[\code{package}] a character string naming an installed package.  If
given, Sweave files are searched in subdirectory \file{doc}.
\item[\code{dir}] a character string specifying the path to a package's root
source directory.  This subdirectory \file{vignettes} (or if it
does not exist \file{inst/doc}) is searched for Sweave files.
\item[\code{lib.loc}] a character vector of directory names of \R{} libraries,
or \code{NULL}.  The default value of \code{NULL} corresponds to all
libraries currently known.  The specified library trees are used to
search for \code{package}.
\item[\code{quiet}] logical. Run \code{\LinkA{Sweave}{Sweave}} and
\code{\LinkA{texi2pdf}{texi2pdf}} in quiet mode.
\item[\code{clean}] Remove all files generated by the build, even if there
were copies there before.
\end{ldescription}
\end{Arguments}
%
\begin{Details}\relax
\code{buildVignettes} is used by \code{R CMD build} and \code{R CMD
    check} to (re-)build vignette PDFs on the Sweave sources.
\end{Details}
%
\begin{Value}
\code{buildVignettes} is called for its side effect of creating the
PDF versions of all vignettes.

\code{pkgVignettes} returns an object of class \code{"pkgVignettes"}
if a vignette directory is found, otherwise \code{NULL}.
\end{Value}
\HeaderA{charsets}{Conversion Tables between Character Sets}{charsets}
\aliasA{Adobe\_glyphs}{charsets}{Adobe.Rul.glyphs}
\aliasA{charset\_to\_Unicode}{charsets}{charset.Rul.to.Rul.Unicode}
\keyword{datasets}{charsets}
%
\begin{Description}\relax
\code{charset\_to\_Unicode} is a matrix of Unicode points with columns
for the common 8-bit encodings.

\code{Adobe\_glyphs} is a dataframe which gives Adobe glyph names for
Unicode points. It has two character columns, \code{"adobe"} and
\code{"unicode"} (a 4-digit hex representation).
\end{Description}
%
\begin{Usage}
\begin{verbatim}
charset_to_Unicode

Adobe_glyphs
\end{verbatim}
\end{Usage}
%
\begin{Details}\relax
\code{charset\_to\_Unicode} is an integer matrix of class
\code{c("\LinkA{noquote}{noquote}", "\LinkA{hexmode}{hexmode}")} so prints in hexadecimal.
The mappings are those used by \code{libiconv}: there are differences
in the way quotes and minus/hyphen are mapped between sources (and the
postscript encoding files use a different mapping).

\code{Adobe\_glyphs} include all the Adobe glyph names which correspond to
single Unicode characters.  It is sorted by Unicode point and within a
point alphabetically on the glyph (there can be more than one name for
a Unicode point).  The data are in the file
\file{\var{\LinkA{R\_HOME}{R.Rul.HOME}}/share/encodings/Adobe\_glyphlist}.
\end{Details}
%
\begin{Source}\relax
\url{http://partners.adobe.com/public/developer/en/opentype/glyphlist.txt}
\end{Source}
%
\begin{Examples}
\begin{ExampleCode}
## find Adobe names for ISOLatin2 chars.
latin2 <- charset_to_Unicode[, "ISOLatin2"]
aUnicode <- as.numeric(paste("0x", Adobe_glyphs$unicode, sep=""))
keep <- aUnicode %in% latin2
aUnicode <- aUnicode[keep]
aAdobe <- Adobe_glyphs[keep, 1]
## first match
aLatin2 <- aAdobe[match(latin2, aUnicode)]
## all matches
bLatin2 <- lapply(1:256, function(x) aAdobe[aUnicode == latin2[x]])
format(bLatin2, justify="none")
\end{ExampleCode}
\end{Examples}
\HeaderA{checkFF}{Check Foreign Function Calls}{checkFF}
\aliasA{print.checkFF}{checkFF}{print.checkFF}
\keyword{programming}{checkFF}
\keyword{utilities}{checkFF}
%
\begin{Description}\relax
Performs checks on calls to compiled code from R code.  Currently only
checks whether the interface functions such as \code{.C} and
\code{.Fortran} are called with a \code{"\LinkA{NativeSymbolInfo}{NativeSymbolInfo}"}
first argument or with argument \code{PACKAGE} specified, which is
highly recommended to avoid name clashes in foreign function calls.
\end{Description}
%
\begin{Usage}
\begin{verbatim}
checkFF(package, dir, file, lib.loc = NULL,
        verbose = getOption("verbose"))
\end{verbatim}
\end{Usage}
%
\begin{Arguments}
\begin{ldescription}
\item[\code{package}] a character string naming an installed package.  If
given, the installed R code of the package is checked.
\item[\code{dir}] a character string specifying the path to a package's root
source directory.  This should contain the subdirectory \file{R}
(for R code).  Only used if \code{package} is not given.
\item[\code{file}] the name of a file containing R code to be checked.  Used
if neither \code{package} nor \code{dir} are given.
\item[\code{lib.loc}] a character vector of directory names of \R{} libraries,
or \code{NULL}.  The default value of \code{NULL} corresponds to all
libraries currently known.  The specified library trees are used to
search for \code{package}.
\item[\code{verbose}] a logical.  If \code{TRUE}, additional diagnostics are
printed (and the result is returned invisibly).
\end{ldescription}
\end{Arguments}
%
\begin{Details}\relax
Note that we can only check if the \code{name} argument is a symbol or
a character string, not what class of object the symbol resolves to
at run-time.

\emph{If} the package has a namespace and \emph{if} that contains a
\code{useDynLib} directive, calls in top-level functions in the
package are not reported as their symbols will be preferentially
looked up in the DLL named in the first \code{useDynLib} directive.
\end{Details}
%
\begin{Value}
An object of class \code{"checkFF"}, which currently is a list of the
(parsed) foreign function calls with a character first argument and
no \code{PACKAGE} argument.

There is a \code{print} method to display the information contained in
such objects.
\end{Value}
%
\begin{SeeAlso}\relax
\code{\LinkA{.C}{.C}},
\code{\LinkA{.Fortran}{.Fortran}};
\code{\LinkA{Foreign}{Foreign}}.
\end{SeeAlso}
%
\begin{Examples}
\begin{ExampleCode}
 # order is pretty much random
checkFF(package = "stats", verbose = TRUE)
\end{ExampleCode}
\end{Examples}
\HeaderA{checkMD5sums}{Check and Create MD5 Checksum Files}{checkMD5sums}
\keyword{utilities}{checkMD5sums}
%
\begin{Description}\relax
\code{checkMD5sums} checks the files against a file \file{MD5}.
\end{Description}
%
\begin{Usage}
\begin{verbatim}
checkMD5sums(package, dir)
\end{verbatim}
\end{Usage}
%
\begin{Arguments}
\begin{ldescription}
\item[\code{package}] the name of an installed package
\item[\code{dir}] the path to the top-level directory of an installed
package.
\end{ldescription}
\end{Arguments}
%
\begin{Details}\relax
The file \file{MD5} which is created is in a format which can be
checked by \command{md5sum -c MD5} if a suitable command-line version
of \command{md5sum} is available.  (For Windows, one is supplied in
the bundle at \url{http://www.murdoch-sutherland.com/Rtools}.)

If \code{dir} is missing, an installed package of name \code{package}
is searched for.

The private function \code{tools:::.installMD5sums} is used to create
\code{MD5} files in the Windows build.
\end{Details}
%
\begin{Value}
\code{checkMD5sums} returns a logical, \code{NA} if there is no
\file{MD5} file to be checked.
\end{Value}
%
\begin{SeeAlso}\relax
\code{\LinkA{md5sum}{md5sum}}
\end{SeeAlso}
\HeaderA{checkRd}{ Check an Rd Object }{checkRd}
\keyword{documentation}{checkRd}
%
\begin{Description}\relax
Check an help file or the output of the \code{\LinkA{parse\_Rd}{parse.Rul.Rd}} function.
\end{Description}
%
\begin{Usage}
\begin{verbatim}
checkRd(Rd, defines = .Platform$OS.type, stages = "render",
        unknownOK = TRUE, listOK = TRUE, ..., def_enc = FALSE)
\end{verbatim}
\end{Usage}
%
\begin{Arguments}
\begin{ldescription}
\item[\code{Rd}]  a filename or \code{Rd} object to use as input. 
\item[\code{defines}]  string(s) to use in \AsIs{\#ifdef} tests. 
\item[\code{stages}]  at which stage (\code{"build"}, \code{"install"}, or
\code{"render"}) should \AsIs{\bsl{}Sexpr} macros be executed? See the
notes below.
\item[\code{unknownOK}]  unrecognized macros are treated as errors if
\code{FALSE}, otherwise warnings. 
\item[\code{listOK}]  unnecessary non-empty braces (e.g., around text, not as
an argument) are treated as errors if \code{FALSE}, otherwise
warnings.
\item[\code{...}]  additional parameters to pass to \code{\LinkA{parse\_Rd}{parse.Rul.Rd}} when
\code{Rd} is a filename.  One that is often useful is \code{encoding}.
\item[\code{def\_enc}] logical: has the package declared an encoding, so tests
for non-ASCII text are suppressed?
\end{ldescription}
\end{Arguments}
%
\begin{Details}\relax
\code{checkRd} performs consistency checks on an Rd file, confirming that
required sections are present, etc.

It accepts a filename for an Rd file, and will use
\code{\LinkA{parse\_Rd}{parse.Rul.Rd}} to parse it before applying the checks.  If so,
warnings from \code{parse\_Rd} are collected, together with those from
the internal function \code{prepare\_Rd}, which does the
\AsIs{\#ifdef} and \AsIs{\bsl{}Sexpr} processing, drops sections that
would not be rendered or are duplicated (and should not be) and
removes empty sections.

An Rd object is passed through \code{prepare\_Rd}, but it may already
have been (and installed Rd objects have).

Warnings are given a `level': those from \code{prepare\_Rd} have
level 0.  These include

\Tabular{l}{
All text must be in a section\\{}
Only one \var{tag name} section is allowed: the first will be used\\{}
Section \var{name} is unrecognized and will be dropped\\{}
Dropping empty section \var{name}\\{}
}
\code{checkRd} itself can show

\Tabular{rl}{
7 & Tag \var{tag name} not recognized\\{}
7 & \AsIs{\bsl{}tabular} format must be simple text\\{}
7 & Unrecognized \AsIs{\bsl{}tabular} format: \ldots\\{}
7 & Only \var{n} columns allowed in this table\\{}
7 & Must have a \var{tag name}\\{}
7 & Only one \var{tag name} is allowed\\{}
7 & Tag \var{tag name} must not be empty\\{}
7 & \AsIs{\bsl{}docType} must be plain text\\{}
5 & Tag \AsIs{\bsl{}method} is only valid in \AsIs{\bsl{}usage}\\{}
5 & Tag \AsIs{\bsl{}dontrun} is only valid in \AsIs{\bsl{}examples}\\{}
5 & Tag \var{tag name} is invalid in a \var{block name} block\\{}
5 & Title of \AsIs{\bsl{}section} must be non-empty plain text\\{}
5 & \AsIs{\bsl{}title} content must be plain text\\{}
3 & Empty section \var{tag name}\\{}
-1 & Non-ASCII contents without declared encoding\\{}
-1 & Non-ASCII contents in second part of \AsIs{\bsl{}enc}\\{} 
-3 & Tag \AsIs{...} is not valid in a code block\\{}
-3 & Apparent non-ASCII contents without declared encoding\\{}
-3 & Apparent non-ASCII contents in second part of \AsIs{\bsl{}enc}\\{}
-3 & Unnecessary braces at \ldots\\{}
-3 & \AsIs{\bsl{}method} not valid outside a code block\\{}
}
and variations with \AsIs{\bsl{}method} replaced by \AsIs{\bsl{}S3method} or
\AsIs{\bsl{}S4method}.

Note that both \code{prepare\_Rd} and \code{checkRd} have tests for an
empty section: that in \code{checkRd} is stricter (essentially that
nothing is output).
\end{Details}
%
\begin{Value}
This may fail through an \R{} error, but otherwise warnings are
collected as returned as an object of class \code{"checkRd"}, a
character vector of messages.  This class has a \code{print} method
which only prints unique messages, and has argument \code{minlevel}
that can be used to select only more serious messages.  (This is set
to \code{-1} in \command{R CMD check}.)

Possible fatal errors are those from running the parser (e.g. a
non-existent file, unclosed quoted string, non-ASCII input without a
specified encoding) or from \code{prepare\_Rd} (multiple
\AsIs{\bsl{}Rdversion} declarations, invalid \AsIs{\bsl{}encoding} or
\AsIs{\bsl{}docType} or \AsIs{\bsl{}name} sections, and missing or duplicate
\AsIs{\bsl{}name} or \AsIs{\bsl{}title} sections).
\end{Value}
%
\begin{Author}\relax
Duncan Murdoch, Brian Ripley
\end{Author}
%
\begin{SeeAlso}\relax
\code{\LinkA{parse\_Rd}{parse.Rul.Rd}}, \code{\LinkA{Rd2HTML}{Rd2HTML}}.
\end{SeeAlso}
\HeaderA{checkRdaFiles}{Report on Details of Saved Images or Re-saves them}{checkRdaFiles}
\aliasA{resaveRdaFiles}{checkRdaFiles}{resaveRdaFiles}
\keyword{utilities}{checkRdaFiles}
%
\begin{Description}\relax
This reports for each of the files produced by \code{save} the size,
if it was saved in ASCII or XDR binary format, and if it was
compressed (and if so in what format).

Usually such files have extension \file{.rda} or \file{.RData}, hence
the name of the function. 
\end{Description}
%
\begin{Usage}
\begin{verbatim}
checkRdaFiles(paths)
resaveRdaFiles(paths, compress = c("auto", "gzip", "bzip2", "xz"),
               compression_level)
\end{verbatim}
\end{Usage}
%
\begin{Arguments}
\begin{ldescription}
\item[\code{paths}] A character vector of paths to \code{save} files.  If
this specifies a single directory, it is taken to refer to all
\file{.rda} and \file{.RData} files in that directory.

\item[\code{compress, compression\_level}] type and level of compression: see
\code{\LinkA{save}{save}}.
\end{ldescription}
\end{Arguments}
%
\begin{Details}\relax
\code{compress = "auto"} asks \R{} to choose the compression and ignores
\code{compression\_level}.  It will try \code{"gzip"}, \code{"bzip2"}
and if the \code{"gzip"} compressed size is over 10Kb, \code{"xz"} and
choose the smallest compressed file (but with a 10\% bias towards
\code{"gzip"}).  This can be slow.
\end{Details}
%
\begin{Value}
For \code{checkRdaFiles}, a data frame with rows names \code{paths}
and columns
\begin{ldescription}
\item[\code{size}] numeric: file size in bytes, \code{NA} if the file does
not exist.
\item[\code{ASCII}] logical: true for save(ASCII = TRUE), \code{NA} if the
format is not that of an \R{} save file.
\item[\code{compress}] character: type of compression.  One of \code{"gzip"},
\code{"bzip2"}, \code{"xz"}, \code{"none"} or \code{"unknown"} (which
means that if this is an \R{} save file it is from a later version of
\R{}).
\item[\code{version}] integer: the version of the save -- usually \code{2}
but \code{1} for very old files, and \code{NA} for other files.
\end{ldescription}
\end{Value}
%
\begin{Examples}
\begin{ExampleCode}
## Not run: 
## from a package top-level source directory
paths <- sort(Sys.glob(c("data/*.rda", "data/*.RData")))
(res <- checkRdaFiles(paths))
## pick out some that may need attention
bad <- is.na(res$ASCII) | res$ASCII | (res$size > 1e4 & res$compress == "none")
res[bad, ]

## End(Not run)
\end{ExampleCode}
\end{Examples}
\HeaderA{checkTnF}{Check R Packages or Code for T/F}{checkTnF}
\aliasA{print.checkTnF}{checkTnF}{print.checkTnF}
\keyword{utilities}{checkTnF}
\keyword{documentation}{checkTnF}
%
\begin{Description}\relax
Checks the specified R package or code file for occurrences of
\code{T} or \code{F}, and gathers the expression containing these.
This is useful as in R \code{T} and \code{F} are just variables which
are set to the logicals \code{TRUE} and \code{FALSE} by default, but
are not reserved words and hence can be overwritten by the user.
Hence, one should always use \code{TRUE} and \code{FALSE} for the
logicals.
\end{Description}
%
\begin{Usage}
\begin{verbatim}
checkTnF(package, dir, file, lib.loc = NULL)
\end{verbatim}
\end{Usage}
%
\begin{Arguments}
\begin{ldescription}
\item[\code{package}] a character string naming an installed package.  If
given, the installed R code and the examples in the documentation
files of the package are checked.  R code installed as an image file
cannot be checked.
\item[\code{dir}] a character string specifying the path to a package's root
source directory.  This must contain the subdirectory \file{R} (for
R code), and should also contain \file{man} (for documentation).
Only used if \code{package} is not given.  If used, the R code files
and the examples in the documentation files are checked.
\item[\code{file}] the name of a file containing R code to be checked.  Used
if neither \code{package} nor \code{dir} are given.
\item[\code{lib.loc}] a character vector of directory names of \R{} libraries,
or \code{NULL}.  The default value of \code{NULL} corresponds to all
libraries currently known.  The specified library trees are used to
search for \code{package}.
\end{ldescription}
\end{Arguments}
%
\begin{Value}
An object of class \code{"checkTnF"} which is a list containing, for
each file where occurrences of \code{T} or \code{F} were found, a list
with the expressions containing these occurrences.  The names of the
list are the corresponding file names.

There is a \code{print} method for nicely displaying the information
contained in such objects.
\end{Value}
\HeaderA{checkVignettes}{Check Package Vignettes}{checkVignettes}
\aliasA{print.checkVignettes}{checkVignettes}{print.checkVignettes}
\keyword{utilities}{checkVignettes}
\keyword{documentation}{checkVignettes}
%
\begin{Description}\relax
Check all \code{\LinkA{Sweave}{Sweave}} files of a package by running
\code{\LinkA{Sweave}{Sweave}} and/or \code{\LinkA{Stangle}{Stangle}} on them. All
R source code files found after the tangling step are
\code{\LinkA{source}{source}}ed to check whether all code can be
executed without errors.
\end{Description}
%
\begin{Usage}
\begin{verbatim}
checkVignettes(package, dir, lib.loc = NULL,
               tangle = TRUE, weave = TRUE, latex = FALSE,
               workdir = c("tmp", "src", "cur"),
               keepfiles = FALSE)
\end{verbatim}
\end{Usage}
%
\begin{Arguments}
\begin{ldescription}
\item[\code{package}] a character string naming an installed package.  If
given, Sweave files are searched in subdirectory \file{doc}.
\item[\code{dir}] a character string specifying the path to a package's root
source directory.  This subdirectory \file{inst/doc} is searched for
Sweave files.
\item[\code{lib.loc}] a character vector of directory names of \R{} libraries,
or \code{NULL}.  The default value of \code{NULL} corresponds to all
libraries currently known.  The specified library trees are used to
search for \code{package}.
\item[\code{tangle}] Perform a tangle and \code{\LinkA{source}{source}} the extracted code?
\item[\code{weave}] Perform a weave?
\item[\code{latex}] logical: if \code{weave} and \code{latex}
are \code{TRUE} and there is no \file{Makefile} in the vignettes
directory, run the weaved files through \command{pdflatex}.
\item[\code{workdir}] Directory used as working directory while checking the
vignettes. If \code{"tmp"} then a temporary directory is created,
this is the default.  If \code{"src"} then the directory containing
the vignettes itself is used, if \code{"cur"} then the current
working directory of \R{} is used.

\item[\code{keepfiles}] Delete files in the temporary directory?  This option is
ignored when \code{workdir != "tmp"}.
\end{ldescription}
\end{Arguments}
%
\begin{Details}\relax
A `vignette' is a file in the package's \file{inst/doc} directory
with extension \file{.Rnw} (preferred), \file{.Snw}, \file{.Rtex} or
\file{.Stex} (and lower-case versions are also accepted).

If \code{tangle} is true, this function runs \code{\LinkA{Stangle}{Stangle}} to
produce (one or more) \R{} code files from each vignette, then
\code{source}s each code file in turn.

If \code{weave} is true, the vignettes are run through
\code{\LinkA{Sweave}{Sweave}}, which will produce a \file{.tex} file for each
vignette.  If \code{latex} is also true, \code{\LinkA{texi2pdf}{texi2pdf}} is run
on the \file{.tex} files from those vignettes which did not give
errors in the previous steps.
\end{Details}
%
\begin{Value}
An object of class \code{"checkVignettes"}, which is a list with the
error messages found during the tangle, source, weave and latex
steps. There is a \code{print} method for displaying the
information contained in such objects.
\end{Value}
%
\begin{Note}\relax
Prior to \R{} 2.13.0 this was the code used by \command{R CMD check},
but the latter is now more careful, running the code for each vignette
in a separate \R{} session.
\end{Note}
\HeaderA{codoc}{Check Code/Documentation Consistency}{codoc}
\aliasA{codocClasses}{codoc}{codocClasses}
\aliasA{codocData}{codoc}{codocData}
\aliasA{print.codoc}{codoc}{print.codoc}
\aliasA{print.codocClasses}{codoc}{print.codocClasses}
\aliasA{print.codocData}{codoc}{print.codocData}
\keyword{documentation}{codoc}
%
\begin{Description}\relax
Find inconsistencies between actual and documented `structure'
of \R{} objects in a package.  \code{codoc} compares names and
optionally also corresponding positions and default values of the
arguments of functions.  \code{codocClasses} and \code{codocData}
compare slot names of S4 classes and variable names of data sets,
respectively.
\end{Description}
%
\begin{Usage}
\begin{verbatim}
codoc(package, dir, lib.loc = NULL,
      use.values = NULL, verbose = getOption("verbose"))
codocClasses(package, lib.loc = NULL)
codocData(package, lib.loc = NULL)
\end{verbatim}
\end{Usage}
%
\begin{Arguments}
\begin{ldescription}
\item[\code{package}] a character string naming an installed package.
\item[\code{dir}] a character string specifying the path to a package's root
source directory.  This must contain the subdirectories \file{man}
with \R{} documentation sources (in Rd format) and \file{R} with \R{}
code.  Only used if \code{package} is not given.
\item[\code{lib.loc}] a character vector of directory names of \R{} libraries,
or \code{NULL}.  The default value of \code{NULL} corresponds to all
libraries currently known.  The specified library trees are used to
search for \code{package}.
\item[\code{use.values}] if \code{FALSE}, do not use function default values
when comparing code and docs.  Otherwise, compare \emph{all} default
values if \code{TRUE}, and only the ones documented in the usage
otherwise (default).
\item[\code{verbose}] a logical.  If \code{TRUE}, additional diagnostics are
printed.
\end{ldescription}
\end{Arguments}
%
\begin{Details}\relax
The purpose of \code{codoc} is to check whether the documented usage
of function objects agrees with their formal arguments as defined in
the \R{} code.  This is not always straightforward, in particular as the
usage information for methods to generic functions often employs the
name of the generic rather than the method.

The following algorithm is used.  If an installed package is used, it
is loaded (unless it is the \pkg{base} package), after possibly
detaching an already loaded version of the package.  Otherwise, if the
sources are used, the \R{} code files of the package are collected and
sourced in a new environment.  Then, the usage sections of the Rd
files are extracted and parsed `as much as possible' to give
the formals documented.  For interpreted functions in the code
environment, the formals are compared between code and documentation
according to the values of the argument \code{use.values}.  Synopsis
sections are used if present; their occurrence is reported if
\code{verbose} is true.

If a package has a namespace both exported and unexported objects are
checked, as well as registered S3 methods.  (In the unlikely event of
differences the order is exported objects in the package, registered
S3 methods and finally objects in the namespace and only the first
found is checked.)

Currently, the R documentation format has no high-level markup for the
basic `structure' of classes and data sets (similar to the usage
sections for function synopses).  Variable names for data frames in
documentation objects obtained by suitably editing `templates'
created by \code{\LinkA{prompt}{prompt}} are recognized by \code{codocData}
and used provided that the documentation object is for a single data
frame (i.e., only has one alias).  \code{codocClasses} analogously
handles slot names for classes in documentation objects obtained by
editing shells created by \code{\LinkA{promptClass}{promptClass}}.

Help files named \file{\var{pkgname}-defunct.Rd} for the
appropriate \var{pkgname} are checked more loosely, as they may
have undocumented arguments.
\end{Details}
%
\begin{Value}
\code{codoc} returns an object of class \code{"codoc"}.  Currently,
this is a list which, for each Rd object in the package where an
inconsistency was found, contains an element with a list of the
mismatches (which in turn are lists with elements \code{code} and
\code{docs}, giving the corresponding arguments obtained from the
function's code and documented usage).

\code{codocClasses} and \code{codocData} return objects of class
\code{"codocClasses"} and \code{"codocData"}, respectively, with a
structure similar to class \code{"codoc"}.

There are \code{print} methods for nicely displaying the information
contained in such objects.
\end{Value}
%
\begin{Note}\relax
The default for \code{use.values} has been changed from
\code{FALSE} to \code{NULL}, for \R{} versions 1.9.0 and later.
\end{Note}
%
\begin{SeeAlso}\relax
\code{\LinkA{undoc}{undoc}}, \code{\LinkA{QC}{QC}}
\end{SeeAlso}
\HeaderA{compactPDF}{Compact PDF Files}{compactPDF}
\aliasA{format.compactPDF}{compactPDF}{format.compactPDF}
\keyword{utilities}{compactPDF}
%
\begin{Description}\relax
Re-save PDF files (especially vignettes) more compactly.
Support function for \command{R CMD build --compact-vignettes}.
\end{Description}
%
\begin{Usage}
\begin{verbatim}
compactPDF(paths,
           qpdf = Sys.which(Sys.getenv("R_QPDF", "qpdf")),
           gs_cmd = Sys.getenv("R_GSCMD", ""),
           gs_quality = Sys.getenv("GS_QUALITY", "none"),
           gs_extras = character())

## S3 method for class 'compactPDF'
format(x, ratio = 0.9, diff = 1e4, ...)
\end{verbatim}
\end{Usage}
%
\begin{Arguments}
\begin{ldescription}
\item[\code{paths}] A character vector of paths to PDF files, or a length-one
character vector naming a directory, when all \file{.pdf} files in
that directory will be used.
\item[\code{qpdf}] Character string giving the path to the \command{qpdf}
command.  If empty, \command{qpdf} will not be used.
\item[\code{gs\_cmd}] Character string giving the path to the GhostScript
executable, if that is to be used.  On Windows this is the path to
\file{gswin32c.exe} or \file{gswin64c.exe}.  If \code{""}, the
function will try to find a platform-specific path to GhostScript.
\item[\code{gs\_quality}] A character string indicating the quality required:
the options are \code{"none"} (so GhostScript is not used),
\code{"printer"} (300dpi), \code{"ebook"} (150dpi) and
\code{"screen"} (72dpi).
\item[\code{gs\_extras}] An optional character vector of further options to be
passed to GhostScript.
\item[\code{x}] An object of class \code{"compactPDF"}.
\item[\code{ratio, diff}] Limits for reporting: files are only reported whose
sizes are reduced both by a factor of \code{ratio} and by
\code{diff} bytes.
\item[\code{...}] Further arguments to be passed to or from other methods.
\end{ldescription}
\end{Arguments}
%
\begin{Details}\relax
This by default makes use of \command{qpdf}, available from
\url{http://qpdf.sourceforge.net/} (including as a Windows binary) and
included with the CRAN Mac OS X distribution of \R{}.  If \code{gs\_cmd}
is non-empty and \code{gs\_quality != "none"}, GhostScript will used
first, then \command{qpdf} if it is available.  If \code{gs\_quality !=
  "none"} and \code{gs\_cmd} is empty, an attempt will be made to find a
GhostScript executable.

\command{qpdf} and/or \command{gs\_cmd} are run on all PDF files found,
and those which are reduced in size by at least 10\% and 10Kb are
replaced.

The strategy of our use of \command{qpdf} is to (losslessly) compress
both PDF streams and objects.  Ghostscript compresses streams and more
(including downsampling and compressing embedded images) and
consequently is much slower and may lose quality (but can also produce
much smaller PDF files).  However, quality \code{"ebook"} is perfectly
adequate for screen viewing and printing on laser printers.

Where PDF files are changed they will become PDF version 1.5 files:
these have been supported by Acrobat Reader since version 6 in 2003,
so this is very unlikely to cause difficulties.

Stream compression is what most often has large gains: \R{}'s
\code{\LinkA{pdf}{pdf}} device prior to \R{} 2.14.0 did not use it, and
older vignette PDFs often benefit from being re-generated.  Most PDF
documents are generated with object compression, but this does not
seem to be the default for MiKTeX's \command{pdflatex}.  For some PDF
files (and especially package vignettes), using GhostScript can
dramatically reduce the space taken by embedded images (often screenshots).

Where both Ghostscript and \command{qpdf} are selected (when
\code{gs\_quality != "none"} and both executables are found), they are
run in that order and the size reductions apply to the total
compression achieved.
\end{Details}
%
\begin{Value}
An object of class \code{c("compactPDF", "data.frame")}.
This has two columns, the old and new sizes in bytes for the files
that were changed.

There are \code{format} and \code{print} methods: the latter passes
\code{...} to the format method, so will accept \code{ratio} and
\code{diff} arguments.
\end{Value}
%
\begin{Note}\relax
The external tools used may change in future releases.
\end{Note}
%
\begin{SeeAlso}\relax
\code{\LinkA{resaveRdaFiles}{resaveRdaFiles}}.

Many other (and sometimes more effective) tools to compact PDF files
are available, including Adobe Acrobat (not Reader).
See the `Writing R Extensions' manual.
\end{SeeAlso}
\HeaderA{delimMatch}{Delimited Pattern Matching}{delimMatch}
\keyword{character}{delimMatch}
%
\begin{Description}\relax
Match delimited substrings in a character vector, with proper nesting.
\end{Description}
%
\begin{Usage}
\begin{verbatim}
delimMatch(x, delim = c("{", "}"), syntax = "Rd")
\end{verbatim}
\end{Usage}
%
\begin{Arguments}
\begin{ldescription}
\item[\code{x}] a character vector.
\item[\code{delim}] a character vector of length 2 giving the start and end
delimiters.  Future versions might allow for arbitrary regular
expressions.
\item[\code{syntax}] currently, always the string \code{"Rd"} indicating Rd
syntax (i.e., \samp{\%} starts a comment extending till the end of
the line, and \samp{\bsl{}} escapes).  Future versions might know about
other syntax, perhaps via `syntax tables' allowing to
flexibly specify comment, escape, and quote characters.
\end{ldescription}
\end{Arguments}
%
\begin{Value}
An integer vector of the same length as \code{x} giving the starting
position (in characters) of the first match, or \eqn{-1}{} if there is
none, with attribute \code{"match.length"} giving the length (in
characters) of the matched text (or \eqn{-1}{} for no match).
\end{Value}
%
\begin{SeeAlso}\relax
\code{\LinkA{regexpr}{regexpr}} for `simple' pattern matching.
\end{SeeAlso}
%
\begin{Examples}
\begin{ExampleCode}
x <- c("\\value{foo}", "function(bar)")
delimMatch(x)
delimMatch(x, c("(", ")"))
\end{ExampleCode}
\end{Examples}
\HeaderA{dependsOnPkgs}{Find Reverse Dependencies}{dependsOnPkgs}
\keyword{utilities}{dependsOnPkgs}
%
\begin{Description}\relax
Find `reverse' dependencies of packages, that is those packages
which depend on this one, and (optionally) so on recursively.
\end{Description}
%
\begin{Usage}
\begin{verbatim}
dependsOnPkgs(pkgs,
              dependencies = c("Depends", "Imports", "LinkingTo"),
              recursive = TRUE, lib.loc = NULL,
          installed = installed.packages(lib.loc, fields = "Enhances"))
\end{verbatim}
\end{Usage}
%
\begin{Arguments}
\begin{ldescription}
\item[\code{pkgs}] a character vector of package names.
\item[\code{dependencies}] a character vector listing the types of
dependencies, a subset of \code{c("Depends", "Imports", "LinkingTo",
      "Suggests", "Enhances")}.  Character string \code{"all"} is
shorthand for that vector.
\item[\code{recursive}] logical: should reverse dependencies of reverse
dependencies (and so on) be included?
\item[\code{lib.loc}] a character vector of \R{} library trees, or \code{NULL}
for all known trees (see \code{\LinkA{.libPaths}{.libPaths}}).
\item[\code{installed}] a result of calling \code{\LinkA{installed.packages}{installed.packages}}.
\end{ldescription}
\end{Arguments}
%
\begin{Value}
A character vector of package names, which does not include any from
\code{pkgs}.
\end{Value}
%
\begin{Examples}
\begin{ExampleCode}
## there are few dependencies in a vanilla R installation
dependsOnPkgs("lattice")
\end{ExampleCode}
\end{Examples}
\HeaderA{encoded\_text\_to\_latex}{Translate non-ASCII Text to LaTeX Escapes}{encoded.Rul.text.Rul.to.Rul.latex}
\keyword{utilities}{encoded\_text\_to\_latex}
%
\begin{Description}\relax
Translate non-ASCII characters in text to LaTeX escape sequences.
\end{Description}
%
\begin{Usage}
\begin{verbatim}
encoded_text_to_latex(x,
                      encoding = c("latin1", "latin2", "latin9",
                                   "UTF-8", "utf8"))
\end{verbatim}
\end{Usage}
%
\begin{Arguments}
\begin{ldescription}
\item[\code{x}] a character vector.
\item[\code{encoding}] the encoding to be assumed.  \code{"latin9"} is
officially ISO-8859-15 or Latin-9, but known as latin9 to LaTeX's
\code{inputenc} package.
\end{ldescription}
\end{Arguments}
%
\begin{Details}\relax
Non-ASCII characters in \code{x} are replaced by an appropriate LaTeX
escape sequence, or \samp{?} if there is no appropriate sequence.

Even if there is an appropriate sequence, it may not be supported by
the font in use.  Hyphen is mapped to \samp{\bsl{}-}.
\end{Details}
%
\begin{Value}
A character vector of the same length as \code{x}.
\end{Value}
%
\begin{SeeAlso}\relax
\code{\LinkA{iconv}{iconv}}
\end{SeeAlso}
%
\begin{Examples}
\begin{ExampleCode}
x <- "fa\xE7ile"
encoded_text_to_latex(x, "latin1")
## Not run: 
## create a tex file to show the upper half of 8-bit charsets
x <- rawToChar(as.raw(160:255), multiple=TRUE)
(x <- matrix(x, ncol=16, byrow=TRUE))
xx <- x
xx[] <- encoded_text_to_latex(x, "latin1") # or latin2 or latin9
xx <- apply(xx, 1, paste, collapse="&")
con <- file("test-encoding.tex", "w")
header <- c(
"\\documentclass{article}",
"\\usepackage[T1]{fontenc}",
"\\usepackage{Rd}",
"\\begin{document}",
"\\HeaderA{test}{}{test}",
"\\begin{Details}\relax",
"\\Tabular{cccccccccccccccc}{")
trailer <- c("}", "\\end{Details}", "\\end{document}")
writeLines(header, con)
writeLines(paste(xx, "\\", sep=""), con)
writeLines(trailer, con)
close(con)
## and some UTF_8 chars
x <- intToUtf8(as.integer(
    c(160:383,0x0192,0x02C6,0x02C7,0x02CA,0x02D8,
      0x02D9, 0x02DD, 0x200C, 0x2018, 0x2019, 0x201C,
      0x201D, 0x2020, 0x2022, 0x2026, 0x20AC)),
               multiple=TRUE)
x <- matrix(x, ncol=16, byrow=TRUE)
xx <- x
xx[] <- encoded_text_to_latex(x, "UTF-8")
xx <- apply(xx, 1, paste, collapse="&")
con <- file("test-utf8.tex", "w")
writeLines(header, con)
writeLines(paste(xx, "\\", sep=""), con)
writeLines(trailer, con)
close(con)

## End(Not run)
\end{ExampleCode}
\end{Examples}
\HeaderA{fileutils}{File Utilities}{fileutils}
\aliasA{file\_ext}{fileutils}{file.Rul.ext}
\aliasA{file\_path\_as\_absolute}{fileutils}{file.Rul.path.Rul.as.Rul.absolute}
\aliasA{file\_path\_sans\_ext}{fileutils}{file.Rul.path.Rul.sans.Rul.ext}
\aliasA{list\_files\_with\_exts}{fileutils}{list.Rul.files.Rul.with.Rul.exts}
\aliasA{list\_files\_with\_type}{fileutils}{list.Rul.files.Rul.with.Rul.type}
\keyword{file}{fileutils}
%
\begin{Description}\relax
Utilities for listing files, and manipulating file paths.
\end{Description}
%
\begin{Usage}
\begin{verbatim}
file_ext(x)
file_path_as_absolute(x)
file_path_sans_ext(x, compression = FALSE)

list_files_with_exts(dir, exts, all.files = FALSE,
                     full.names = TRUE)
list_files_with_type(dir, type, all.files = FALSE,
                     full.names = TRUE, OS_subdirs = .OStype())
\end{verbatim}
\end{Usage}
%
\begin{Arguments}
\begin{ldescription}
\item[\code{x}] character vector giving file paths.
\item[\code{compression}] logical: should compression extension \file{.gz},
\file{.bz2} or \file{.xz} be removed first?
\item[\code{dir}] a character string with the path name to a directory.
\item[\code{exts}] a character vector of possible file extensions.
\item[\code{all.files}] a logical.  If \code{FALSE} (default), only visible
files are considered; if \code{TRUE}, all files are used.
\item[\code{full.names}] a logical indicating whether the full paths of the
files found are returned (default), or just the file names.
\item[\code{type}] a character string giving the `type' of the files
to be listed, as characterized by their extensions.  Currently,
possible values are \code{"code"} (R code), \code{"data"} (data
sets), \code{"demo"} (demos), \code{"docs"} (R documentation), and
\code{"vignette"} (vignettes).
\item[\code{OS\_subdirs}] a character vector with the names of OS-specific
subdirectories to possibly include in the listing of R code and
documentation files.  By default, the value of the environment
variable \env{R\_OSTYPE}, or if this is empty, the value of
\code{\LinkA{.Platform}{.Platform}\$OS.type}, is used.
\end{ldescription}
\end{Arguments}
%
\begin{Details}\relax
\code{file\_ext} returns the file (name) extensions.
(Only purely alphanumeric extensions are recognized.)

\code{file\_path\_as\_absolute} turns a possibly relative file path
absolute, performing tilde expansion if necessary. As from \R{} 2.13.0
this is a wrapper for \code{\LinkA{normalizePath}{normalizePath}}.  Currently,
\code{x} must be a single existing path.

\code{file\_path\_sans\_ext} returns the file paths without extensions.
(Only purely alphanumeric extensions are recognized.)

\code{list\_files\_with\_exts} returns the paths or names of the files in
directory \code{dir} with extension matching one of the elements of
\code{exts}.  Note that by default, full paths are returned, and that
only visible files are used.

\code{list\_files\_with\_type} returns the paths of the files in \code{dir}
of the given `type', as determined by the extensions recognized
by \R{}.  When listing R code and documentation files, files in
OS-specific subdirectories are included if present according to the
value of \code{OS\_subdirs}.  Note that by default, full paths are
returned, and that only visible files are used.
\end{Details}
%
\begin{SeeAlso}\relax
\code{\LinkA{file.path}{file.path}},
\code{\LinkA{file.info}{file.info}},
\code{\LinkA{list.files}{list.files}}
\end{SeeAlso}
%
\begin{Examples}
\begin{ExampleCode}

dir <- file.path(R.home(), "library", "stats")
list_files_with_exts(file.path(dir, "demo"), "R")
list_files_with_type(file.path(dir, "demo"), "demo") # the same
file_path_sans_ext(list.files(file.path(R.home("modules"))))
\end{ExampleCode}
\end{Examples}
\HeaderA{getDepList}{Functions to Retrieve Dependency Information}{getDepList}
\aliasA{pkgDepends}{getDepList}{pkgDepends}
\keyword{utilities}{getDepList}
%
\begin{Description}\relax
Given a dependency matrix, will create a \code{DependsList} object for that
package which will include the dependencies for that matrix, which
ones are installed, which unresolved dependencies were found online,
which unresolved dependencies were not found online, and any R dependencies.
\end{Description}
%
\begin{Usage}
\begin{verbatim}
getDepList(depMtrx, instPkgs, recursive = TRUE, local = TRUE,
           reduce = TRUE, lib.loc = NULL)

pkgDepends(pkg, recursive = TRUE, local = TRUE, reduce = TRUE,
           lib.loc = NULL)
\end{verbatim}
\end{Usage}
%
\begin{Arguments}
\begin{ldescription}
\item[\code{depMtrx}] A dependency matrix as from \code{package.dependencies}
\item[\code{pkg}] The name of the package
\item[\code{instPkgs}] A matrix specifying all packages installed on the
local system, as from \code{installed.packages}
\item[\code{recursive}] Whether or not to include indirect dependencies
\item[\code{local}] Whether or not to search only locally
\item[\code{reduce}] Whether or not to collapse all sets of dependencies to a
minimal value
\item[\code{lib.loc}] What libraries to use when looking for installed
packages.  \code{NULL} indicates all library directories in the
user's \code{.libPaths()}.
\end{ldescription}
\end{Arguments}
%
\begin{Details}\relax
The function \code{pkgDepends} is a convenience function which wraps
\code{getDepList} and takes as input a package name.  It will then
query \code{\LinkA{installed.packages}{installed.packages}}
and also generate a dependency
matrix, calling \code{getDepList} with this information and returning
the result.

These functions will retrieve information about the dependencies of
the matrix, resulting in a \code{DependsList} object.  This is a
list with four elements:
\begin{description}

\item[Depends] A vector of the dependencies for this package.
\item[Installed] A vector of the dependencies which have been
satisfied by the currently installed packages.
\item[Found] A list representing the dependencies which are not in
\code{Installed} but were found online.  This list has element
names which are the URLs for the repositories in which packages
were found and the elements themselves are vectors of package
names which were found in the respective repositories.  If
\code{local=TRUE}, the \code{Found} element will always be empty.
\item[R] Any R version dependencies.

\end{description}


If \code{recursive} is \code{TRUE}, any package that is specified as a
dependency will in turn have its dependencies included (and so on),
these are known as indirect dependencies.  If \code{recursive} is 
\code{FALSE}, only the dependencies directly stated by the package will 
be used.

If \code{local} is \code{TRUE}, the system will only look at the user's local
install and not online to find unresolved dependencies.

If \code{reduce} is \code{TRUE}, the system will collapse the fields in the
\code{DependsList} object such that a minimal set of dependencies
are specified (for instance if there was
\samp{foo, foo (>= 1.0.0), foo (>= 1.3.0)}, it would only return
\samp{foo (>= 1.3.0)}).
\end{Details}
%
\begin{Value}
An object of class \code{"DependsList"}.
\end{Value}
%
\begin{Author}\relax
 Jeff Gentry 
\end{Author}
%
\begin{SeeAlso}\relax
\code{\LinkA{installFoundDepends}{installFoundDepends}}
\end{SeeAlso}
%
\begin{Examples}
\begin{ExampleCode}
pkgDepends("tools", local = FALSE)
\end{ExampleCode}
\end{Examples}
\HeaderA{HTMLheader}{Generate a standard HTML header for R help}{HTMLheader}
\keyword{utilities}{HTMLheader}
\keyword{documentation}{HTMLheader}
%
\begin{Description}\relax
This function generates the standard HTML header used on R help pages.
\end{Description}
%
\begin{Usage}
\begin{verbatim}
HTMLheader(title = "R", logo = TRUE, up = NULL,
           top = file.path(Rhome, "doc/html/index.html"), 
           Rhome = "",
           css = file.path(Rhome, "doc/html/R.css"),
           headerTitle = paste("R:", title),
           outputEncoding = "UTF-8")
\end{verbatim}
\end{Usage}
%
\begin{Arguments}
\begin{ldescription}
\item[\code{title}] 
The title to display and use in the HTML headers.  Should have had any
HTML escaping already done.

\item[\code{logo}] 
Whether to display the \R{} logo after the title.

\item[\code{up}] 
Which page (if any) to link to on the ``up'' button.

\item[\code{top}] 
Which page (if any) to link to on the ``top'' button.

\item[\code{Rhome}] 
A \bold{relative} path to the R home directory.  See the `Details'.

\item[\code{css}] 
The relative URL for the Cascading Style Sheet.

\item[\code{headerTitle}] 
The title used in the headers.

\item[\code{outputEncoding}] 
The declared encoding for the whole page.

\end{ldescription}
\end{Arguments}
%
\begin{Details}\relax
The \code{up} and \code{top} links should be relative to the current page.
The \code{Rhome} path default works with dynamic help; for static help, a
relative path (e.g. \file{../..}) to it should be used.
\end{Details}
%
\begin{Value}
A character vector containing the lines of an HTML header which can be used
to start a page in the R help system.
\end{Value}
%
\begin{Examples}
\begin{ExampleCode}
cat(HTMLheader("This is a sample header"), sep="\n")
\end{ExampleCode}
\end{Examples}
\HeaderA{HTMLlinks}{Collect HTML Links from Package Documentation}{HTMLlinks}
\aliasA{findHTMLlinks}{HTMLlinks}{findHTMLlinks}
\keyword{documentation}{HTMLlinks}
%
\begin{Description}\relax
Compute relative file paths for URLs to other package's installed HTML
documentation.
\end{Description}
%
\begin{Usage}
\begin{verbatim}
findHTMLlinks(pkgDir = "", lib.loc = NULL, level = 0:2)
\end{verbatim}
\end{Usage}
%
\begin{Arguments}
\begin{ldescription}
\item[\code{pkgDir}] the top-level directory of an installed package.  The
default indicates no package.
\item[\code{lib.loc}] character vector describing the location of \R{} library
trees to scan: the default indicates \code{\LinkA{.libPaths}{.libPaths}()}.
\item[\code{level}] Which level(s) to include.
\end{ldescription}
\end{Arguments}
%
\begin{Details}\relax
\code{findHTMLlinks} tries to resolve links from one help page to
another.  It uses in decreasing priority
\begin{itemize}

\item The package in \code{pkgDir}: this is used when converting
HTML help for that package (level 0).
\item The base and recommended packages (level 1).
\item Other packages found in the library trees specified by
\code{lib.loc} in the order of the trees and alphabetically within a
library tree (level 2).

\end{itemize}

\end{Details}
%
\begin{Value}
A named character vector of file paths, relative to the \file{html}
directory of an installed package.  So these are of the form
\file{"../../\var{somepkg}/html/\var{sometopic}.html"}.
\end{Value}
%
\begin{Author}\relax
Duncan Murdoch, Brian Ripley
\end{Author}
\HeaderA{installFoundDepends}{A function to install unresolved dependencies}{installFoundDepends}
\keyword{utilities}{installFoundDepends}
%
\begin{Description}\relax
This function will take the \code{Found} element of a
\code{pkgDependsList} object and attempt to install 
all of the listed packages from the specified repositories.
\end{Description}
%
\begin{Usage}
\begin{verbatim}
installFoundDepends(depPkgList, ...)
\end{verbatim}
\end{Usage}
%
\begin{Arguments}
\begin{ldescription}
\item[\code{depPkgList}] A \code{Found} element from a \code{pkgDependsList} object
\item[\code{...}] Arguments to pass on to \code{\LinkA{install.packages}{install.packages}}
\end{ldescription}
\end{Arguments}
%
\begin{Details}\relax
This function takes as input the \code{Found} list from a
\code{pkgDependsList} object.  This list will have element names being
URLs corresponding to repositories and the elements will be vectors of
package names.  For each element, \code{\LinkA{install.packages}{install.packages}} is
called for that URL to install all packages listed in the vector.
\end{Details}
%
\begin{Author}\relax
Jeff Gentry
\end{Author}
%
\begin{SeeAlso}\relax
\code{\LinkA{pkgDepends}{pkgDepends}}, \code{\LinkA{install.packages}{install.packages}}
\end{SeeAlso}
%
\begin{Examples}
\begin{ExampleCode}
## Set up a temporary directory to install packages to
tmp <- tempfile()
dir.create(tmp)

pDL <- pkgDepends("tools",local=FALSE)
installFoundDepends(pDL$Found, destdir=tmp)
\end{ExampleCode}
\end{Examples}
\HeaderA{md5sum}{Compute MD5 Checksums}{md5sum}
\keyword{utilities}{md5sum}
%
\begin{Description}\relax
Compute the 32-byte MD5 checksums of one or more files.
\end{Description}
%
\begin{Usage}
\begin{verbatim}
md5sum(files)
\end{verbatim}
\end{Usage}
%
\begin{Arguments}
\begin{ldescription}
\item[\code{files}] character. The paths of file(s) to be check-summed.
\end{ldescription}
\end{Arguments}
%
\begin{Value}
A character vector of the same length as \code{files}, with names
equal to \code{files}. The elements
will be \code{NA} for non-existent or unreadable files, otherwise
a 32-character string of hexadecimal digits.

On Windows all files are read in binary mode (as the \code{md5sum}
utilities there do): on other OSes the files are read in the default way.
\end{Value}
%
\begin{SeeAlso}\relax
\code{\LinkA{checkMD5sums}{checkMD5sums}}
\end{SeeAlso}
%
\begin{Examples}
\begin{ExampleCode}
as.vector(md5sum(dir(R.home(), pattern="^COPY", full.names=TRUE)))
\end{ExampleCode}
\end{Examples}
\HeaderA{package.dependencies}{Check Package Dependencies}{package.dependencies}
\keyword{utilities}{package.dependencies}
%
\begin{Description}\relax
Parses and checks the dependencies of a package against the currently
installed version of R (and other packages). 
\end{Description}
%
\begin{Usage}
\begin{verbatim}
package.dependencies(x, check = FALSE,
                     depLevel = c("Depends", "Imports", "Suggests"))
\end{verbatim}
\end{Usage}
%
\begin{Arguments}
\begin{ldescription}
\item[\code{x}] A matrix of package descriptions as returned by
\code{\LinkA{available.packages}{available.packages}}.
\item[\code{check}] If \code{TRUE}, return logical vector of check results.
If \code{FALSE}, return parsed list of dependencies.
\item[\code{depLevel}] Whether to look for \code{Depends} or \code{Suggests}
level dependencies.
\end{ldescription}
\end{Arguments}
%
\begin{Details}\relax
Currently we only check if the package conforms with the currently
running version of R.  In the future we might add checks for 
inter-package dependencies.
\end{Details}
%
\begin{SeeAlso}\relax
\code{\LinkA{update.packages}{update.packages}}
\end{SeeAlso}
\HeaderA{package\_dependencies}{Computations on the Dependency Hierarchy of Packages}{package.Rul.dependencies}
\keyword{utilities}{package\_dependencies}
%
\begin{Description}\relax
Find (recursively) dependencies or reverse dependencies of packages.
\end{Description}
%
\begin{Usage}
\begin{verbatim}
package_dependencies(packages = NULL, db, 
        which = c("Depends", "Imports", "LinkingTo"), 
	recursive = FALSE, reverse = FALSE)
\end{verbatim}
\end{Usage}
%
\begin{Arguments}
\begin{ldescription}
\item[\code{packages}] a character vector of package names.
\item[\code{db}] character matrix as from available.packages(), or data frame variants thereof.
Alternatively, a package database like the one available from
\url{http://cran.R-project.org/web/packages/packages.rds}.

\item[\code{which}] a character vector listing the types of
dependencies, a subset of \code{c("Depends", "Imports", "LinkingTo",
      "Suggests", "Enhances")}.  Character string \code{"all"} is
shorthand for that vector, character string \code{"most"} 
for the same vector without \code{"Enhances"}.

\item[\code{recursive}] logical: should (reverse) dependencies of (reverse)
dependencies (and so on) be included?

\item[\code{reverse}] logical: if \code{FALSE} (default), regular dependencies are calculated,
otherwise reverse dependencies.

\end{ldescription}
\end{Arguments}
%
\begin{Value}
Named list with one element for each package in argument \code{packages},
each consists of a character vector naming the (recursive) (reverse) dependencies
of that package.

For given packages which are not found in the db, \code{NULL} entries are returned, 
as opposed to \code{character(0)} entries which indicate no dependencies.
\end{Value}
%
\begin{SeeAlso}\relax
\code{\LinkA{dependsOnPkgs}{dependsOnPkgs}}, and \code{\LinkA{package.dependencies}{package.dependencies}} for checking
dependencies
\end{SeeAlso}
%
\begin{Examples}
\begin{ExampleCode}
## Not run: 
pdb <- available.packages()
deps <- package_dependencies(packages = "MASS", pdb,
                    which = c("Depends", "Imports", "LinkingTo", "Suggests"),
                    recursive = TRUE, reverse = TRUE)
length(deps$MASS)

## End(Not run)
\end{ExampleCode}
\end{Examples}
\HeaderA{parseLatex}{These experimental functions work with a subset of LaTeX code.}{parseLatex}
\aliasA{deparseLatex}{parseLatex}{deparseLatex}
\aliasA{latexToUtf8}{parseLatex}{latexToUtf8}
\keyword{utilities}{parseLatex}
\keyword{documentation}{parseLatex}
%
\begin{Description}\relax
The \code{parseLatex} function parses LaTeX source, producing a
structured object; \code{deparseLatex} reverses the process.  The
\code{latexToUtf8} function takes a LaTeX object, and processes a number
of different macros to convert them into the corresponding UTF-8
characters.
\end{Description}
%
\begin{Usage}
\begin{verbatim}
parseLatex(text, filename = deparse(substitute(text)), verbose = FALSE,
           verbatim = c("verbatim", "verbatim*", "Sinput", "Soutput"))
deparseLatex(x, dropBraces=FALSE)
latexToUtf8(x)
\end{verbatim}
\end{Usage}
%
\begin{Arguments}
\begin{ldescription}
\item[\code{text}] 
A character vector containing LaTeX source code.

\item[\code{filename}] 
A filename to use in syntax error messages.

\item[\code{verbose}] 
If \code{TRUE}, print debug error messages.

\item[\code{verbatim}] 
A character vector containing the names of LaTeX environments holding verbatim text.

\item[\code{x}] 
A \code{"LaTeX"} object.

\item[\code{dropBraces}] 
Drop unnecessary braces when displaying a \code{"LaTeX"} object.

\end{ldescription}
\end{Arguments}
%
\begin{Details}\relax

The parser does not recognize all legal LaTeX code, only relatively
simple examples.  It does not associate arguments with macros, that
needs to be done after parsing, with knowledge of the definitions of
each macro.  The main intention for this function is to process simple
LaTeX code used in bibliographic references, not fully general LaTeX
documents.

Verbose text is allowed in two forms:  the \AsIs{\bsl{}verb} macro (with
single character delimiters), and environments whose names are listed
in the \code{verbatim} argument.

\end{Details}
%
\begin{Value}
The \code{parseLatex()} function returns a recursive object of class
\code{"LaTeX"}.  Each of the entries in this object will have a \code{
"latex\_tag"} attribute identifying its syntactic role.

The \code{deparseLatex()} function returns a single element character
vector, possibly containing embedded newlines.

The \code{latexToUtf8()} function returns a modified version of the
\code{"LaTeX"} object that was passed to it.
\end{Value}
%
\begin{Author}\relax
Duncan Murdoch
\end{Author}
%
\begin{Examples}
\begin{ExampleCode}

latex <- parseLatex("fa\\c{c}ile")
deparseLatex(latexToUtf8(latex))
\end{ExampleCode}
\end{Examples}
\HeaderA{parse\_Rd}{ Parse an Rd file }{parse.Rul.Rd}
\aliasA{as.character.Rd}{parse\_Rd}{as.character.Rd}
\aliasA{print.Rd}{parse\_Rd}{print.Rd}
\keyword{utilities}{parse\_Rd}
\keyword{documentation}{parse\_Rd}
%
\begin{Description}\relax
This function reads an R documentation (Rd) file and parses it, for
processing by other functions.
\end{Description}
%
\begin{Usage}
\begin{verbatim}
parse_Rd(file, srcfile = NULL, encoding = "unknown",
         verbose = FALSE, fragment = FALSE, warningCalls = TRUE)
## S3 method for class 'Rd'
print(x, deparse = FALSE, ...)
## S3 method for class 'Rd'
as.character(x, deparse = FALSE, ...)
\end{verbatim}
\end{Usage}
%
\begin{Arguments}
\begin{ldescription}
\item[\code{file}] A filename or text-mode connection.  At present filenames
work best.
\item[\code{srcfile}] \code{NULL}, or a \code{"srcfile"} object.  See the
`Details' section.
\item[\code{encoding}] Encoding to be assumed for input strings.
\item[\code{verbose}] Logical indicating whether detailed parsing
information should be printed.
\item[\code{fragment}] Logical indicating whether file represents a complete
Rd file, or a fragment.
\item[\code{warningCalls}] Logical: should parser warnings include the call?
\item[\code{x}] An object of class Rd.
\item[\code{deparse}] If \code{TRUE}, attempt to reinstate the escape characters
so that the resulting characters will parse to the same object.
\item[\code{...}] Further arguments to be passed to or from other methods.
\end{ldescription}
\end{Arguments}
%
\begin{Details}\relax
This function parses \file{Rd} files according to the specification given
in \url{http://developer.r-project.org/parseRd.pdf}.

It generates a warning for each parse error and attempts to continue
parsing.  In order to continue, it is generally necessary to drop some
parts of the file, so such warnings should not be ignored.
\end{Details}
%
\begin{Value}
\code{parse\_Rd} returns an object of class \code{"Rd"}.  The
internal format of this object is subject to change.  The
\code{as.character()} and \code{print()} methods defined for the
class return character vectors and print them, respectively.

Files without a marked encoding are by default assumed to be in the
native encoding.  An alternate default can be set using the
\code{encoding} argument.  All text in files is translated to the
UTF-8 encoding in the parsed object.
\end{Value}
%
\begin{Author}\relax
 Duncan Murdoch 
\end{Author}
%
\begin{References}\relax
 \url{http://developer.r-project.org/parseRd.pdf} 
\end{References}
%
\begin{SeeAlso}\relax
\code{\LinkA{Rd2HTML}{Rd2HTML}} for the converters that use the output of
\code{parse\_Rd()}.
\end{SeeAlso}
\HeaderA{pskill}{Kill a Process}{pskill}
\aliasA{SIGCHLD}{pskill}{SIGCHLD}
\aliasA{SIGCONT}{pskill}{SIGCONT}
\aliasA{SIGHUP}{pskill}{SIGHUP}
\aliasA{SIGINT}{pskill}{SIGINT}
\aliasA{SIGKILL}{pskill}{SIGKILL}
\aliasA{SIGQUIT}{pskill}{SIGQUIT}
\aliasA{SIGSTOP}{pskill}{SIGSTOP}
\aliasA{SIGTERM}{pskill}{SIGTERM}
\aliasA{SIGTSTP}{pskill}{SIGTSTP}
\aliasA{SIGUSR1}{pskill}{SIGUSR1}
\aliasA{SIGUSR2}{pskill}{SIGUSR2}
\keyword{utilities}{pskill}
%
\begin{Description}\relax
\code{pskill} sends a signal to a process, usually to terminate it.
\end{Description}
%
\begin{Usage}
\begin{verbatim}
pskill(pid, signal = SIGTERM)

SIGHUP
SIGINT
SIGQUIT
SIGKILL
SIGTERM
SIGSTOP
SIGTSTP
SIGCHLD
SIGUSR1
SIGUSR2
\end{verbatim}
\end{Usage}
%
\begin{Arguments}
\begin{ldescription}
\item[\code{pid}] positive integers: one or more process IDs as returned by
\code{\LinkA{Sys.getpid}{Sys.getpid}}.
\item[\code{signal}] integer, most often one of the symbolic constants.
\end{ldescription}
\end{Arguments}
%
\begin{Details}\relax
Signals are a C99 concept, but only a small number are required to be
supported (of those listed, only \code{SIGINT} and \code{SIGTERM}).
They are much more widely used on POSIX operating systems (which
should define all of those listed here), which also support a
\code{kill} system call to send a signal to a process, most often to
terminate it.  Function \code{pskill} provides a wrapper: it silently
ignores invalid values of its arguments, including zero or negative pids.

In normal use on a Unix-alike, \kbd{Ctrl-C} sends \code{SIGINT},
\kbd{Ctrl-\bsl{}} sends \code{SIGQUIT} and \kbd{Ctrl-Z} sends
\code{SIGTSTP}: that and \code{SIGSTOP} suspend a process which can be
resumed by \code{SIGCONT}.

The signals are small integers, but the actual numeric values are not
standardized (and most do differ between OSes).  The \code{SIG*}
objects contain the appropriate integer values for the current platform
(or \code{NA\_INTEGER\_} if the signal is not defined).

Only \code{SIGINT} and \code{SIGKILL} will be defined on Windows, and
\code{pskill} will always use the Windows system call
\code{TerminateProcess}.
\end{Details}
%
\begin{Value}
A logical vector of the same length as \code{pid},
\code{TRUE} (for success) or \code{FALSE}, invisibly.
\end{Value}
%
\begin{SeeAlso}\relax
Package \pkg{parallel} has several means to launch child processes
which record the process IDs.

\code{\LinkA{psnice}{psnice}}
\end{SeeAlso}
%
\begin{Examples}
\begin{ExampleCode}
## Not run: 
pskill(c(237, 245), SIGKILL)

## End(Not run)
\end{ExampleCode}
\end{Examples}
\HeaderA{psnice}{Get or Set the Priority (Niceness) of a Process}{psnice}
\keyword{utility}{psnice}
%
\begin{Description}\relax
Get or set the `niceness' of the current process, or one or
more other processes.
\end{Description}
%
\begin{Usage}
\begin{verbatim}
psnice(pid = Sys.getpid(), value = NA_integer_)
\end{verbatim}
\end{Usage}
%
\begin{Arguments}
\begin{ldescription}
\item[\code{pid}] positive integers: the process IDs of one of more
processes: defaults to the \R{} session process.
\item[\code{value}] The niceness to be set, or \code{NA} for an enquiry.
\end{ldescription}
\end{Arguments}
%
\begin{Details}\relax
POSIX operating systems have a concept of process priorities, usually
from 0 to 39 (or 40) with 20 being a normal priority and (somewhat
confusingly) larger numeric values denoting lower priority.  To add to
the confusion, there is a `niceness' value, the amount by which
the priority numerically exceeds 20 (which can be negative).
Processes with high niceness will receive less CPU time than those
with normal priority.  On some OSes, processes with niceness
\code{+19} are only run when the system would otherwise be idle.

On many OSes utilities such as \command{top} report the priority and
not the niceness.  Niceness is used by the utility
\file{/usr/bin/renice}: \file{/usr/bin/nice} (and
\command{/usr/bin/renice -n}) specifies an \emph{increment} in niceness.

Only privileged users (usually super-users) can lower the niceness.

Windows has a slightly different concept of `priority
classes'.  We have mapped the idle priority to niceness \code{19},
`below normal' to \code{15}, normal to \code{0}, `above
normal' to \code{-5} and `realtime' to \code{-10}.  Unlike
Unix-alikes, a non-privileged user can increase the priority class on
Windows (but using `realtime' is inadvisable).
\end{Details}
%
\begin{Value}
An integer vector of \emph{previous} niceness values,
\code{NA} if unknown for any reason.
\end{Value}
%
\begin{SeeAlso}\relax
Various functions in package \pkg{parallel} create child processes
whose priority may need to be changed.

\code{\LinkA{pskill}{pskill}}.
\end{SeeAlso}
\HeaderA{QC}{QC Checks for R Code and/or Documentation}{QC}
\aliasA{checkDocFiles}{QC}{checkDocFiles}
\aliasA{checkDocStyle}{QC}{checkDocStyle}
\aliasA{checkReplaceFuns}{QC}{checkReplaceFuns}
\aliasA{checkS3methods}{QC}{checkS3methods}
\aliasA{print.checkDocFiles}{QC}{print.checkDocFiles}
\aliasA{print.checkDocStyle}{QC}{print.checkDocStyle}
\aliasA{print.checkReplaceFuns}{QC}{print.checkReplaceFuns}
\aliasA{print.checkS3methods}{QC}{print.checkS3methods}
\keyword{utilities}{QC}
\keyword{documentation}{QC}
%
\begin{Description}\relax
Functions for performing various quality checks.
\end{Description}
%
\begin{Usage}
\begin{verbatim}
checkDocFiles(package, dir, lib.loc = NULL)
checkDocStyle(package, dir, lib.loc = NULL)
checkReplaceFuns(package, dir, lib.loc = NULL)
checkS3methods(package, dir, lib.loc = NULL)
\end{verbatim}
\end{Usage}
%
\begin{Arguments}
\begin{ldescription}
\item[\code{package}] a character string naming an installed package.
\item[\code{dir}] a character string specifying the path to a package's root
source directory.  This should contain the subdirectories \file{R}
(for R code) and \file{man} with \R{} documentation sources (in Rd
format).  Only used if \code{package} is not given.
\item[\code{lib.loc}] a character vector of directory names of \R{} libraries,
or \code{NULL}.  The default value of \code{NULL} corresponds to all
libraries currently known.  The specified library trees are used to
search for \code{package}.
\end{ldescription}
\end{Arguments}
%
\begin{Details}\relax
\code{checkDocFiles} checks, for all Rd files in a package, whether all
arguments shown in the usage sections of the Rd file are documented in
its arguments section.  It also reports duplicated entries in the
arguments section, and `over-documented' arguments which are
given in the arguments section but not in the usage.  Note that the
match is for the usage section and not a possibly existing synopsis
section, as the usage is what gets displayed.

\code{checkDocStyle} investigates how (S3) methods are shown in the
usages of the Rd files in a package.  It reports the methods shown by
their full name rather than using the Rd \AsIs{\bsl{}method} markup for
indicating S3 methods.  Earlier versions of \R{} also reported about
methods shown along with their generic, which typically caused
problems for the documentation of the primary argument in the generic
and its methods.  With \AsIs{\bsl{}method} now being expanded in a way
that class information is preserved, joint documentation is
no longer necessarily a problem.  (The corresponding information is
still contained in the object returned by \code{checkDocStyle}.)

\code{checkReplaceFuns} checks whether replacement functions or S3/S4
replacement methods in the package R code have their final argument
named \code{value}.

\code{checkS3methods} checks whether all S3 methods defined in the
package R code have all arguments of the corresponding generic, with
positional arguments of the generics in the same positions for the
method.  As an exception, the first argument of a formula method
\emph{may} be called \code{formula} even if this is not the name used
by the generic.  The rules when \code{...} is involved are subtle:
see the source code.  Functions recognized as S3 generics are those
with a call to \code{UseMethod} in their body, internal S3 generics
(see \LinkA{InternalMethods}{InternalMethods}), and S3 group generics (see
\code{\LinkA{Math}{S3groupGeneric}}).  Possible dispatch under a
different name is not taken into account.  The generics are sought
first in the given package, then in the \pkg{base} package and
(currently) the packages \pkg{graphics}, \pkg{stats}, and \pkg{utils}
added in R 1.9.0 by splitting the former \pkg{base}, and, if an
installed package is tested, also in the loaded namespaces/packages
listed in the package's \file{DESCRIPTION} Depends field.

If using an installed package, the checks needing access to all R
objects of the package will load the package (unless it is the
\pkg{base} package), after possibly detaching an already loaded
version of the package.
\end{Details}
%
\begin{Value}
The functions return objects of class the same as the respective
function names containing the information about problems detected.
There are \code{print} methods for nicely displaying the information
contained in such objects.
\end{Value}
\HeaderA{Rd2HTML}{ Rd Converters }{Rd2HTML}
\aliasA{Rd2ex}{Rd2HTML}{Rd2ex}
\aliasA{Rd2latex}{Rd2HTML}{Rd2latex}
\aliasA{Rd2txt}{Rd2HTML}{Rd2txt}
\keyword{documentation}{Rd2HTML}
%
\begin{Description}\relax
These functions take the output of the \code{\LinkA{parse\_Rd}{parse.Rul.Rd}} function
and produce a help page from it.  As they are mainly
intended for internal use, their interfaces are subject to change.
\end{Description}
%
\begin{Usage}
\begin{verbatim}
Rd2HTML(Rd, out = "", package = "", defines = .Platform$OS.type,
        Links = NULL, Links2 = NULL,
        stages = "render", outputEncoding = "UTF-8", 
        dynamic = FALSE, no_links = FALSE, fragment = FALSE,
        stylesheet = "R.css", ...)

Rd2txt(Rd, out = "", package = "", defines = .Platform$OS.type,
       stages = "render", outputEncoding = "",
       fragment = FALSE, options, ...)

Rd2latex(Rd, out = "", defines = .Platform$OS.type,
         stages = "render", outputEncoding = "ASCII",
         fragment = FALSE, ..., writeEncoding = TRUE)

Rd2ex(Rd, out = "", defines = .Platform$OS.type,
      stages = "render", outputEncoding = "UTF-8", ...)
\end{verbatim}
\end{Usage}
%
\begin{Arguments}
\begin{ldescription}
\item[\code{Rd}]  a filename or \code{Rd} object to use as input. 
\item[\code{out}]  a filename or connection object to which to write the output. 
\item[\code{package}]  the package to list in the output. 
\item[\code{defines}]  string(s) to use in \AsIs{\#ifdef} tests. 
\item[\code{stages}]  at which stage (\code{"build"}, \code{"install"}, or
\code{"render"}) should \AsIs{\bsl{}Sexpr} macros be executed? See the
notes below.
\item[\code{outputEncoding}]  see the `Encodings' section below.
\item[\code{dynamic}] logical: set links for render-time resolution by
dynamic help system.
\item[\code{no\_links}] logical: suppress hyperlinks to other help topics.
Used by \command{R CMD \LinkA{Rdconv}{Rdconv}}.
\item[\code{fragment}] logical:  should fragments of Rd files be accepted?  See the
notes below.
\item[\code{stylesheet}] character: a URL for a stylesheet to be used in the header
of the HTML output page.
\item[\code{Links, Links2}] \code{NULL} or a named (by topics) character vector of
links, as returned by \code{\LinkA{findHTMLlinks}{findHTMLlinks}}.
\item[\code{options}] An optional named list of options to pass to
\code{\LinkA{Rd2txt\_options}{Rd2txt.Rul.options}}.
\item[\code{...}]  additional parameters to pass to \code{\LinkA{parse\_Rd}{parse.Rul.Rd}} when
\code{Rd} is a filename. 
\item[\code{writeEncoding}] should \AsIs{\bsl{}inputencoding} lines be written in
the file for non-ASCII encodings?
\end{ldescription}
\end{Arguments}
%
\begin{Details}\relax
These functions convert help documents: \code{Rd2HTML} produces HTML,
\code{Rd2txt} produces plain text, \code{Rd2latex} produces LaTeX.
\code{Rd2ex} extracts the examples in the format used by
\code{\LinkA{example}{example}} and \R{} utilities.

Each of the functions accepts a filename for an Rd file, and
will use \code{\LinkA{parse\_Rd}{parse.Rul.Rd}} to parse it before applying the
conversions or checks.

The difference between arguments \code{Link} and \code{Link2} is that
links are looked in them in turn, so lazy-evaluation can be used to
only do a second-level search for links if required.

Note that the default for \code{Rd2latex} is to output ASCII,
including using the second option of \AsIs{\bsl{}enc} markup.  This was
chosen because use of UTF-8 in LaTeX requires version
`2005/12/01' or later, and even with that version the coverage
of UTF-8 glyphs is not extensive (and not even as complete as
Latin-1).

\code{Rd2txt} will format text paragraphs to a width determined by
\code{width}, with appropriate margins.  The default is to be close to
the rendering in versions of \R{} < 2.10.0.

\code{Rd2txt} will use directional quotes (see \code{\LinkA{sQuote}{sQuote}})
if option \code{"useFancyQuotes"} is true (the default) and
the current encoding is UTF-8.

Various aspects of formatting by \code{Rd2txt} are controlled by the
\code{options} argument, documented with the \code{\LinkA{Rd2txt\_options}{Rd2txt.Rul.options}}
function. Changes made using \code{options} are temporary, those
made with \code{\LinkA{Rd2txt\_options}{Rd2txt.Rul.options}} are persistent.

When \code{fragment = TRUE}, the \code{Rd} file will be rendered
with no processing of \AsIs{\bsl{}Sexpr} elements or conditional defines
using \AsIs{\#ifdef} or \AsIs{\#ifndef}.  Normally a fragment represents
text within a section, but if the first element of the fragment
is a section macro, the whole fragment will be rendered as 
a series of sections, without the usual sorting.
\end{Details}
%
\begin{Value}
These functions are executed mainly for the side effect of
writing the converted help page.  Their value is the name of the output
file (invisibly).  For \code{Rd2latex}, the output name is given an
attribute \code{"latexEncoding"} giving the encoding of the file in a
form suitable for use with the LaTeX \samp{inputenc} package.
\end{Value}
%
\begin{Section}{Encodings}
Rd files are normally intended to be rendered on a wide variety of
systems, so care must be taken in the encoding of non-ASCII
characters.  In general, any such encoding should be declared using
the \samp{encoding} section for there to be any hope of correct
rendering.

For output, the \code{outputEncoding} argument will be used:
\code{outputEncoding = ""} will choose the native encoding for the
current system.

If the text cannot be converted to the \code{outputEncoding}, byte
substitution will be used (see \code{\LinkA{iconv}{iconv}}): \code{Rd2latex}
and \code{Rd2ex} give a warning. 
\end{Section}
%
\begin{Note}\relax
The \AsIs{\bsl{}Sexpr} macro is a new addition to Rd files.  It includes
\R{} code that will be executed at one of three times: \emph{build} time
(when a package's source code is built into a tarball), 
\emph{install} time (when the package is installed or
built into a binary package), and \emph{render} time (when the man
page is converted to a readable format).

For example, this man page was:
\begin{enumerate}

\item built on
2012-09-21 at 13:24:15,
\item installed on
2012-09-21 at 13:24:15, and
\item rendered on
2012-09-21 at 13:46:18.

\end{enumerate}

\end{Note}
%
\begin{Author}\relax
Duncan Murdoch, Brian Ripley
\end{Author}
%
\begin{References}\relax
 \url{http://developer.r-project.org/parseRd.pdf} 
\end{References}
%
\begin{SeeAlso}\relax
\code{\LinkA{parse\_Rd}{parse.Rul.Rd}}, \code{\LinkA{checkRd}{checkRd}},
\code{\LinkA{findHTMLlinks}{findHTMLlinks}}, \code{\LinkA{Rd2txt\_options}{Rd2txt.Rul.options}}.
\end{SeeAlso}
%
\begin{Examples}
\begin{ExampleCode}


## Not run: 
## Simulate install and rendering of this page in HTML and text format:

Rd <- file.path("src/library/tools/man/Rd2HTML.Rd")

outfile <- tempfile(fileext = ".html")
browseURL(Rd2HTML(Rd, outfile, package = "tools",
          stages = c("install", "render")))

outfile <- tempfile(fileext = ".txt")
file.show(Rd2txt(Rd, outfile, package = "tools",
          stages = c("install", "render")))

checkRd(Rd) # A stricter test than Rd2HTML uses

## End(Not run)
\end{ExampleCode}
\end{Examples}
\HeaderA{Rd2txt\_options}{Set formatting options for text help}{Rd2txt.Rul.options}
\keyword{documentation}{Rd2txt\_options}
%
\begin{Description}\relax
This function sets various options for displaying text help.
\end{Description}
%
\begin{Usage}
\begin{verbatim}
Rd2txt_options(...)
\end{verbatim}
\end{Usage}
%
\begin{Arguments}
\begin{ldescription}
\item[\code{...}] A list containing named options, or 
options passed as individual named arguments.  
See below for currently defined ones.
\end{ldescription}
\end{Arguments}
%
\begin{Details}\relax
This function persistently sets various formatting options for the
\code{\LinkA{Rd2txt}{Rd2txt}} function which is used in displaying text format
help.  Currently defined options are:
\begin{description}

\item[width] (default 80):  The width of the output page.
\item[minIndent] (default 10):  The minimum indent to use in a list.
\item[extraIndent] (default 4):  The extra indent to use in each
level of nested lists.
\item[sectionIndent] (default 5):  The indent level for a section.
\item[sectionExtra] (default 2):  The extra indentation for each
nested section level.
\item[itemBullet] (default \code{"* "}, with the asterisk replaced by a
Unicode bullet in UTF-8 and most Windows locales):
The symbol to use as a bullet in itemized lists.
\item[enumFormat] :  A function to format item numbers in enumerated lists.
\item[showURLs] (default \code{FALSE}):  Whether to show URLs when
expanding \AsIs{\bsl{}href} tags.
\item[code\_quote] (default \code{TRUE}):  Whether to render
\AsIs{\bsl{}code} and similar with single quotes.
\item[underline\_titles] default \code{TRUE}):  Whether to render
section titles with underlines (via backspacing).

\end{description}

\end{Details}
%
\begin{Value}
 
If called with no arguments, returns all option settings in a list.
Otherwise, it changes the named settings and invisibly returns their
previous values.
\end{Value}
%
\begin{Author}\relax
Duncan Murdoch
\end{Author}
%
\begin{SeeAlso}\relax
\code{\LinkA{Rd2txt}{Rd2txt}}
\end{SeeAlso}
%
\begin{Examples}
\begin{ExampleCode}
saveOpts <- Rd2txt_options()
saveOpts
Rd2txt_options(minIndent=4)
Rd2txt_options()
Rd2txt_options(saveOpts)
Rd2txt_options()
\end{ExampleCode}
\end{Examples}
\HeaderA{Rdiff}{Difference R Output Files}{Rdiff}
\keyword{utilities}{Rdiff}
%
\begin{Description}\relax
Given two \R{} output files, compute differences ignoring headers,
footers and some other differences.
\end{Description}
%
\begin{Usage}
\begin{verbatim}
Rdiff(from, to, useDiff = FALSE, forEx = FALSE, nullPointers = TRUE,
      Log = FALSE)
\end{verbatim}
\end{Usage}
%
\begin{Arguments}
\begin{ldescription}
\item[\code{from, to}] filepaths to be compared
\item[\code{useDiff}] should \command{diff} always be used to compare
results?
\item[\code{forEx}] logical: extra pruning for \file{-Ex.Rout} files to
exclude the header.
\item[\code{nullPointers}] logical: should the displayed addresses of
pointers be set to \code{0x00000000} before comparison?
\item[\code{Log}] logical: should the returned value include a log of
differences found?
\end{ldescription}
\end{Arguments}
%
\begin{Details}\relax
The \R{} startup banner and any timing information from \command{R CMD
BATCH} are removed from both files, together with lines about loading
packages.  UTF-8 fancy quotes (see \code{\LinkA{sQuote}{sQuote}}) and on
Windows, Windows' so-called `smart quotes' are mapped to a
simple quote.  Addresses of environments, compiled bytecode and other
exotic types expressed as hex addresses (e.g. \code{<environment:
    0x12345678>}) are mapped to \code{0x00000000}.  The files are then
compared line-by-line.  If there are the same number of lines and
\code{useDiff} is false, a simple \command{diff}-like display of
differences is printed, otherwise \command{diff -bw} is called on the
edited files.

As from \R{} 2.14.0 this can compare uncompressed PDF files, ignoring
differences in creation and modification dates.
\end{Details}
%
\begin{Value}
If \code{Log} is true, a list with components \code{status} (see
below) and \code{out}, a character vector of descriptions of
differences, possibly of zero length.

Otherwise, a status indicator, \code{0L} if and only if no differences
were found.
\end{Value}
%
\begin{SeeAlso}\relax
The shell script run as \command{R CMD Rdiff}.
\end{SeeAlso}
\HeaderA{Rdindex}{Generate Index from Rd Files}{Rdindex}
\keyword{documentation}{Rdindex}
\keyword{utilities}{Rdindex}
%
\begin{Description}\relax
Print a 2-column index table with names and titles from given
R documentation files to a given output file or connection.  The
titles are nicely formatted between two column positions (typically 25
and 72, respectively).
\end{Description}
%
\begin{Usage}
\begin{verbatim}
Rdindex(RdFiles, outFile = "", type = NULL,
        width = 0.9 * getOption("width"), indent = NULL)
\end{verbatim}
\end{Usage}
%
\begin{Arguments}
\begin{ldescription}
\item[\code{RdFiles}] a character vector specifying the Rd files to be used
for creating the index, either by giving the paths to the files, or
the path to a single directory with the sources of a package.
\item[\code{outFile}] a connection, or a character string naming the output
file to print to.  \code{""} (the default) indicates output to the
console.
\item[\code{type}] a character string giving the documentation type of the Rd
files to be included in the index, or \code{NULL} (the default).
The type of an Rd file is typically specified via the
\AsIs{\bsl{}docType} tag; if \code{type} is \code{"data"}, Rd files
whose \emph{only} keyword is \code{datasets} are included as well.
\item[\code{width}] a positive integer giving the target column for wrapping
lines in the output.
\item[\code{indent}] a positive integer specifying the indentation of the
second column.  Must not be greater than \code{width/2}, and
defaults to \code{width/3}.
\end{ldescription}
\end{Arguments}
%
\begin{Details}\relax
If a name is not a valid alias, the first alias (or the empty string
if there is none) is used instead.
\end{Details}
\HeaderA{RdTextFilter}{Select text in an Rd file.}{RdTextFilter}
\keyword{utilities}{RdTextFilter}
\keyword{documentation}{RdTextFilter}
%
\begin{Description}\relax
This function blanks out all non-text in an Rd file, 
for spell checking or other uses.
\end{Description}
%
\begin{Usage}
\begin{verbatim}
RdTextFilter(ifile, encoding = "unknown", keepSpacing = TRUE,
             drop = character(), keep = character())
\end{verbatim}
\end{Usage}
%
\begin{Arguments}
\begin{ldescription}
\item[\code{ifile}] 
An input file specified as a filename or connection, or an \code{"Rd"} object
from \code{\LinkA{parse\_Rd}{parse.Rul.Rd}}.

\item[\code{encoding}] 
An encoding name to pass to \code{\LinkA{parse\_Rd}{parse.Rul.Rd}}.

\item[\code{keepSpacing}] 
Whether to try to leave the text in the same lines and columns
as in the original file.

\item[\code{drop}] 
Additional sections of the Rd to drop.

\item[\code{keep}] 
Sections of the Rd file to keep.

\end{ldescription}
\end{Arguments}
%
\begin{Details}\relax
This function parses the Rd file, then walks through it, element by
element.  Items with tag \code{"TEXT"} are kept in the same position as
they appeared in the original file, while other parts of the file are
replaced with blanks, so a spell checker such as \code{\LinkA{aspell}{aspell}}
can check only the text and report the position in the original file.
(If \code{keepSpacing} is \code{FALSE}, blank filling will not occur,
and text will not be output in its original location.)

By default, the tags \AsIs{\bsl{}S3method}, \AsIs{\bsl{}S4method},
\AsIs{\bsl{}command}, \AsIs{\bsl{}docType}, \AsIs{\bsl{}email}, \AsIs{\bsl{}encoding},
\AsIs{\bsl{}file}, \AsIs{\bsl{}keyword}, \AsIs{\bsl{}link}, \AsIs{\bsl{}linkS4class},
\AsIs{\bsl{}method}, \AsIs{\bsl{}pkg}, and \AsIs{\bsl{}var} are skipped.  Additional
tags can be skipped by listing them in the \code{drop} argument;
listing tags in the \code{keep} argument will stop them from being
skipped.  It is also possible to \code{keep} any of the
\code{c("RCODE", "COMMENT", "VERB")} tags, which correspond to R-like
code, comments, and verbatim text respectively, or to drop
\code{"TEXT"}.
\end{Details}
%
\begin{Value}
A character vector which if written to a file, one element per line, would 
duplicate the text elements of the original Rd file.
\end{Value}
%
\begin{Note}\relax
The filter attempts to merge text elements into single words when markup in
the Rd file is used to highlight just the start of a word.
\end{Note}
%
\begin{Author}\relax
Duncan Murdoch
\end{Author}
%
\begin{SeeAlso}\relax
\code{\LinkA{aspell}{aspell}}, for which this is an acceptable \code{filter}.
\end{SeeAlso}
\HeaderA{Rdutils}{Rd Utilities}{Rdutils}
\aliasA{Rd\_db}{Rdutils}{Rd.Rul.db}
\keyword{utilities}{Rdutils}
\keyword{documentation}{Rdutils}
%
\begin{Description}\relax
Utilities for computing on the information in Rd objects.
\end{Description}
%
\begin{Usage}
\begin{verbatim}
Rd_db(package, dir, lib.loc = NULL)
\end{verbatim}
\end{Usage}
%
\begin{Arguments}
\begin{ldescription}
\item[\code{package}] a character string naming an installed package.
\item[\code{dir}] a character string specifying the path to a package's root
source directory.  This should contain the subdirectory \file{man}
with \R{} documentation sources (in Rd format).  Only used if
\code{package} is not given.
\item[\code{lib.loc}] a character vector of directory names of \R{} libraries,
or \code{NULL}.  The default value of \code{NULL} corresponds to all
libraries currently known.  The specified library trees are used to
search for \code{package}.
\end{ldescription}
\end{Arguments}
%
\begin{Details}\relax
\code{Rd\_db} builds a simple database of all Rd objects in a package,
as a list of the results of running \code{\LinkA{parse\_Rd}{parse.Rul.Rd}} on the Rd
source files in the package and processing platform conditionals.
\end{Details}
%
\begin{SeeAlso}\relax
\code{\LinkA{parse\_Rd}{parse.Rul.Rd}}
\end{SeeAlso}
%
\begin{Examples}
\begin{ExampleCode}

## Build the Rd db for the (installed) base package.
db <- Rd_db("base")

## Keyword metadata per Rd object.
keywords <- lapply(db, tools:::.Rd_get_metadata, "keyword")
## Tabulate the keyword entries.
kw_table <- sort(table(unlist(keywords)))
## The 5 most frequent ones:
rev(kw_table)[1 : 5]
## The "most informative" ones:
kw_table[kw_table == 1]

## Concept metadata per Rd file.
concepts <- lapply(db, tools:::.Rd_get_metadata, "concept")
## How many files already have \concept metadata?
sum(sapply(concepts, length) > 0)
## How many concept entries altogether?
length(unlist(concepts))
\end{ExampleCode}
\end{Examples}
\HeaderA{read.00Index}{Read 00Index-style Files}{read.00Index}
\keyword{file}{read.00Index}
\keyword{connection}{read.00Index}
%
\begin{Description}\relax
Read item/description information from \file{00Index}-like files.  Such
files are description lists rendered in tabular form, and currently
used for the \file{INDEX} and \file{demo/00Index} files of add-on
packages.
\end{Description}
%
\begin{Usage}
\begin{verbatim}
read.00Index(file)
\end{verbatim}
\end{Usage}
%
\begin{Arguments}
\begin{ldescription}
\item[\code{file}] the name of a file to read data values from.  If the
specified file is \code{""}, then input is taken from the keyboard
(in this case input can be terminated by a blank line).
Alternatively, \code{file} can be a \code{\LinkA{connection}{connection}}, which
will be opened if necessary, and if so closed at the end of the
function call.
\end{ldescription}
\end{Arguments}
%
\begin{Value}
A character matrix with 2 columns named \code{"Item"} and
\code{"Description"} which hold the items and descriptions.
\end{Value}
%
\begin{SeeAlso}\relax
\code{\LinkA{formatDL}{formatDL}} for the inverse operation of creating a
00Index-style file from items and their descriptions.
\end{SeeAlso}
\HeaderA{readNEWS}{Read R's NEWS file or a similar one}{readNEWS}
\aliasA{checkNEWS}{readNEWS}{checkNEWS}
\keyword{file}{readNEWS}
\keyword{documentation}{readNEWS}
%
\begin{Description}\relax
Read/check \R{}'s \file{NEWS} file or a similarly formatted one.
This was an experimental feature added in \R{} 2.4.0: as from
\R{} 2.12.0 the preferred format is \file{NEWS.Rd}.
\end{Description}
%
\begin{Usage}
\begin{verbatim}
readNEWS(file = file.path(R.home(), "NEWS"), trace = FALSE,
         chop = c("first", "1", "par1", "keepAll"))
checkNEWS(file = file.path(R.home(), "NEWS"))
\end{verbatim}
\end{Usage}
%
\begin{Arguments}
\begin{ldescription}
\item[\code{file}] the name of the file which the data are to be read from.
Alternatively, \code{file} can be a \code{\LinkA{connection}{connection}}, which
will be opened if necessary, and can also be a complete URL.  For
more details, see the \code{file} argument of \code{\LinkA{read.table}{read.table}}.
\item[\code{trace}] logical indicating if the recursive reading should be
traced, i.e., print what it is doing.
\item[\code{chop}] a character string specifying how the news entries should
be \emph{chopped}; \code{chop = "keepAll"} saves the full entries.
\end{ldescription}
\end{Arguments}
%
\begin{Details}\relax
\code{readNEWS()} reads a pre-\R{}-2.12.0-style \file{NEWS} file;
\code{checkNEWS()} checks for common errors in formatting.  Currently it
detects an incorrect number of spaces before the \code{"o"} item marker.

If non-ASCII characters are needed, the \file{NEWS} file may be encoded in
UTF-8 with a byte-order mark (BOM) at the beginning, which
\code{readNEWS()} will recognize.  Other encodings will display
incorrectly on some systems.  However, BOMs are discouraged on many
systems and not all editors recognize them, so \file{NEWS} files should
normally be written in ASCII.
\end{Details}
%
\begin{Value}
\code{readNEWS()} returns an (S3) object of class \code{"newsTree"};
effectively a \code{\LinkA{list}{list}} of lists which is a tree of NEWS
entries.

\code{checkNEWS()} returns \code{TRUE} if no suspected errors are found,
or prints a message for each suspected error and returns \code{FALSE}.

Note that this was only ever experimental and may be removed soon,
since the NEWS-file format is not longer supported (but there are
examples \file{ONEWS}, \file{OONEWS} and perhaps in packages).
\end{Value}
%
\begin{Examples}
\begin{ExampleCode}

NEWStr  <- readNEWS(trace = TRUE)# chop = "first" ( = "first non-empty")
## keep the full NEWS entry text i.e. "no chopping":
NEWStrA <- readNEWS(chop = "keepAll")
object.size(NEWStr)
object.size(NEWStrA) ## (no chopping) ==> about double the size

str(NEWStr, max.level = 3)

str(NEWStr[[c("2.3", "2.3.1")]], max.level=2, vec.len=1)

NEWStr [[c("2.3", "2.3.1", "NEW FEATURES")]]
NEWStrA[[c("2.4", "2.4.0", "NEW FEATURES")]]

# Check the current NEWS file

stopifnot(checkNEWS())

\end{ExampleCode}
\end{Examples}
\HeaderA{showNonASCII}{Pick Out Non-ASCII Characters}{showNonASCII}
\aliasA{showNonASCIIfile}{showNonASCII}{showNonASCIIfile}
\keyword{utilities}{showNonASCII}
%
\begin{Description}\relax
This function prints elements of a character vector which contain
non-ASCII bytes, printing such bytes as a escape like \samp{<fc>}.
\end{Description}
%
\begin{Usage}
\begin{verbatim}
showNonASCII(x)

showNonASCIIfile(file)
\end{verbatim}
\end{Usage}
%
\begin{Arguments}
\begin{ldescription}
\item[\code{x}] a character vector.
\item[\code{file}] path to a file.
\end{ldescription}
\end{Arguments}
%
\begin{Details}\relax
This was originally written to help detect non-portable text in
files in packages.

It prints all element of \code{x} which contain non-ASCII characters,
preceded by the element number and with non-ASCII bytes highlighted
\emph{via} \code{\LinkA{iconv}{iconv}(sub = "byte")}.
\end{Details}
%
\begin{Value}
The elements of \code{x} containing non-ASCII characters will be returned 
invisibly.
\end{Value}
%
\begin{Examples}
\begin{ExampleCode}
out <- c(
"fa\xE7ile test of showNonASCII():",
"\\details{",
"   This is a good line",
"   This has an \xfcmlaut in it.",
"   OK again.",
"}")
f <- tempfile()
cat(out, file = f, sep = "\n")

showNonASCIIfile(f)
unlink(f)
\end{ExampleCode}
\end{Examples}
\HeaderA{startDynamicHelp}{Start the Dynamic HTML Help System}{startDynamicHelp}
\aliasA{help.ports}{startDynamicHelp}{help.ports}
\keyword{documentation}{startDynamicHelp}
%
\begin{Description}\relax
This function starts the internal help server, so that HTML help pages
are rendered when requested.
\end{Description}
%
\begin{Usage}
\begin{verbatim}
startDynamicHelp(start=TRUE)
\end{verbatim}
\end{Usage}
%
\begin{Arguments}
\begin{ldescription}
\item[\code{start}] logical: whether to start or shut down the dynamic help
system.
\end{ldescription}
\end{Arguments}
%
\begin{Details}\relax
This function starts the internal HTTP server, which runs on the
loopback interface (\code{127.0.0.1}).  If \code{options("help.ports")}
is set to a vector of integer values, \code{startDynamicHelp} will try
those ports in order; otherwise, it tries up to 10 random ports to
find one not in use.  It can be disabled by setting the environment
variable \env{R\_DISABLE\_HTTPD} to a non-empty value.

\code{startDynamicHelp} is called by functions that need to use the
server, so would rarely be called directly by a user.

Note that \code{options(help\_type="html")} must be set to actually make use
of HTML help, although it might be the default for an \R{} installation.

If the server cannot be started or is disabled,
\code{\LinkA{help.start}{help.start}} will be unavailable and requests for HTML
help will give text help (with a warning).

The browser in use does need to be able to connect to the loopback
interface: occasionally it is set to use a proxy for HTTP on all
interfaces, which will not work -- the solution is to add an exception
for \code{127.0.0.1}.
\end{Details}
%
\begin{Value}
The chosen port number is returned invisibly (which will be \code{0}
if the server has been stopped).
\end{Value}
%
\begin{SeeAlso}\relax
\code{\LinkA{help.start}{help.start}} and \code{\LinkA{help}{help}(help\_type = "html")}
will attempt to start the HTTP server if required

\code{\LinkA{Rd2HTML}{Rd2HTML}} is used to render the package help pages.
\end{SeeAlso}
\HeaderA{SweaveTeXFilter}{Strip R code out of Sweave file}{SweaveTeXFilter}
\keyword{utilities}{SweaveTeXFilter}
\keyword{documentation}{SweaveTeXFilter}
%
\begin{Description}\relax
This function blanks out code chunks and Noweb markup in an Sweave input file, 
for spell checking or other uses.
\end{Description}
%
\begin{Usage}
\begin{verbatim}
SweaveTeXFilter(ifile, encoding = "unknown")
\end{verbatim}
\end{Usage}
%
\begin{Arguments}
\begin{ldescription}
\item[\code{ifile}] 
Input file or connection.

\item[\code{encoding}] 
Text encoding to pass to \code{\LinkA{readLines}{readLines}}.

\end{ldescription}
\end{Arguments}
%
\begin{Details}\relax
This function blanks out all Noweb markup and code chunks from an Sweave input 
file, leaving behind the LaTeX source, so that a LaTeX-aware spelling checker
can check it and report errors in their original locations.
\end{Details}
%
\begin{Value}
A character vector which if written to a file, one element per line, would 
duplicate the text elements of the original Rd file.
\end{Value}
%
\begin{Author}\relax
Duncan Murdoch
\end{Author}
%
\begin{SeeAlso}\relax
\code{\LinkA{aspell}{aspell}}, for which this is used with \code{filter="Sweave"}.
\end{SeeAlso}
\HeaderA{testInstalledPackage}{Test Installed Packages}{testInstalledPackage}
\aliasA{testInstalledBasic}{testInstalledPackage}{testInstalledBasic}
\aliasA{testInstalledPackages}{testInstalledPackage}{testInstalledPackages}
\keyword{utilities}{testInstalledPackage}
%
\begin{Description}\relax
These functions allow an installed package to be tested, or all base
and recommended packages.
\end{Description}
%
\begin{Usage}
\begin{verbatim}
testInstalledPackage(pkg, lib.loc = NULL, outDir = ".",
                     types = c("examples", "tests", "vignettes"),
                     srcdir = NULL, Ropts = "")

testInstalledPackages(outDir = ".", errorsAreFatal = TRUE,
                      scope = c("both", "base", "recommended"),
                      types = c("examples", "tests", "vignettes"),
                      srcdir = NULL, Ropts = "")

testInstalledBasic(scope = c("basic", "devel", "both"))
\end{verbatim}
\end{Usage}
%
\begin{Arguments}
\begin{ldescription}
\item[\code{pkg}] name of an installed package.
\item[\code{lib.loc}] library path(s) in which to look for the package.  See
\code{\LinkA{library}{library}}.
\item[\code{outDir}] the directory into which to write the output files: this
should already exist.
\item[\code{types}] type(s) of tests to be done.
\item[\code{errorsAreFatal}] logical: should testing terminate at the first
error?
\item[\code{srcdir}] Optional directory to look for \code{.save} files.
\item[\code{Ropts}] Additional options such as \option{-d valgrind} to be
passed to \command{R CMD BATCH} when running examples or tests.
\item[\code{scope}] Which set(s) should be tested?
\end{ldescription}
\end{Arguments}
%
\begin{Details}\relax
These tests depend on having the package example files installed (which
is the default).  If package-specific tests are found in a
\file{tests} directory they can be tested: these are not
installed by default, but will be if
\command{R CMD INSTALL --install-tests} was used.  Finally, the \R{}
code in any vignettes can be extracted and tested.

Package tests are run in a \file{\var{pkg}-tests} subdirectory of
\file{outDir}, and leave their output there.

\code{testInstalledBasic} runs the basic tests, if installed.  This
should be run with \code{LC\_COLLATE=C} set: the function tries to set
this by it may not work on all OSes.  For non-English locales it may
be desirable to set environment variables \env{LANGUAGE} to \samp{en}
and \env{LC\_TIME} to \samp{C} to reduce the number of differences from
reference results.

The package-specific tests for the base and recommended packages are
not normally installed, but \command{make install-tests} is provided
to do so (as well as the basic tests).
\end{Details}
%
\begin{Value}
Invisibly \code{0L} for success, \code{1L} for failure.
\end{Value}
\HeaderA{texi2dvi}{Compile LaTeX Files}{texi2dvi}
\aliasA{texi2pdf}{texi2dvi}{texi2pdf}
\keyword{utilities}{texi2dvi}
%
\begin{Description}\relax
Run \code{latex} and \code{bibtex} until all cross-references are
resolved and create either a dvi or PDF file.
\end{Description}
%
\begin{Usage}
\begin{verbatim}
texi2dvi(file, pdf = FALSE, clean = FALSE, quiet = TRUE,
         texi2dvi = getOption("texi2dvi"),
         texinputs = NULL, index = TRUE)

texi2pdf(file, clean = FALSE, quiet = TRUE,
         texi2dvi = getOption("texi2dvi"),
         texinputs = NULL, index = TRUE)
\end{verbatim}
\end{Usage}
%
\begin{Arguments}
\begin{ldescription}
\item[\code{file}] character.  Name of LaTeX source file.
\item[\code{pdf}] logical.  If \code{TRUE}, a PDF file is produced instead of
the default dvi file (\code{texi2dvi} command line option \option{--pdf}).
\item[\code{clean}] logical.  If \code{TRUE}, all auxiliary files are removed
(\code{texi2dvi} command line option \option{--clean}).  May not work on
some platforms.
\item[\code{quiet}] logical.  No output unless an error occurs.  Ignored if
emulation (see the \code{texi2dvi} argument) is used.
\item[\code{texi2dvi}] character (or \code{NULL}).  Script or program used to
compile a TeX file to dvi or PDF, respectively.
The default (selected by \code{""} or \code{NULL}) is to look for an
executable on the search path and otherwise emulate the script with
\code{system} calls.
\item[\code{texinputs}] \code{NULL} or a character vector of paths to add to
the LaTeX and bibtex input search paths.
\item[\code{index}] logical: should indices be prepared?
\end{ldescription}
\end{Arguments}
%
\begin{Details}\relax
\code{texi2pdf} is a wrapper for the common case of
\code{texi2dvi(pdf = TRUE)}.

Despite the name, this is used in \R{} to compile LaTeX files,
specifically those generated from vignettes.  It
ensures that the \file{\var{\LinkA{R\_HOME}{R.Rul.HOME}}/share/texmf} directory is
in the \env{TEXINPUTS} path, so \R{} style files such as \file{Sweave}
and \file{Rd} will be found.  The search path used is first the
existing \env{TEXINPUTS} setting (or the current directory if unset),
then elements of \code{texinputs}, then
\file{\var{R\_HOME}/share/texmf} and finally the default
path.  Analogous changes are made to \env{BIBINPUTS} and
\env{BSTINPUTS} settings.

MiKTeX has a \code{texi2dvi} executable but no other Windows TeX
installation that we know of does, so emulation is used on e.g. TeXLive
installations on Windows.

Occasionally indices contain special characters which cause indexing
to fail (particularly when using the \samp{hyperref} LaTeX package)
even on valid input.  The argument \code{index = FALSE} is provided to
allow package manuals to be made when this happens: it uses emulation.
\end{Details}
%
\begin{Value}
Invisible \code{NULL}.  Used for the side effect of creating a dvi or
PDF file in the current working directory (and maybe other files,
especially if \code{clean = FALSE}). 
\end{Value}
%
\begin{Note}\relax
There are various versions of the \command{texi2dvi} script on
Unix-alikes and quite a number of bugs have been seen, some of which
this \R{} wrapper works around.

One that is current is that it may not work correctly for paths which
contain spaces, nor even if the absolute path to a file would contain
spaces.

The three possible approaches all have their quirks.  For example the
Unix-alike \code{texi2dvi} script removes ancillary files that already
exist but the other two approaches do not (and may get confused by
such files). 
\end{Note}
%
\begin{Author}\relax
Originally Achim Zeileis but largely rewritten by R-core.
\end{Author}
\HeaderA{toHTML}{Display an object in HTML.}{toHTML}
\methaliasA{toHTML.news\_db}{toHTML}{toHTML.news.Rul.db}
\methaliasA{toHTML.packageIQR}{toHTML}{toHTML.packageIQR}
\keyword{utilities}{toHTML}
\keyword{documentation}{toHTML}
%
\begin{Description}\relax
This generic function generates a complete HTML page from an object.
\end{Description}
%
\begin{Usage}
\begin{verbatim}
toHTML(x, ...)
## S3 method for class 'packageIQR'
toHTML(x, ...)
## S3 method for class 'news_db'
toHTML(x, ...)
\end{verbatim}
\end{Usage}
%
\begin{Arguments}
\begin{ldescription}
\item[\code{x}]  An object to display. 
\item[\code{...}]  Optional parameters for methods; the \code{"packageIQR"} and
\code{"news\_db"} methods pass these to \code{\LinkA{HTMLheader}{HTMLheader}}. 
\end{ldescription}
\end{Arguments}
%
\begin{Value}
A character vector to display the object \code{x}.  The \code{"packageIQR"} method
is designed to display lists in the \R{} help system.
\end{Value}
%
\begin{SeeAlso}\relax
\code{\LinkA{HTMLheader}{HTMLheader}}
\end{SeeAlso}
%
\begin{Examples}
\begin{ExampleCode}
cat(toHTML(demo(package="base")), sep="\n")
\end{ExampleCode}
\end{Examples}
\HeaderA{tools-deprecated}{Deprecated Objects in Package \pkg{tools}}{tools.Rdash.deprecated}
\keyword{misc}{tools-deprecated}
%
\begin{Description}\relax
The functions or variables listed here are provided for compatibility
with older versions of \R{} only, and may be defunct as soon as of the
next release.
\end{Description}
%
\begin{SeeAlso}\relax
\code{\LinkA{Deprecated}{Deprecated}},
\code{\LinkA{Defunct}{Defunct}}
\end{SeeAlso}
\HeaderA{toRd}{Generic function to convert object to a fragment of Rd code.}{toRd}
\methaliasA{toRd.bibentry}{toRd}{toRd.bibentry}
\methaliasA{toRd.default}{toRd}{toRd.default}
\keyword{utilities}{toRd}
\keyword{documentation}{toRd}
%
\begin{Description}\relax
Methods for this function render their associated classes as a fragment of Rd
code, which can then be rendered into text, HTML, or LaTeX.
\end{Description}
%
\begin{Usage}
\begin{verbatim}
toRd(obj, ...)
## S3 method for class 'bibentry'
toRd(obj, style = NULL, ...)
\end{verbatim}
\end{Usage}
%
\begin{Arguments}
\begin{ldescription}
\item[\code{obj}] 
The object to be rendered.

\item[\code{style}] 
The style to be used in converting a \code{\LinkA{bibentry}{bibentry}} object.  

\item[\code{...}] 
Additional arguments used by methods.

\end{ldescription}
\end{Arguments}
%
\begin{Details}\relax
See \code{\LinkA{bibstyle}{bibstyle}} for a discussion of styles.  The default \code{style = NULL} value
gives the default style.
\end{Details}
%
\begin{Value}
Returns a character vector containing a fragment of Rd code that could
be parsed and rendered.  The default method converts \code{obj} to mode \code{character},
then escapes any Rd markup within it.  The \code{bibentry} method converts an
object of that class to markup appropriate for use in a bibliography.
\end{Value}
\HeaderA{undoc}{Find Undocumented Objects}{undoc}
\aliasA{print.undoc}{undoc}{print.undoc}
\keyword{documentation}{undoc}
%
\begin{Description}\relax
Finds the objects in a package which are undocumented, in the sense
that they are visible to the user (or data objects or S4 classes
provided by the package), but no documentation entry exists.
\end{Description}
%
\begin{Usage}
\begin{verbatim}
undoc(package, dir, lib.loc = NULL)
\end{verbatim}
\end{Usage}
%
\begin{Arguments}
\begin{ldescription}
\item[\code{package}] a character string naming an installed package.
\item[\code{dir}] a character string specifying the path to a package's root
source directory.  This must contain the subdirectory \file{man}
with \R{} documentation sources (in Rd format), and at least one of
the \file{R} or \file{data} subdirectories with \R{} code or data
objects, respectively.
\item[\code{lib.loc}] a character vector of directory names of \R{} libraries,
or \code{NULL}.  The default value of \code{NULL} corresponds to all
libraries currently known.  The specified library trees are used to
search for \code{package}.
\end{ldescription}
\end{Arguments}
%
\begin{Details}\relax
This function is useful for package maintainers mostly.  In principle,
\emph{all} user-level \R{} objects should be documented.

The \pkg{base} package is special as it contains the primitives and
these do not have definitions available at code level.  We provide
equivalent closures in environments \code{.ArgsEnv} and
\code{.GenericArgsEnv} in the \pkg{base} package that are used
for various purposes: \code{undoc("base")} checks that all the
primitives that are not language constructs are prototyped in those
environments and no others are.
\end{Details}
%
\begin{Value}
An object of class \code{"undoc"} which is a list of character vectors
containing the names of the undocumented objects split according to
documentation type.

There is a \code{print} method for nicely displaying the information
contained in such objects.
\end{Value}
%
\begin{SeeAlso}\relax
\code{\LinkA{codoc}{codoc}}, \code{\LinkA{QC}{QC}}
\end{SeeAlso}
%
\begin{Examples}
\begin{ExampleCode}
undoc("tools")                  # Undocumented objects in 'tools'
\end{ExampleCode}
\end{Examples}
\HeaderA{vignetteDepends}{Retrieve Dependency Information for a Vignette}{vignetteDepends}
\keyword{utilities}{vignetteDepends}
%
\begin{Description}\relax
Given a vignette name, will create a \code{DependsList} object that
reports information about the packages the vignette depends on.
\end{Description}
%
\begin{Usage}
\begin{verbatim}
vignetteDepends(vignette, recursive = TRUE, reduce = TRUE,
                local = TRUE, lib.loc = NULL)
\end{verbatim}
\end{Usage}
%
\begin{Arguments}
\begin{ldescription}
\item[\code{vignette}] The path to the vignette source
\item[\code{recursive}] Whether or not to include indirect dependencies
\item[\code{reduce}] Whether or not to collapse all sets of dependencies to a
minimal value
\item[\code{local}] Whether or not to search only locally
\item[\code{lib.loc}] What libraries to search in locally
\end{ldescription}
\end{Arguments}
%
\begin{Details}\relax
If \code{recursive} is \code{TRUE}, any package that is specified as a
dependency will in turn have its dependencies included (and so on),
these are known as indirect dependencies.  If recursive is
\code{FALSE}, only the dependencies directly named by the vignette
will be used.

If \code{local} is \code{TRUE}, the system will only look at the
user's local machine and not online to find dependencies.

If \code{reduce} is \code{TRUE}, the system will collapse the fields in the
\code{DependsList} object such that a minimal set of dependencies
are specified (for instance if there was
\samp{foo, foo (>= 1.0.0), foo (>= 1.3.0}, it would only return
\samp{foo (>= 1.3.0)}).
\end{Details}
%
\begin{Value}
An object of class \code{"DependsList"}.
\end{Value}
%
\begin{Author}\relax
 Jeff Gentry 
\end{Author}
%
\begin{SeeAlso}\relax
\code{\LinkA{pkgDepends}{pkgDepends}}
\end{SeeAlso}
%
\begin{Examples}
\begin{ExampleCode}
## This may not be installed
gridEx <- system.file("doc", "grid.Rnw", package = "grid")
vignetteDepends(gridEx)
\end{ExampleCode}
\end{Examples}
\HeaderA{write\_PACKAGES}{Generate PACKAGES files}{write.Rul.PACKAGES}
\keyword{file}{write\_PACKAGES}
\keyword{utilities}{write\_PACKAGES}
%
\begin{Description}\relax
Generate \file{PACKAGES} and \file{PACKAGES.gz} files for a
repository of source or Mac/Windows binary packages.
\end{Description}
%
\begin{Usage}
\begin{verbatim}
write_PACKAGES(dir = ".", fields = NULL,
               type = c("source", "mac.binary", "win.binary"),
               verbose = FALSE, unpacked = FALSE, subdirs = FALSE,
               latestOnly = TRUE, addFiles = FALSE)
\end{verbatim}
\end{Usage}
%
\begin{Arguments}
\begin{ldescription}
\item[\code{dir}] Character vector describing the location of the repository
(directory including source or binary packages) to generate the
\file{PACKAGES} and \file{PACKAGES.gz} files from and write them to.

\item[\code{fields}] 
a character vector giving the fields to be used in the
\file{PACKAGES} and \file{PACKAGES.gz} files in addition to the
default ones, or \code{NULL} (default).  The default corresponds to
the fields needed by \code{\LinkA{available.packages}{available.packages}}:
\code{"Package"}, \code{"Version"}, \code{"Priority"},
\code{"Depends"}, \code{"Imports"}, \code{"LinkingTo"},
\code{"Suggests"}, \code{"Enhances"}, \code{"OS\_type"},
\code{"License"} and \code{"Archs"}, and those fields will always
be included, plus
the file name in field \code{"File"} if \code{addFile = TRUE}
and the path to the subdirectory in field \code{"Path"} if subdirectories
are used.

\item[\code{type}] 
Type of packages: currently source \file{.tar.gz} archives, and Mac
or Windows binary (\file{.tgz} or \file{.zip}, respectively)
packages are supported.  Defaults to \code{"win.binary"} on Windows
and to \code{"source"} otherwise.

\item[\code{verbose}] logical.  Should packages be listed as they are
processed?
\item[\code{unpacked}] a logical indicating whether the package contents are
available in unpacked form or not (default).
\item[\code{subdirs}] either logical (to indicate if subdirectories should be
included, recursively) or a character vector of name of subdirectories
to include.
\item[\code{latestOnly}] logical: if multiple versions of a package are
available should only the latest version be included?
\item[\code{addFiles}] logical: should the filenames be included as field
\samp{File} in the \file{PACKAGES} file.
\end{ldescription}
\end{Arguments}
%
\begin{Details}\relax
\code{write\_PACKAGES} scans the named directory for R packages,
extracts information from each package's \file{DESCRIPTION} file, and
writes this information into the \file{PACKAGES} and
\file{PACKAGES.gz} files.

Including non-latest versions of packages is only useful if they have
less constraining version requirements, so for example
\code{latestOnly = FALSE} could be used for a source repository when
\samp{foo\_1.0} depends on \samp{R >= 2.15.0} but \samp{foo\_0.9} is
available which depends on \samp{R >= 2.11.0}.

Support for repositories with subdirectories and hence for
\code{subdirs != FALSE} was added in \R{} 2.7.0: this depends on
recording a \code{"Path"} field in the \file{PACKAGES} file.

Support for more general file names (e.g. other types of compression)
\emph{via} a \code{"File"} field in the \file{PACKAGES} file was added
in \R{} 2.10.0 and can be used by \code{\LinkA{download.packages}{download.packages}}.  If
the file names are not of the standard form, use
\code{addFiles = TRUE}.

\code{type = "win.binary"} uses \code{\LinkA{unz}{unz}} connections to read
all \file{DESCRIPTION} files contained in the (zipped) binary packages
for Windows in the given directory \code{dir}, and builds files
\file{PACKAGES} and \file{PACKAGES.gz} files from this information.
\end{Details}
%
\begin{Value}
Invisibly returns the number of packages described in the resulting
\file{PACKAGES} and \file{PACKAGES.gz} files.  If \code{0}, no
packages were found and no files were written.
\end{Value}
%
\begin{Note}\relax
Processing \file{.tar.gz} archives to extract the \file{DESCRIPTION}
files is quite slow.

This function can be useful on other OSes to prepare a repository to
be accessed by Windows machines, so \code{type = "win.binary"} should
work on all OSes.
\end{Note}
%
\begin{Author}\relax
Uwe Ligges and R-core.
\end{Author}
%
\begin{SeeAlso}\relax
See \code{\LinkA{read.dcf}{read.dcf}} and \code{\LinkA{write.dcf}{write.dcf}} for reading
\file{DESCRIPTION} files and writing the \file{PACKAGES} and
\file{PACKAGES.gz} files.
\end{SeeAlso}
%
\begin{Examples}
\begin{ExampleCode}
## Not run: 
write_PACKAGES("c:/myFolder/myRepository")  # on Windows
write_PACKAGES("/pub/RWin/bin/windows/contrib/2.9",
               type="win.binary")  # on Linux

## End(Not run)
\end{ExampleCode}
\end{Examples}
\HeaderA{xgettext}{Extract Translatable Messages from R Files in a Package}{xgettext}
\aliasA{xgettext2pot}{xgettext}{xgettext2pot}
\aliasA{xngettext}{xgettext}{xngettext}
\keyword{utilities}{xgettext}
%
\begin{Description}\relax
For each file in the \file{R} directory (including system-specific
subdirectories) of a package, extract the unique arguments passed
to \code{\LinkA{stop}{stop}}, \code{\LinkA{warning}{warning}}, \code{\LinkA{message}{message}},
\code{\LinkA{gettext}{gettext}} and \code{\LinkA{gettextf}{gettextf}}, or to
\code{\LinkA{ngettext}{ngettext}}.
\end{Description}
%
\begin{Usage}
\begin{verbatim}
xgettext(dir, verbose = FALSE, asCall = TRUE)

xngettext(dir, verbose = FALSE)

xgettext2pot(dir, potFile)
\end{verbatim}
\end{Usage}
%
\begin{Arguments}
\begin{ldescription}
\item[\code{dir}] the directory of a source package.
\item[\code{verbose}] logical: should each file be listed as it is processed?
\item[\code{asCall}] logical: if \code{TRUE} each argument is returned whole,
otherwise the strings within each argument are extracted.

\item[\code{potFile}] name of \code{po} template file to be produced.
Defaults to \file{R-\var{pkgname}.pot} where
\var{pkgname} is the basename of \file{dir}.
\end{ldescription}
\end{Arguments}
%
\begin{Details}\relax
Leading and trailing white space (space, tab and linefeed) is removed
for calls to \code{gettext}, \code{gettextf}, \code{stop},
\code{warning}, and \code{message}, as it is by the internal code that
passes strings for translation.

We look to see if these functions were called with \code{domain = NA}
and if so omit the call if \code{asCall = TRUE}: note that the
call might contain a call to \code{gettext} which would be visible if
\code{asCall = FALSE}.

\code{xgettext2pot} calls \code{xgettext} and then \code{xngettext},
and writes a PO template file for use with the \pkg{GNU Gettext}
tools.  This ensures that the strings for simple translation are
unique in the file (as \pkg{GNU Gettext} requires), but does not do so
for \code{ngettext} calls (and the rules are not stated in the Gettext
manual).

If applied to the \pkg{base} package, this also looks in the \file{.R}
files in \file{\var{\LinkA{R\_HOME}{R.Rul.HOME}}/share/R}.
\end{Details}
%
\begin{Value}
For \code{xgettext}, a list of objects of class \code{"xgettext"}
(which has a \code{print} method), one per source file that
potentially contains translatable strings.

For \code{xngettext}, a list of objects of class \code{"xngettext"},
which are themselves lists of length-2 character strings.
\end{Value}
%
\begin{Examples}
\begin{ExampleCode}
## Not run: ## in a source-directory build of R:
xgettext(file.path(R.home(), "src", "library", "splines"))

## End(Not run)
\end{ExampleCode}
\end{Examples}
\clearpage
