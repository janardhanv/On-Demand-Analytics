
\chapter{The \texttt{tcltk} package}
\HeaderA{tcltk-package}{Tcl/Tk Interface}{tcltk.Rdash.package}
\aliasA{tcltk}{tcltk-package}{tcltk}
\keyword{package}{tcltk-package}
%
\begin{Description}\relax
Interface and language bindings to Tcl/Tk GUI elements.
\end{Description}
%
\begin{Details}\relax
This package provides access to the platform-independent Tcl scripting
language and Tk GUI elements.  See \LinkA{TkWidgets}{TkWidgets} for a list of
supported widgets, \LinkA{TkWidgetcmds}{TkWidgetcmds} for commands to work with them,
and references in those files for more.  

The Tcl/Tk documentation is in
the system man pages.

For a complete list of functions, use \code{ls("package:tcltk")}.

Note that Tk will not be initialized if there is no \code{DISPLAY}
variable set, but Tcl can still be used.  This is most useful to allow
the loading of a package which depends on \pkg{tcltk} in a session that
does not actually use it (e.g. during installation).
\end{Details}
%
\begin{Author}\relax
R Core Team

Maintainer: R Core Team \email{R-core@r-project.org}
\end{Author}
\HeaderA{TclInterface}{Low-level Tcl/Tk Interface}{TclInterface}
\aliasA{\$.tclArray}{TclInterface}{.Rdol..tclArray}
\aliasA{\$<\Rdash.tclArray}{TclInterface}{.Rdol.<.Rdash..tclArray}
\aliasA{.Tcl}{TclInterface}{.Tcl}
\methaliasA{.Tcl.args}{TclInterface}{.Tcl.args}
\methaliasA{.Tcl.args.objv}{TclInterface}{.Tcl.args.objv}
\methaliasA{.Tcl.callback}{TclInterface}{.Tcl.callback}
\methaliasA{.Tcl.objv}{TclInterface}{.Tcl.objv}
\aliasA{.Tk.ID}{TclInterface}{.Tk.ID}
\aliasA{.Tk.newwin}{TclInterface}{.Tk.newwin}
\aliasA{.Tk.subwin}{TclInterface}{.Tk.subwin}
\aliasA{.TkRoot}{TclInterface}{.TkRoot}
\aliasA{addTclPath}{TclInterface}{addTclPath}
\aliasA{as.character.tclObj}{TclInterface}{as.character.tclObj}
\aliasA{as.character.tclVar}{TclInterface}{as.character.tclVar}
\aliasA{as.double.tclObj}{TclInterface}{as.double.tclObj}
\aliasA{as.integer.tclObj}{TclInterface}{as.integer.tclObj}
\aliasA{as.logical.tclObj}{TclInterface}{as.logical.tclObj}
\aliasA{as.raw.tclObj}{TclInterface}{as.raw.tclObj}
\aliasA{as.tclObj}{TclInterface}{as.tclObj}
\aliasA{is.tclObj}{TclInterface}{is.tclObj}
\aliasA{is.tkwin}{TclInterface}{is.tkwin}
\aliasA{length.tclArray}{TclInterface}{length.tclArray}
\aliasA{length<\Rdash.tclArray}{TclInterface}{length<.Rdash..tclArray}
\aliasA{names.tclArray}{TclInterface}{names.tclArray}
\aliasA{names<\Rdash.tclArray}{TclInterface}{names<.Rdash..tclArray}
\aliasA{print.tclObj}{TclInterface}{print.tclObj}
\aliasA{tclArray}{TclInterface}{tclArray}
\aliasA{tclObj}{TclInterface}{tclObj}
\methaliasA{tclObj.tclVar}{TclInterface}{tclObj.tclVar}
\aliasA{tclObj<\Rdash}{TclInterface}{tclObj<.Rdash.}
\methaliasA{tclObj<\Rdash.tclVar}{TclInterface}{tclObj<.Rdash..tclVar}
\aliasA{tclRequire}{TclInterface}{tclRequire}
\aliasA{tclvalue}{TclInterface}{tclvalue}
\methaliasA{tclvalue.default}{TclInterface}{tclvalue.default}
\methaliasA{tclvalue.tclObj}{TclInterface}{tclvalue.tclObj}
\methaliasA{tclvalue.tclVar}{TclInterface}{tclvalue.tclVar}
\aliasA{tclvalue<\Rdash}{TclInterface}{tclvalue<.Rdash.}
\methaliasA{tclvalue<\Rdash.default}{TclInterface}{tclvalue<.Rdash..default}
\methaliasA{tclvalue<\Rdash.tclVar}{TclInterface}{tclvalue<.Rdash..tclVar}
\aliasA{tclVar}{TclInterface}{tclVar}
\aliasA{tclvar}{TclInterface}{tclvar}
\aliasA{tkdestroy}{TclInterface}{tkdestroy}
\aliasA{[[.tclArray}{TclInterface}{[[.tclArray}
\aliasA{[[<\Rdash.tclArray}{TclInterface}{[[<.Rdash..tclArray}
\keyword{misc}{TclInterface}
%
\begin{Description}\relax
These functions and variables provide the basic glue between \R{} and the
Tcl interpreter and Tk GUI toolkit. Tk
windows may be represented via \R{} objects. Tcl variables can be accessed
via objects of class \code{tclVar} and the C level interface to Tcl
objects is accessed via objects of class \code{tclObj}.
\end{Description}
%
\begin{Usage}
\begin{verbatim}
.Tcl(...)
.Tcl.objv(objv)
.Tcl.args(...)
.Tcl.args.objv(...)
.Tcl.callback(...)
.Tk.ID(win)
.Tk.newwin(ID)
.Tk.subwin(parent)
.TkRoot

tkdestroy(win)
is.tkwin(x)

tclvalue(x)
tclvalue(x) <- value

tclVar(init="")
## S3 method for class 'tclVar'
as.character(x, ...)
## S3 method for class 'tclVar'
tclvalue(x)
## S3 replacement method for class 'tclVar'
tclvalue(x) <- value

tclArray()
## S3 method for class 'tclArray'
x[[...]]
## S3 replacement method for class 'tclArray'
x[[...]] <- value
## S3 method for class 'tclArray'
x$i
## S3 replacement method for class 'tclArray'
x$i <- value

## S3 method for class 'tclArray'
names(x)
## S3 method for class 'tclArray'
length(x)

tclObj(x)
tclObj(x) <- value
## S3 method for class 'tclVar'
tclObj(x)
## S3 replacement method for class 'tclVar'
tclObj(x) <- value

as.tclObj(x, drop=FALSE)
is.tclObj(x)

## S3 method for class 'tclObj'
as.character(x, ...)
## S3 method for class 'tclObj'
as.integer(x, ...)
## S3 method for class 'tclObj'
as.double(x, ...)
## S3 method for class 'tclObj'
as.logical(x, ...)
## S3 method for class 'tclObj'
as.raw(x, ...)
## S3 method for class 'tclObj'
tclvalue(x)

## Default S3 method:
tclvalue(x)
## Default S3 replacement method:
tclvalue(x) <- value


addTclPath(path = ".")
tclRequire(package, warn = TRUE)
\end{verbatim}
\end{Usage}
%
\begin{Arguments}
\begin{ldescription}
\item[\code{objv}] a named vector of Tcl objects
\item[\code{win}] a window structure
\item[\code{x}] an object
\item[\code{i}] character or (unquoted) name
\item[\code{drop}] logical. Indicates whether a single-element vector should
be made into a simple Tcl object or a list of length one
\item[\code{value}] For \code{tclvalue} assignments, a character string. For
\code{tclObj} assignments, an object of class \code{tclObj}
\item[\code{ID}] a window ID
\item[\code{parent}] a window which becomes the parent of the resulting window
\item[\code{path}] path to a directory containing Tcl packages
\item[\code{package}] a Tcl package name
\item[\code{warn}] logical. Warn if not found?
\item[\code{...}] Additional arguments. See below.
\item[\code{init}] initialization value
\end{ldescription}
\end{Arguments}
%
\begin{Details}\relax
Many of these functions are not intended for general use but are used
internally by the commands that create and manipulate Tk widgets and
Tcl objects.  At the lowest level \code{.Tcl} sends a command as a text
string to the Tcl interpreter and returns the result as an object of
class \code{tclObj} (see below).  A newer variant \code{.Tcl.objv}
accepts arguments in the form of a named list of \code{tclObj}
objects. 

\code{.Tcl.args} converts an R argument list of \code{tag=value} pairs
to the Tcl \code{-option value} style, thus
enabling a simple translation between the two languages. To send a
value with no preceding option flag to Tcl, just use an untagged
argument. In the rare case one needs an option with no subsequent
value \code{tag=NULL} can be used. Most values are just converted to
character mode and inserted in the command string, but window objects
are passed using their ID string, and callbacks are passed via the
result of \code{.Tcl.callback}. Tags are converted to option flags
simply by prepending a \code{-}

\code{.Tcl.args.objv} serves a similar purpose as \code{.Tcl.args} but
produces a list of \code{tclObj} objects suitable for passing to
\code{.Tcl.objv}. The names of the list are converted to Tcl option
style internally by \code{.Tcl.objv}. 

Callbacks can be either \emph{atomic callbacks} handled by
\code{.Tcl.callback} or expressions. An expression is treated as a
list of atomic callbacks, with the following exceptions: if an
element is a name, it is first evaluated in the callers frame, and
likewise if it is an explicit function definition; the \code{break}
expression is translated directly to the Tcl counterpart.
\code{.Tcl.callback} converts \R{} functions and unevaluated calls to
Tcl command strings.  The argument must be either a function closure
or an object of mode \code{"call"} followed by an environment.  The
return value in the first case is of the form \code{R\_call
  0x408b94d4} in which the hexadecimal number is the memory address of
the function. In the second case it will be of the form
\code{R\_call\_lang 0x8a95904 0x819bfd0}. For expressions, a sequence
of similar items is generated, separated by
semicolons. \code{.Tcl.args} takes special precautions to ensure
that functions or calls will continue to exist at the specified
address by assigning the
callback into the relevant window environment (see below).

Tk windows are represented as objects of class \code{tkwin} which are
lists containing  a \code{ID} field and an \code{env} field which is
an \R{} environments, enclosed in the global environment.  The value of
the \code{ID} field is identical to the Tk window name. The \code{env}
environment contains a \code{parent} variable and a \code{num.subwin}
variable.  If the window obtains sub-windows and  callbacks, they are
added as variables to the environment.   \code{.TkRoot} is the top
window with ID "."; this window is not  displayed in order to avoid
ill effects of closing it via window  manager controls. The
\code{parent} variable is undefined for \code{.TkRoot}.

\code{.Tk.ID} extracts the \code{ID} of a window,
\code{.Tk.newwin} creates a new window environment with a given ID and
\code{.Tk.subwin} creates a new window which is a sub-window of a given
parent window.

\code{tkdestroy} destroys a window and also removes the reference to a
window from its parent.

\code{is.tkwin} can be used to test whether a given object is a window
environment.

\code{tclVar} creates a new Tcl variable and initializes it to
\code{init}.  An R object of class \code{tclVar} is created to
represent it.  Using \code{as.character} on the object returns the Tcl
variable name.  Accessing the Tcl variable from R is done using the 
\code{tclvalue} function, which can also occur on the left-hand side of
assignments.  If \code{tclvalue} is passed an argument which is not a
\code{tclVar} object, then it will assume that it is a character string
explicitly naming global Tcl variable. Tcl variables created by 
\code{tclVar} are uniquely named and automatically unset by the
garbage collector when the representing object is no longer in use. 

\code{tclArray} creates a new Tcl array and initializes it to the empty
array.  An R object of class \code{tclArray} and inheriting from class
\code{tclVar} is created to represent it. You can access elements of
the Tcl array using indexing with \code{[[} or \code{\$}, which also
allow replacement forms.  Notice that Tcl arrays are associative by
nature and hence unordered; indexing with a numeric index \code{i}
refers to the element with the \emph{name}
\code{as.character(i)}.  Multiple indices are pasted together separated
by commas to form a single name.  You can query the
length and the set of names in an array using methods for
\code{length} and \code{names}, respectively; these cannot
meaningfully be set so assignment forms exist only to print an error
message. 

It is possible to access Tcl's `dual-ported' objects directly,
thus avoiding parsing and deparsing of their string representation.
This works by using objects of class \code{tclObj}.  The string
representation of such objects can be extracted (but not set) using
\code{tclvalue} and conversion to vectors of mode \code{"character"},
\code{"double"}, \code{"integer"}, \code{"logical"}.  Conversely, such
vectors can be converted using \code{as.tclObj}.  There is an
ambiguity as to what should happen for length one vectors, controlled
by the \code{drop} argument; there are cases where the distinction
matters to Tcl, although mostly it treats them equivalently.  Notice
that \code{tclvalue} and \code{as.character} differ on an object whose
string representation has embedded spaces, the former is sometimes to
be preferred, in particular when applied to the result of
\code{tclread}, \code{tkgetOpenFile}, and similar functions.  The
\code{as.raw} method returns a raw vector or a list of raw vectors and
can be used to return binary data from Tcl.

The object behind a \code{tclVar} object is extracted using
\code{tclObj(x)} which also allows an assignment form, in which the
right hand side of the assignment is automatically converted using
\code{as.tclObj}.  There is a print method for \code{tclObj} objects;
it prints \code{<Tcl>} followed by the string representation of the
object.  Notice that \code{as.character} on a \code{tclVar} object is
the \emph{name} of the corresponding Tcl variable and not the value.

Tcl packages can be loaded with \code{tclRequire}; it may be necessary
to add the directory where they are found to the Tcl search path with
\code{addTclPath}.  The return value is a class \code{"tclObj"} object
if it succeeds, or \code{FALSE} if it fails (when a warning is issued).
\end{Details}
%
\begin{Note}\relax
 Strings containing unbalanced braces are currently not handled
well in many circumstances.
\end{Note}
%
\begin{SeeAlso}\relax
\code{\LinkA{TkWidgets}{TkWidgets}},
\code{\LinkA{TkCommands}{TkCommands}},
\code{\LinkA{TkWidgetcmds}{TkWidgetcmds}}.

\code{\LinkA{capabilities}{capabilities}("tcltk")} to see if Tcl/Tk support was
compiled into this build of \R{}.
\end{SeeAlso}
%
\begin{Examples}
\begin{ExampleCode}
## Not run: 
## These cannot be run by example() but should be OK when pasted
## into an interactive R session with the tcltk package loaded
.Tcl("format \"%s\n\" \"Hello, World!\"")
f <- function()cat("HI!\n")
.Tcl.callback(f)
.Tcl.args(text="Push!", command=f) # NB: Different address

xyzzy <- tclVar(7913)
tclvalue(xyzzy)
tclvalue(xyzzy) <- "foo"
as.character(xyzzy)
tcl("set", as.character(xyzzy))

top <- tktoplevel() # a Tk widget, see Tk-widgets
ls(envir=top$env, all=TRUE)
ls(envir=.TkRoot$env, all=TRUE)# .Tcl.args put a callback ref in here

## End(Not run)
\end{ExampleCode}
\end{Examples}
\HeaderA{tclServiceMode}{ Allow Tcl events to be serviced or not }{tclServiceMode}
\keyword{misc}{tclServiceMode}
%
\begin{Description}\relax
This function controls or reports on the Tcl service mode,
i.e. whether Tcl will respond to events.
\end{Description}
%
\begin{Usage}
\begin{verbatim}
tclServiceMode(on = NULL)
\end{verbatim}
\end{Usage}
%
\begin{Arguments}
\begin{ldescription}
\item[\code{on}] (logical) Whether event servicing is turned on. 
\end{ldescription}
\end{Arguments}
%
\begin{Details}\relax
If called with \code{on == NULL} (the default), no change is made.

Note that this blocks all Tcl/Tk activity, including for widgets from
other packages.  It may be better to manage mapping of windows individually.
\end{Details}
%
\begin{Value}
The value of the Tcl service mode before the call.
\end{Value}
%
\begin{Examples}
\begin{ExampleCode}
## see demo(tkcanvas) for an example
## Not run:     
oldmode <- tclServiceMode(FALSE)
# Do some work to create a nice picture.
# Nothing will be displayed until...
tclServiceMode(oldmode)

## End(Not run)
## another idea is to use tkwm.withdraw() ... tkwm.deiconify()
\end{ExampleCode}
\end{Examples}
\HeaderA{TkCommands}{Tk non-widget commands}{TkCommands}
\aliasA{tcl}{TkCommands}{tcl}
\aliasA{tclclose}{TkCommands}{tclclose}
\aliasA{tclfile.dir}{TkCommands}{tclfile.dir}
\aliasA{tclfile.tail}{TkCommands}{tclfile.tail}
\aliasA{tclopen}{TkCommands}{tclopen}
\aliasA{tclputs}{TkCommands}{tclputs}
\aliasA{tclread}{TkCommands}{tclread}
\aliasA{tkbell}{TkCommands}{tkbell}
\aliasA{tkbind}{TkCommands}{tkbind}
\aliasA{tkbindtags}{TkCommands}{tkbindtags}
\aliasA{tkchooseDirectory}{TkCommands}{tkchooseDirectory}
\aliasA{tkclipboard.append}{TkCommands}{tkclipboard.append}
\aliasA{tkclipboard.clear}{TkCommands}{tkclipboard.clear}
\aliasA{tkdialog}{TkCommands}{tkdialog}
\aliasA{tkevent.add}{TkCommands}{tkevent.add}
\aliasA{tkevent.delete}{TkCommands}{tkevent.delete}
\aliasA{tkevent.generate}{TkCommands}{tkevent.generate}
\aliasA{tkevent.info}{TkCommands}{tkevent.info}
\aliasA{tkfocus}{TkCommands}{tkfocus}
\aliasA{tkfont.actual}{TkCommands}{tkfont.actual}
\aliasA{tkfont.configure}{TkCommands}{tkfont.configure}
\aliasA{tkfont.create}{TkCommands}{tkfont.create}
\aliasA{tkfont.delete}{TkCommands}{tkfont.delete}
\aliasA{tkfont.families}{TkCommands}{tkfont.families}
\aliasA{tkfont.measure}{TkCommands}{tkfont.measure}
\aliasA{tkfont.metrics}{TkCommands}{tkfont.metrics}
\aliasA{tkfont.names}{TkCommands}{tkfont.names}
\aliasA{tkgetOpenFile}{TkCommands}{tkgetOpenFile}
\aliasA{tkgetSaveFile}{TkCommands}{tkgetSaveFile}
\aliasA{tkgrab}{TkCommands}{tkgrab}
\methaliasA{tkgrab.current}{TkCommands}{tkgrab.current}
\methaliasA{tkgrab.release}{TkCommands}{tkgrab.release}
\methaliasA{tkgrab.set}{TkCommands}{tkgrab.set}
\methaliasA{tkgrab.status}{TkCommands}{tkgrab.status}
\aliasA{tkgrid}{TkCommands}{tkgrid}
\methaliasA{tkgrid.bbox}{TkCommands}{tkgrid.bbox}
\methaliasA{tkgrid.columnconfigure}{TkCommands}{tkgrid.columnconfigure}
\methaliasA{tkgrid.configure}{TkCommands}{tkgrid.configure}
\methaliasA{tkgrid.forget}{TkCommands}{tkgrid.forget}
\methaliasA{tkgrid.info}{TkCommands}{tkgrid.info}
\methaliasA{tkgrid.location}{TkCommands}{tkgrid.location}
\methaliasA{tkgrid.propagate}{TkCommands}{tkgrid.propagate}
\methaliasA{tkgrid.remove}{TkCommands}{tkgrid.remove}
\methaliasA{tkgrid.rowconfigure}{TkCommands}{tkgrid.rowconfigure}
\methaliasA{tkgrid.size}{TkCommands}{tkgrid.size}
\methaliasA{tkgrid.slaves}{TkCommands}{tkgrid.slaves}
\aliasA{tkimage.cget}{TkCommands}{tkimage.cget}
\aliasA{tkimage.configure}{TkCommands}{tkimage.configure}
\aliasA{tkimage.create}{TkCommands}{tkimage.create}
\aliasA{tkimage.names}{TkCommands}{tkimage.names}
\aliasA{tklower}{TkCommands}{tklower}
\aliasA{tkmessageBox}{TkCommands}{tkmessageBox}
\aliasA{tkpack}{TkCommands}{tkpack}
\methaliasA{tkpack.configure}{TkCommands}{tkpack.configure}
\methaliasA{tkpack.forget}{TkCommands}{tkpack.forget}
\methaliasA{tkpack.info}{TkCommands}{tkpack.info}
\methaliasA{tkpack.propagate}{TkCommands}{tkpack.propagate}
\methaliasA{tkpack.slaves}{TkCommands}{tkpack.slaves}
\aliasA{tkplace}{TkCommands}{tkplace}
\methaliasA{tkplace.configure}{TkCommands}{tkplace.configure}
\methaliasA{tkplace.forget}{TkCommands}{tkplace.forget}
\methaliasA{tkplace.info}{TkCommands}{tkplace.info}
\methaliasA{tkplace.slaves}{TkCommands}{tkplace.slaves}
\aliasA{tkpopup}{TkCommands}{tkpopup}
\aliasA{tkraise}{TkCommands}{tkraise}
\aliasA{tktitle}{TkCommands}{tktitle}
\aliasA{tktitle<\Rdash}{TkCommands}{tktitle<.Rdash.}
\aliasA{tkwait.variable}{TkCommands}{tkwait.variable}
\aliasA{tkwait.visibility}{TkCommands}{tkwait.visibility}
\aliasA{tkwait.window}{TkCommands}{tkwait.window}
\aliasA{tkwinfo}{TkCommands}{tkwinfo}
\aliasA{tkwm.aspect}{TkCommands}{tkwm.aspect}
\aliasA{tkwm.client}{TkCommands}{tkwm.client}
\aliasA{tkwm.colormapwindows}{TkCommands}{tkwm.colormapwindows}
\aliasA{tkwm.command}{TkCommands}{tkwm.command}
\aliasA{tkwm.deiconify}{TkCommands}{tkwm.deiconify}
\aliasA{tkwm.focusmodel}{TkCommands}{tkwm.focusmodel}
\aliasA{tkwm.frame}{TkCommands}{tkwm.frame}
\aliasA{tkwm.geometry}{TkCommands}{tkwm.geometry}
\aliasA{tkwm.grid}{TkCommands}{tkwm.grid}
\aliasA{tkwm.group}{TkCommands}{tkwm.group}
\aliasA{tkwm.iconbitmap}{TkCommands}{tkwm.iconbitmap}
\aliasA{tkwm.iconify}{TkCommands}{tkwm.iconify}
\aliasA{tkwm.iconmask}{TkCommands}{tkwm.iconmask}
\aliasA{tkwm.iconname}{TkCommands}{tkwm.iconname}
\aliasA{tkwm.iconposition}{TkCommands}{tkwm.iconposition}
\aliasA{tkwm.iconwindow}{TkCommands}{tkwm.iconwindow}
\aliasA{tkwm.maxsize}{TkCommands}{tkwm.maxsize}
\aliasA{tkwm.minsize}{TkCommands}{tkwm.minsize}
\aliasA{tkwm.overrideredirect}{TkCommands}{tkwm.overrideredirect}
\aliasA{tkwm.positionfrom}{TkCommands}{tkwm.positionfrom}
\aliasA{tkwm.protocol}{TkCommands}{tkwm.protocol}
\aliasA{tkwm.resizable}{TkCommands}{tkwm.resizable}
\aliasA{tkwm.sizefrom}{TkCommands}{tkwm.sizefrom}
\aliasA{tkwm.state}{TkCommands}{tkwm.state}
\aliasA{tkwm.title}{TkCommands}{tkwm.title}
\aliasA{tkwm.transient}{TkCommands}{tkwm.transient}
\aliasA{tkwm.withdraw}{TkCommands}{tkwm.withdraw}
\aliasA{tkXselection.clear}{TkCommands}{tkXselection.clear}
\aliasA{tkXselection.get}{TkCommands}{tkXselection.get}
\aliasA{tkXselection.handle}{TkCommands}{tkXselection.handle}
\aliasA{tkXselection.own}{TkCommands}{tkXselection.own}
\keyword{misc}{TkCommands}
%
\begin{Description}\relax
These functions interface to Tk non-widget commands, such as the
window manager interface commands and the geometry managers.
\end{Description}
%
\begin{Usage}
\begin{verbatim}
tcl(...)
tktitle(x)

tktitle(x) <- value

tkbell(...)
tkbind(...)
tkbindtags(...)
tkfocus(...)
tklower(...)
tkraise(...)

tkclipboard.append(...)
tkclipboard.clear(...)

tkevent.add(...)
tkevent.delete(...)
tkevent.generate(...)
tkevent.info(...)

tkfont.actual(...)
tkfont.configure(...)
tkfont.create(...)
tkfont.delete(...)
tkfont.families(...)
tkfont.measure(...)
tkfont.metrics(...)
tkfont.names(...)

tkgrab(...)
tkgrab.current(...)
tkgrab.release(...)
tkgrab.set(...)
tkgrab.status(...)

tkimage.cget(...)
tkimage.configure(...)
tkimage.create(...)
tkimage.names(...)

## NB: some widgets also have a selection.clear command,
## hence the "X".

tkXselection.clear(...)
tkXselection.get(...)
tkXselection.handle(...)
tkXselection.own(...)

tkwait.variable(...)
tkwait.visibility(...)
tkwait.window(...)

## winfo actually has a large number of subcommands,
## but it's rarely used,
## so use tkwinfo("atom", ...) etc. instead.

tkwinfo(...)

# Window manager interface

tkwm.aspect(...)
tkwm.client(...)
tkwm.colormapwindows(...)
tkwm.command(...)
tkwm.deiconify(...)
tkwm.focusmodel(...)
tkwm.frame(...)
tkwm.geometry(...)
tkwm.grid(...)
tkwm.group(...)
tkwm.iconbitmap(...)
tkwm.iconify(...)
tkwm.iconmask(...)
tkwm.iconname(...)
tkwm.iconposition(...)
tkwm.iconwindow(...)
tkwm.maxsize(...)
tkwm.minsize(...)
tkwm.overrideredirect(...)
tkwm.positionfrom(...)
tkwm.protocol(...)
tkwm.resizable(...)
tkwm.sizefrom(...)
tkwm.state(...)
tkwm.title(...)
tkwm.transient(...)
tkwm.withdraw(...)


### Geometry managers

tkgrid(...)
tkgrid.bbox(...)
tkgrid.columnconfigure(...)
tkgrid.configure(...)
tkgrid.forget(...)
tkgrid.info(...)
tkgrid.location(...)
tkgrid.propagate(...)
tkgrid.rowconfigure(...)
tkgrid.remove(...)
tkgrid.size(...)
tkgrid.slaves(...)

tkpack(...)
tkpack.configure(...)
tkpack.forget(...)
tkpack.info(...)
tkpack.propagate(...)
tkpack.slaves(...)

tkplace(...)
tkplace.configure(...)
tkplace.forget(...)
tkplace.info(...)
tkplace.slaves(...)

## Standard dialogs
tkgetOpenFile(...)
tkgetSaveFile(...)
tkchooseDirectory(...)
tkmessageBox(...)
tkdialog(...)
tkpopup(...)


## File handling functions
tclfile.tail(...)
tclfile.dir(...)
tclopen(...)
tclclose(...)
tclputs(...)
tclread(...)
\end{verbatim}
\end{Usage}
%
\begin{Arguments}
\begin{ldescription}
\item[\code{x}] A window object
\item[\code{value}] For \code{tktitle} assignments, a character string.
\item[\code{...}] Handled via \code{.Tcl.args}
\end{ldescription}
\end{Arguments}
%
\begin{Details}\relax

\code{tcl} provides a generic interface to calling any Tk or Tcl
command by simply running \code{.Tcl.args.objv} on the argument list
and passing the
result to \code{.Tcl.objv}.  Most of the other commands simply call
\code{tcl} with a particular
first argument and sometimes also a second argument giving the
subcommand.

\code{tktitle} and its assignment form provides an alternate interface
to Tk's \code{wm title}

There are far too many of these commands to describe them and their
arguments in full.  Please refer to the Tcl/Tk documentation for details.
With a few exceptions, the pattern is that Tk subcommands like
\code{pack configure} are converted to function names like
\code{tkpack.configure}, and Tcl subcommands are like
\code{tclfile.dir}.  
\end{Details}
%
\begin{SeeAlso}\relax
\code{\LinkA{TclInterface}{TclInterface}}, \code{\LinkA{TkWidgets}{TkWidgets}},
\code{\LinkA{TkWidgetcmds}{TkWidgetcmds}}
\end{SeeAlso}
%
\begin{Examples}
\begin{ExampleCode}
## Not run: 
## These cannot be run by examples() but should be OK when pasted
## into an interactive R session with the tcltk package loaded

tt <- tktoplevel()
tkpack(l1<-tklabel(tt,text="Heave"), l2<-tklabel(tt,text="Ho"))
tkpack.configure(l1, side="left")

## Try stretching the window and then

tkdestroy(tt)

## End(Not run)

\end{ExampleCode}
\end{Examples}
\HeaderA{tkpager}{Page file using Tk text widget}{tkpager}
\keyword{misc}{tkpager}
%
\begin{Description}\relax
This plugs into \code{file.show}, showing files in separate windows. 
\end{Description}
%
\begin{Usage}
\begin{verbatim}
tkpager(file, header, title, delete.file)
\end{verbatim}
\end{Usage}
%
\begin{Arguments}
\begin{ldescription}
\item[\code{file}] character vector containing the names of the
files to be displayed
\item[\code{header}] headers to use for each file
\item[\code{title}] common title to use for the window(s). Pasted together
with the \code{header} to form actual window title.
\item[\code{delete.file}] logical. Should file(s) be deleted after display?
\end{ldescription}
\end{Arguments}
%
\begin{Note}\relax
  The \code{"\bsl{}b\_"} string used for underlining is currently
quietly removed. The font and background colour are currently
hardcoded to Courier and gray90.
\end{Note}
%
\begin{SeeAlso}\relax
\code{\LinkA{file.show}{file.show}}
\end{SeeAlso}
\HeaderA{tkProgressBar}{Progress Bars via Tk}{tkProgressBar}
\aliasA{close.tkProgressBar}{tkProgressBar}{close.tkProgressBar}
\aliasA{getTkProgressBar}{tkProgressBar}{getTkProgressBar}
\aliasA{setTkProgressBar}{tkProgressBar}{setTkProgressBar}
\keyword{utilities}{tkProgressBar}
%
\begin{Description}\relax
Put up a Tk progress bar widget.
\end{Description}
%
\begin{Usage}
\begin{verbatim}
tkProgressBar(title = "R progress bar", label = "",
              min = 0, max = 1, initial = 0, width = 300)

getTkProgressBar(pb)
setTkProgressBar(pb, value, title = NULL, label = NULL)
## S3 method for class 'tkProgressBar'
close(con, ...)
\end{verbatim}
\end{Usage}
%
\begin{Arguments}
\begin{ldescription}
\item[\code{title, label}] character strings, giving the window title and the
label on the dialog box respectively.
\item[\code{min, max}] (finite) numeric values for the extremes of the
progress bar.
\item[\code{initial, value}] initial or new value for the progress bar.
\item[\code{width}] the width of the progress bar in pixels: the dialog box
will be 40 pixels wider (plus frame).
\item[\code{pb, con}] an object of class \code{"tkProgressBar"}.
\item[\code{...}] for consistency with the generic.
\end{ldescription}
\end{Arguments}
%
\begin{Details}\relax
\code{tkProgressBar} will display a widget containing a label and
progress bar.

\code{setTkProgessBar} will update the value and for non-\code{NULL}
values, the title and label (provided there was one when the widget
was created).  Missing (\code{\LinkA{NA}{NA}}) and out-of-range values of
\code{value} will be (silently) ignored.

The progress bar should be \code{close}d when finished with.

This will use the \code{ttk::progressbar} widget for Tk version 8.5 or
later, otherwise \R{}'s copy of BWidget's \code{progressbar}.
\end{Details}
%
\begin{Value}
For \code{tkProgressBar} an object of class \code{"tkProgressBar"}.

For \code{getTkProgressBar} and \code{setTkProgressBar}, a
length-one numeric vector giving the previous value (invisibly for
\code{setTkProgressBar}).
\end{Value}
%
\begin{SeeAlso}\relax
\code{\LinkA{txtProgressBar}{txtProgressBar}}  
\end{SeeAlso}
%
\begin{Examples}
\begin{ExampleCode}

pb <- tkProgressBar("test progress bar", "Some information in %",
                    0, 100, 50)
Sys.sleep(0.5)
u <- c(0, sort(runif(20, 0 ,100)), 100)
for(i in u) {
    Sys.sleep(0.1)
    info <- sprintf("%d%% done", round(i))
    setTkProgressBar(pb, i, sprintf("test (%s)", info), info)
}
Sys.sleep(5)
close(pb)
\end{ExampleCode}
\end{Examples}
\HeaderA{tkStartGUI}{Tcl/Tk GUI startup}{tkStartGUI}
\keyword{misc}{tkStartGUI}
%
\begin{Description}\relax
Starts up the Tcl/Tk GUI
\end{Description}
%
\begin{Usage}
\begin{verbatim}
tkStartGUI()
\end{verbatim}
\end{Usage}
%
\begin{Details}\relax
Starts a GUI console implemented via a Tk text widget. This should
probably be called at most once per session. Also redefines the file
pager (as used by \code{help()}) to be the Tk pager. 
\end{Details}
%
\begin{Note}\relax
\code{tkStartGUI()} saves its evaluation environment as
\code{.GUIenv}. This means that the user interface elements can be
accessed in order to extend the interface. The three main objects are
named \code{Term}, \code{Menu}, and \code{Toolbar}, and the various
submenus and callback functions can be seen with
\code{ls(envir=.GUIenv)}.
\end{Note}
%
\begin{Author}\relax
Peter Dalgaard
\end{Author}
\HeaderA{TkWidgetcmds}{Tk widget commands}{TkWidgetcmds}
\aliasA{tkactivate}{TkWidgetcmds}{tkactivate}
\aliasA{tkadd}{TkWidgetcmds}{tkadd}
\aliasA{tkaddtag}{TkWidgetcmds}{tkaddtag}
\aliasA{tkbbox}{TkWidgetcmds}{tkbbox}
\aliasA{tkcanvasx}{TkWidgetcmds}{tkcanvasx}
\aliasA{tkcanvasy}{TkWidgetcmds}{tkcanvasy}
\aliasA{tkcget}{TkWidgetcmds}{tkcget}
\aliasA{tkcompare}{TkWidgetcmds}{tkcompare}
\aliasA{tkconfigure}{TkWidgetcmds}{tkconfigure}
\aliasA{tkcoords}{TkWidgetcmds}{tkcoords}
\aliasA{tkcreate}{TkWidgetcmds}{tkcreate}
\aliasA{tkcurselection}{TkWidgetcmds}{tkcurselection}
\aliasA{tkdchars}{TkWidgetcmds}{tkdchars}
\aliasA{tkdebug}{TkWidgetcmds}{tkdebug}
\aliasA{tkdelete}{TkWidgetcmds}{tkdelete}
\aliasA{tkdelta}{TkWidgetcmds}{tkdelta}
\aliasA{tkdeselect}{TkWidgetcmds}{tkdeselect}
\aliasA{tkdlineinfo}{TkWidgetcmds}{tkdlineinfo}
\aliasA{tkdtag}{TkWidgetcmds}{tkdtag}
\aliasA{tkdump}{TkWidgetcmds}{tkdump}
\aliasA{tkentrycget}{TkWidgetcmds}{tkentrycget}
\aliasA{tkentryconfigure}{TkWidgetcmds}{tkentryconfigure}
\aliasA{tkfind}{TkWidgetcmds}{tkfind}
\aliasA{tkflash}{TkWidgetcmds}{tkflash}
\aliasA{tkfraction}{TkWidgetcmds}{tkfraction}
\aliasA{tkget}{TkWidgetcmds}{tkget}
\aliasA{tkgettags}{TkWidgetcmds}{tkgettags}
\aliasA{tkicursor}{TkWidgetcmds}{tkicursor}
\aliasA{tkidentify}{TkWidgetcmds}{tkidentify}
\aliasA{tkindex}{TkWidgetcmds}{tkindex}
\aliasA{tkinsert}{TkWidgetcmds}{tkinsert}
\aliasA{tkinvoke}{TkWidgetcmds}{tkinvoke}
\aliasA{tkitembind}{TkWidgetcmds}{tkitembind}
\aliasA{tkitemcget}{TkWidgetcmds}{tkitemcget}
\aliasA{tkitemconfigure}{TkWidgetcmds}{tkitemconfigure}
\aliasA{tkitemfocus}{TkWidgetcmds}{tkitemfocus}
\aliasA{tkitemlower}{TkWidgetcmds}{tkitemlower}
\aliasA{tkitemraise}{TkWidgetcmds}{tkitemraise}
\aliasA{tkitemscale}{TkWidgetcmds}{tkitemscale}
\aliasA{tkmark.gravity}{TkWidgetcmds}{tkmark.gravity}
\aliasA{tkmark.names}{TkWidgetcmds}{tkmark.names}
\aliasA{tkmark.next}{TkWidgetcmds}{tkmark.next}
\aliasA{tkmark.previous}{TkWidgetcmds}{tkmark.previous}
\aliasA{tkmark.set}{TkWidgetcmds}{tkmark.set}
\aliasA{tkmark.unset}{TkWidgetcmds}{tkmark.unset}
\aliasA{tkmove}{TkWidgetcmds}{tkmove}
\aliasA{tknearest}{TkWidgetcmds}{tknearest}
\aliasA{tkpost}{TkWidgetcmds}{tkpost}
\aliasA{tkpostcascade}{TkWidgetcmds}{tkpostcascade}
\aliasA{tkpostscript}{TkWidgetcmds}{tkpostscript}
\aliasA{tkscan.dragto}{TkWidgetcmds}{tkscan.dragto}
\aliasA{tkscan.mark}{TkWidgetcmds}{tkscan.mark}
\aliasA{tksearch}{TkWidgetcmds}{tksearch}
\aliasA{tksee}{TkWidgetcmds}{tksee}
\aliasA{tkselect}{TkWidgetcmds}{tkselect}
\aliasA{tkselection.adjust}{TkWidgetcmds}{tkselection.adjust}
\aliasA{tkselection.anchor}{TkWidgetcmds}{tkselection.anchor}
\aliasA{tkselection.clear}{TkWidgetcmds}{tkselection.clear}
\aliasA{tkselection.from}{TkWidgetcmds}{tkselection.from}
\aliasA{tkselection.includes}{TkWidgetcmds}{tkselection.includes}
\aliasA{tkselection.present}{TkWidgetcmds}{tkselection.present}
\aliasA{tkselection.range}{TkWidgetcmds}{tkselection.range}
\aliasA{tkselection.set}{TkWidgetcmds}{tkselection.set}
\aliasA{tkselection.to}{TkWidgetcmds}{tkselection.to}
\aliasA{tkset}{TkWidgetcmds}{tkset}
\aliasA{tksize}{TkWidgetcmds}{tksize}
\aliasA{tktag.add}{TkWidgetcmds}{tktag.add}
\aliasA{tktag.bind}{TkWidgetcmds}{tktag.bind}
\aliasA{tktag.cget}{TkWidgetcmds}{tktag.cget}
\aliasA{tktag.configure}{TkWidgetcmds}{tktag.configure}
\aliasA{tktag.delete}{TkWidgetcmds}{tktag.delete}
\aliasA{tktag.lower}{TkWidgetcmds}{tktag.lower}
\aliasA{tktag.names}{TkWidgetcmds}{tktag.names}
\aliasA{tktag.nextrange}{TkWidgetcmds}{tktag.nextrange}
\aliasA{tktag.prevrange}{TkWidgetcmds}{tktag.prevrange}
\aliasA{tktag.raise}{TkWidgetcmds}{tktag.raise}
\aliasA{tktag.ranges}{TkWidgetcmds}{tktag.ranges}
\aliasA{tktag.remove}{TkWidgetcmds}{tktag.remove}
\aliasA{tktoggle}{TkWidgetcmds}{tktoggle}
\aliasA{tktype}{TkWidgetcmds}{tktype}
\aliasA{tkunpost}{TkWidgetcmds}{tkunpost}
\aliasA{tkwindow.cget}{TkWidgetcmds}{tkwindow.cget}
\aliasA{tkwindow.configure}{TkWidgetcmds}{tkwindow.configure}
\aliasA{tkwindow.create}{TkWidgetcmds}{tkwindow.create}
\aliasA{tkwindow.names}{TkWidgetcmds}{tkwindow.names}
\aliasA{tkxview}{TkWidgetcmds}{tkxview}
\methaliasA{tkxview.moveto}{TkWidgetcmds}{tkxview.moveto}
\methaliasA{tkxview.scroll}{TkWidgetcmds}{tkxview.scroll}
\aliasA{tkyposition}{TkWidgetcmds}{tkyposition}
\aliasA{tkyview}{TkWidgetcmds}{tkyview}
\methaliasA{tkyview.moveto}{TkWidgetcmds}{tkyview.moveto}
\methaliasA{tkyview.scroll}{TkWidgetcmds}{tkyview.scroll}
\keyword{misc}{TkWidgetcmds}
%
\begin{Description}\relax
These functions interface to Tk widget commands.
\end{Description}
%
\begin{Usage}
\begin{verbatim}
tkactivate(widget, ...)
tkadd(widget, ...)
tkaddtag(widget, ...)
tkbbox(widget, ...)
tkcanvasx(widget, ...)
tkcanvasy(widget, ...)
tkcget(widget, ...)
tkcompare(widget, ...)
tkconfigure(widget, ...)
tkcoords(widget, ...)
tkcreate(widget, ...)
tkcurselection(widget,...)
tkdchars(widget, ...)
tkdebug(widget, ...)
tkdelete(widget, ...)
tkdelta(widget, ...)
tkdeselect(widget, ...)
tkdlineinfo(widget, ...)
tkdtag(widget, ...)
tkdump(widget, ...)
tkentrycget(widget, ...)
tkentryconfigure(widget, ...)
tkfind(widget, ...)
tkflash(widget, ...)
tkfraction(widget, ...)
tkget(widget, ...)
tkgettags(widget, ...)
tkicursor(widget, ...)
tkidentify(widget, ...)
tkindex(widget, ...)
tkinsert(widget, ...)
tkinvoke(widget, ...)
tkitembind(widget, ...)
tkitemcget(widget, ...)
tkitemconfigure(widget, ...)
tkitemfocus(widget, ...)
tkitemlower(widget, ...)
tkitemraise(widget, ...)
tkitemscale(widget, ...)
tkmark.gravity(widget, ...)
tkmark.names(widget, ...)
tkmark.next(widget, ...)
tkmark.previous(widget, ...)
tkmark.set(widget, ...)
tkmark.unset(widget, ...)
tkmove(widget, ...)
tknearest(widget, ...)
tkpost(widget, ...)
tkpostcascade(widget, ...)
tkpostscript(widget, ...)
tkscan.mark(widget, ...)
tkscan.dragto(widget, ...)
tksearch(widget, ...)
tksee(widget, ...)
tkselect(widget, ...)
tkselection.adjust(widget, ...)
tkselection.anchor(widget, ...)
tkselection.clear(widget, ...)
tkselection.from(widget, ...)
tkselection.includes(widget, ...)
tkselection.present(widget, ...)
tkselection.range(widget, ...)
tkselection.set(widget, ...)
tkselection.to(widget,...)
tkset(widget, ...)
tksize(widget, ...)
tktoggle(widget, ...)
tktag.add(widget, ...)
tktag.bind(widget, ...)
tktag.cget(widget, ...)
tktag.configure(widget, ...)
tktag.delete(widget, ...)
tktag.lower(widget, ...)
tktag.names(widget, ...)
tktag.nextrange(widget, ...)
tktag.prevrange(widget, ...)
tktag.raise(widget, ...)
tktag.ranges(widget, ...)
tktag.remove(widget, ...)
tktype(widget, ...)
tkunpost(widget, ...)
tkwindow.cget(widget, ...)
tkwindow.configure(widget, ...)
tkwindow.create(widget, ...)
tkwindow.names(widget, ...)
tkxview(widget, ...)
tkxview.moveto(widget, ...)
tkxview.scroll(widget, ...)
tkyposition(widget, ...)
tkyview(widget, ...)
tkyview.moveto(widget, ...)
tkyview.scroll(widget, ...)
\end{verbatim}
\end{Usage}
%
\begin{Arguments}
\begin{ldescription}
\item[\code{widget}] The widget this applies to
\item[\code{...}] Handled via \code{.Tcl.args}
\end{ldescription}
\end{Arguments}
%
\begin{Details}\relax
There are far too many of these commands to describe them and their
arguments in full. Please refer to the Tcl/Tk documentation for details.
Except for a few exceptions, the pattern is that  Tcl widget commands
possibly with subcommands like
\code{.a.b selection clear} are converted to function names like
\code{tkselection.clear} and the widget is given as the first argument. 
\end{Details}
%
\begin{SeeAlso}\relax
\code{\LinkA{TclInterface}{TclInterface}}, \code{\LinkA{TkWidgets}{TkWidgets}},
\code{\LinkA{TkCommands}{TkCommands}}
\end{SeeAlso}
%
\begin{Examples}
\begin{ExampleCode}
## Not run: 
## These cannot be run by examples() but should be OK when pasted
## into an interactive R session with the tcltk package loaded

tt <- tktoplevel()
tkpack(txt.w <- tktext(tt))
tkinsert(txt.w, "0.0", "plot(1:10)")

# callback function 
eval.txt <- function()
   eval(parse(text=tclvalue(tkget(txt.w, "0.0", "end"))))
tkpack(but.w <- tkbutton(tt,text="Submit", command=eval.txt))

## Try pressing the button, edit the text and when finished:

tkdestroy(tt)

## End(Not run)

\end{ExampleCode}
\end{Examples}
\HeaderA{TkWidgets}{Tk widgets}{TkWidgets}
\aliasA{tkbutton}{TkWidgets}{tkbutton}
\aliasA{tkcanvas}{TkWidgets}{tkcanvas}
\aliasA{tkcheckbutton}{TkWidgets}{tkcheckbutton}
\aliasA{tkentry}{TkWidgets}{tkentry}
\aliasA{tkframe}{TkWidgets}{tkframe}
\aliasA{tklabel}{TkWidgets}{tklabel}
\aliasA{tklistbox}{TkWidgets}{tklistbox}
\aliasA{tkmenu}{TkWidgets}{tkmenu}
\aliasA{tkmenubutton}{TkWidgets}{tkmenubutton}
\aliasA{tkmessage}{TkWidgets}{tkmessage}
\aliasA{tkradiobutton}{TkWidgets}{tkradiobutton}
\aliasA{tkscale}{TkWidgets}{tkscale}
\aliasA{tkscrollbar}{TkWidgets}{tkscrollbar}
\aliasA{tktext}{TkWidgets}{tktext}
\aliasA{tktoplevel}{TkWidgets}{tktoplevel}
\aliasA{tkwidget}{TkWidgets}{tkwidget}
\aliasA{ttkbutton}{TkWidgets}{ttkbutton}
\aliasA{ttkcheckbutton}{TkWidgets}{ttkcheckbutton}
\aliasA{ttkcombobox}{TkWidgets}{ttkcombobox}
\aliasA{ttkentry}{TkWidgets}{ttkentry}
\aliasA{ttkframe}{TkWidgets}{ttkframe}
\aliasA{ttkimage}{TkWidgets}{ttkimage}
\aliasA{ttklabel}{TkWidgets}{ttklabel}
\aliasA{ttklabelframe}{TkWidgets}{ttklabelframe}
\aliasA{ttkmenubutton}{TkWidgets}{ttkmenubutton}
\aliasA{ttknotebook}{TkWidgets}{ttknotebook}
\aliasA{ttkpanedwindow}{TkWidgets}{ttkpanedwindow}
\aliasA{ttkprogressbar}{TkWidgets}{ttkprogressbar}
\aliasA{ttkradiobutton}{TkWidgets}{ttkradiobutton}
\aliasA{ttkscrollbar}{TkWidgets}{ttkscrollbar}
\aliasA{ttkseparator}{TkWidgets}{ttkseparator}
\aliasA{ttksizegrip}{TkWidgets}{ttksizegrip}
\aliasA{ttktreeview}{TkWidgets}{ttktreeview}
\keyword{misc}{TkWidgets}
%
\begin{Description}\relax
Create Tk widgets and associated \R{} objects.
\end{Description}
%
\begin{Usage}
\begin{verbatim}
tkwidget(parent, type, ...)

tkbutton(parent, ...)
tkcanvas(parent, ...)
tkcheckbutton(parent, ...)
tkentry(parent, ...)
ttkentry(parent, ...)
tkframe(parent, ...)
tklabel(parent, ...)
tklistbox(parent, ...)
tkmenu(parent, ...)
tkmenubutton(parent, ...)
tkmessage(parent, ...)
tkradiobutton(parent, ...)
tkscale(parent, ...)
tkscrollbar(parent, ...)
tktext(parent, ...)
tktoplevel(parent = .TkRoot, ...)

ttkbutton(parent, ...)
ttkcheckbutton(parent, ...)
ttkcombobox(parent, ...)
ttkframe(parent, ...)
ttkimage(parent, ...)
ttklabel(parent, ...)
ttklabelframe(parent, ...)
ttkmenubutton(parent, ...)
ttknotebook(parent, ...)
ttkpanedwindow(parent, ...)
ttkprogressbar(parent, ...)
ttkradiobutton(parent, ...)
ttkscrollbar(parent, ...)
ttkseparator(parent, ...)
ttksizegrip(parent, ...)
ttktreeview(parent, ...)
\end{verbatim}
\end{Usage}
%
\begin{Arguments}
\begin{ldescription}
\item[\code{parent}] Parent of widget window.
\item[\code{type}] string describing the type of widget desired.
\item[\code{...}] handled via \code{\LinkA{.Tcl.args}{.Tcl.args}}.
\end{ldescription}
\end{Arguments}
%
\begin{Details}\relax
These functions create Tk widgets.  \code{tkwidget} creates a widget of
a given type, the others simply call \code{tkwidget} with the
respective \code{type} argument.

The functions starting \code{ttk} are for the themed widget set for Tk
8.5 or later.  A tutorial can be found at \url{http://www.tkdocs.com}.

It is not possible to describe the widgets and their arguments in
full.  Please refer to the Tcl/Tk documentation.
\end{Details}
%
\begin{SeeAlso}\relax
\code{\LinkA{TclInterface}{TclInterface}}, \code{\LinkA{TkCommands}{TkCommands}},
\code{\LinkA{TkWidgetcmds}{TkWidgetcmds}}
\end{SeeAlso}
%
\begin{Examples}
\begin{ExampleCode}
## Not run: 
## These cannot be run by examples() but should be OK when pasted
## into an interactive R session with the tcltk package loaded

tt <- tktoplevel()
label.widget <- tklabel(tt, text="Hello, World!")
button.widget <- tkbutton(tt, text="Push",
                          command=function()cat("OW!\n"))
tkpack(label.widget, button.widget) # geometry manager
                                    # see Tk-commands

## Push the button and then...

tkdestroy(tt)

## test for themed widgets
if(as.character(tcl("info", "tclversion")) >= "8.5") {
  # make use of themed widgets
  # list themes
  as.character(tcl("ttk::style", "theme", "names"))
  # select a theme -- here pre-XP windows
  tcl("ttk::style", "theme use", "winnative")
} else {
  # use Tk 8.0 widgets
}

## End(Not run)
\end{ExampleCode}
\end{Examples}
\HeaderA{tk\_choose.dir}{Choose a Folder Interactively}{tk.Rul.choose.dir}
\keyword{file}{tk\_choose.dir}
%
\begin{Description}\relax
Use a Tk widget to choose a directory interactively.
\end{Description}
%
\begin{Usage}
\begin{verbatim}
tk_choose.dir(default = "", caption = "Select directory")
\end{verbatim}
\end{Usage}
%
\begin{Arguments}
\begin{ldescription}
\item[\code{default}] which directory to show initially.
\item[\code{caption}] the caption on the selection dialog.
\end{ldescription}
\end{Arguments}
%
\begin{Value}
A length-one character vector, character \code{NA} if
`Cancel' was selected. 
\end{Value}
%
\begin{SeeAlso}\relax
\code{\LinkA{tk\_choose.files}{tk.Rul.choose.files}}
\end{SeeAlso}
%
\begin{Examples}
\begin{ExampleCode}
## Not run: 
tk_choose.dir(getwd(), "Choose a suitable folder")

## End(Not run)
\end{ExampleCode}
\end{Examples}
\HeaderA{tk\_choose.files}{Choose a List of Files Interactively}{tk.Rul.choose.files}
\keyword{file}{tk\_choose.files}
%
\begin{Description}\relax
Use a Tk file dialog to choose a list of zero or more files 
interactively.
\end{Description}
%
\begin{Usage}
\begin{verbatim}
tk_choose.files(default = "", caption = "Select files",
                multi = TRUE, filters = NULL, index = 1)
\end{verbatim}
\end{Usage}
%
\begin{Arguments}
\begin{ldescription}
\item[\code{default}] which filename to show initially.
\item[\code{caption}] the caption on the file selection dialog.
\item[\code{multi}] whether to allow multiple files to be selected.
\item[\code{filters}] two-column character matrix of filename filters.
\item[\code{index}] unused.
\end{ldescription}
\end{Arguments}
%
\begin{Details}\relax
Unlike \code{\LinkA{file.choose}{file.choose}}, \code{tk\_choose.files} will always
attempt to return a character vector giving a list of files.  If the
user cancels the dialog, then zero files are returned, whereas
\code{\LinkA{file.choose}{file.choose}} would signal an error.

The format of \code{filters} can be seen from the example.  File
patterns are specified via extensions, with \code{"*"} meaning any
file, and \code{""} any file without an extension (a filename not
containing a period).  (Other forms may work on specific platforms.)
Note that the way to have multiple extensions for one file type is to
have multiple rows with the same name in the first column, and that
whether the extensions are named in file chooser widget is
platform-specific. \bold{The format may change before release.}
\end{Details}
%
\begin{Value}
A character vector giving zero or more file paths.
\end{Value}
%
\begin{Note}\relax
A bug in Tk 8.5.0--8.5.4 prevented multiple selections being used.
\end{Note}
%
\begin{SeeAlso}\relax
\code{\LinkA{file.choose}{file.choose}}, \code{\LinkA{tk\_choose.dir}{tk.Rul.choose.dir}}
\end{SeeAlso}
%
\begin{Examples}
\begin{ExampleCode}
Filters <- matrix(c("R code", ".R", "R code", ".s",
                    "Text", ".txt", "All files", "*"),
                  4, 2, byrow = TRUE)
Filters
if(interactive()) tk_choose.files(filter = Filters)
\end{ExampleCode}
\end{Examples}
\HeaderA{tk\_messageBox}{Tk Message Box}{tk.Rul.messageBox}
\keyword{utilities}{tk\_messageBox}
%
\begin{Description}\relax
An implementation of a generic message box using Tk
\end{Description}
%
\begin{Usage}
\begin{verbatim}
tk_messageBox(type = c("ok", "okcancel", "yesno", "yesnocancel",
                       "retrycancel", "aburtretrycancel"),
              message, caption = "", default = "", ...)
\end{verbatim}
\end{Usage}
%
\begin{Arguments}
\begin{ldescription}
\item[\code{type}] character. The type of dialog box. It will have the
buttons implied by its name.
\item[\code{message}] character. The information field of the dialog box.
\item[\code{caption}] the caption on the widget displayed.
\item[\code{default}] character. The name of the button to be used as the
default.
\item[\code{...}] additional named arguments to be passed to the Tk
function of this name.  An example is \code{icon="warning"}.
\end{ldescription}
\end{Arguments}
%
\begin{Value}
A character string giving the name of the button pressed.
\end{Value}
%
\begin{SeeAlso}\relax
\code{\LinkA{tkmessageBox}{tkmessageBox}} for a `raw' interface.
\end{SeeAlso}
\HeaderA{tk\_select.list}{Select Items from a List}{tk.Rul.select.list}
\keyword{utilities}{tk\_select.list}
%
\begin{Description}\relax
Select item(s) from a character vector using a Tk listbox.
\end{Description}
%
\begin{Usage}
\begin{verbatim}
tk_select.list(choices, preselect = NULL, multiple = FALSE, title = NULL)
\end{verbatim}
\end{Usage}
%
\begin{Arguments}
\begin{ldescription}
\item[\code{choices}] a character vector of items.
\item[\code{preselect}] a character vector, or \code{NULL}.  If non-null and
if the string(s) appear in the list, the item(s) are selected
initially.
\item[\code{multiple}] logical: can more than one item be selected?
\item[\code{title}] optional character string for window title, or
\code{NULL} for no title.
\end{ldescription}
\end{Arguments}
%
\begin{Details}\relax
This is a version of \code{\LinkA{select.list}{select.list}} implemented as a Tk
list box plus \code{OK} and \code{Cancel} buttons.  There will be a
scrollbar if the list is too long to fit comfortably on the screen.

The dialog box is \emph{modal}, so a selection must be made or
cancelled before the \R{} session can proceed.  As from \R{} 2.10.1
double-clicking on an item is equivalent to selecting it and then
clicking \code{OK}.

If Tk is version 8.5 or later, themed widgets will be used.
\end{Details}
%
\begin{Value}
A character vector of selected items.  If \code{multiple} is false and
no item was selected (or \code{Cancel} was used), \code{""} is
returned.   If \code{multiple} is true and no item was selected (or
\code{Cancel} was used) then a character vector of length 0 is returned.
\end{Value}
%
\begin{SeeAlso}\relax
\code{\LinkA{select.list}{select.list}} (a text version except on Windows and the
Mac OS X GUI),
\code{\LinkA{menu}{menu}} (whose \code{graphics=TRUE} mode uses this
on most Unix-alikes).
\end{SeeAlso}
\clearpage
