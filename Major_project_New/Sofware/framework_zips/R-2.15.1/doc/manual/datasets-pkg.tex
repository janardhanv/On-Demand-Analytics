
\chapter{The \texttt{datasets} package}
\HeaderA{datasets-package}{The R Datasets Package}{datasets.Rdash.package}
\aliasA{datasets}{datasets-package}{datasets}
\keyword{package}{datasets-package}
%
\begin{Description}\relax
Base R datasets
\end{Description}
%
\begin{Details}\relax
This package contains a variety of datasets.  For a complete
list, use \code{library(help="datasets")}.
\end{Details}
%
\begin{Author}\relax
R Core Team and contributors worldwide

Maintainer: R Core Team \email{R-core@r-project.org}
\end{Author}
\HeaderA{ability.cov}{Ability and Intelligence Tests}{ability.cov}
\keyword{datasets}{ability.cov}
%
\begin{Description}\relax
Six tests were given to 112 individuals. The covariance matrix is
given in this object.
\end{Description}
%
\begin{Usage}
\begin{verbatim}
ability.cov
\end{verbatim}
\end{Usage}
%
\begin{Details}\relax
The tests are described as
\begin{description}

\item[general:] a non-verbal measure of general intelligence using
Cattell's culture-fair test.
\item[picture:] a picture-completion test
\item[blocks:] block design
\item[maze:] mazes
\item[reading:] reading comprehension
\item[vocab:] vocabulary

\end{description}


Bartholomew gives both covariance and correlation matrices,
but these are inconsistent.  Neither are in the original paper.
\end{Details}
%
\begin{Source}\relax
Bartholomew, D. J. (1987) \emph{Latent Variable Analysis and Factor
Analysis.} Griffin.  

Bartholomew, D. J.  and Knott, M. (1990) \emph{Latent Variable Analysis
and Factor Analysis.} Second Edition, Arnold.  
\end{Source}
%
\begin{References}\relax
Smith, G. A. and Stanley G. (1983)
Clocking \eqn{g}{}: relating intelligence and measures of timed
performance. \emph{Intelligence}, \bold{7}, 353--368.
\end{References}
%
\begin{Examples}
\begin{ExampleCode}

require(stats)
(ability.FA <- factanal(factors = 1, covmat=ability.cov))
update(ability.FA, factors = 2)
## The signs of factors and hence the signs of correlations are
## arbitrary with promax rotation.
update(ability.FA, factors = 2, rotation = "promax")
\end{ExampleCode}
\end{Examples}
\HeaderA{airmiles}{Passenger Miles on Commercial US Airlines, 1937--1960}{airmiles}
\keyword{datasets}{airmiles}
%
\begin{Description}\relax
The revenue passenger miles flown by commercial airlines in
the United States for each year from 1937 to 1960.
\end{Description}
%
\begin{Usage}
\begin{verbatim}
airmiles
\end{verbatim}
\end{Usage}
%
\begin{Format}
A time series of 24 observations; yearly, 1937--1960.
\end{Format}
%
\begin{Source}\relax
F.A.A. Statistical Handbook of Aviation.
\end{Source}
%
\begin{References}\relax
Brown, R. G. (1963)
\emph{Smoothing, Forecasting and Prediction of Discrete Time Series}.
Prentice-Hall.
\end{References}
%
\begin{Examples}
\begin{ExampleCode}
require(graphics)
plot(airmiles, main = "airmiles data",
     xlab = "Passenger-miles flown by U.S. commercial airlines", col = 4)
\end{ExampleCode}
\end{Examples}
\HeaderA{AirPassengers}{Monthly Airline Passenger Numbers 1949-1960}{AirPassengers}
\keyword{datasets}{AirPassengers}
%
\begin{Description}\relax
The classic Box \& Jenkins airline data.  Monthly totals of
international airline passengers, 1949 to 1960.
\end{Description}
%
\begin{Usage}
\begin{verbatim}
AirPassengers
\end{verbatim}
\end{Usage}
%
\begin{Format}
A monthly time series, in thousands.
\end{Format}
%
\begin{Source}\relax
Box, G. E. P., Jenkins, G. M. and Reinsel, G. C. (1976)
\emph{Time Series Analysis, Forecasting and Control.}
Third Edition. Holden-Day. Series G.
\end{Source}
%
\begin{Examples}
\begin{ExampleCode}
## Not run: 
## These are quite slow and so not run by example(AirPassengers)

## The classic 'airline model', by full ML
(fit <- arima(log10(AirPassengers), c(0, 1, 1),
              seasonal = list(order=c(0, 1 ,1), period=12)))
update(fit, method = "CSS")
update(fit, x=window(log10(AirPassengers), start = 1954))
pred <- predict(fit, n.ahead = 24)
tl <- pred$pred - 1.96 * pred$se
tu <- pred$pred + 1.96 * pred$se
ts.plot(AirPassengers, 10^tl, 10^tu, log = "y", lty = c(1,2,2))

## full ML fit is the same if the series is reversed, CSS fit is not
ap0 <- rev(log10(AirPassengers))
attributes(ap0) <- attributes(AirPassengers)
arima(ap0, c(0, 1, 1), seasonal = list(order=c(0, 1 ,1), period=12))
arima(ap0, c(0, 1, 1), seasonal = list(order=c(0, 1 ,1), period=12),
      method = "CSS")

## Structural Time Series
ap <- log10(AirPassengers) - 2
(fit <- StructTS(ap, type= "BSM"))
par(mfrow=c(1,2))
plot(cbind(ap, fitted(fit)), plot.type = "single")
plot(cbind(ap, tsSmooth(fit)), plot.type = "single")

## End(Not run)
\end{ExampleCode}
\end{Examples}
\HeaderA{airquality}{New York Air Quality Measurements}{airquality}
\keyword{datasets}{airquality}
%
\begin{Description}\relax
Daily air quality measurements in New York, May to September 1973.
\end{Description}
%
\begin{Usage}
\begin{verbatim}
airquality
\end{verbatim}
\end{Usage}
%
\begin{Format}
A data frame with 154 observations on 6 variables.


\Tabular{rlll}{
\code{[,1]} & \code{Ozone}   & numeric & Ozone (ppb)\\{}
\code{[,2]} & \code{Solar.R} & numeric & Solar R (lang)\\{}
\code{[,3]} & \code{Wind}    & numeric & Wind (mph)\\{}
\code{[,4]} & \code{Temp}    & numeric & Temperature (degrees F)\\{}
\code{[,5]} & \code{Month}   & numeric & Month (1--12)\\{}
\code{[,6]} & \code{Day}     & numeric & Day of month (1--31)
}
\end{Format}
%
\begin{Details}\relax
Daily readings of the following air quality values for May 1, 1973 (a
Tuesday) to September 30, 1973.

\begin{itemize}

\item \code{Ozone}: Mean ozone in parts per
billion from 1300 to 1500 hours at Roosevelt Island
\item \code{Solar.R}: Solar radiation
in Langleys in the frequency band 4000--7700 Angstroms from
0800 to 1200 hours at Central Park
\item \code{Wind}: Average wind speed in miles
per hour at 0700 and 1000 hours at LaGuardia Airport
\item \code{Temp}: Maximum daily
temperature in degrees Fahrenheit at La Guardia Airport.

\end{itemize}

\end{Details}
%
\begin{Source}\relax
The data were obtained from the New York State Department of
Conservation (ozone data) and the National Weather Service
(meteorological data).
\end{Source}
%
\begin{References}\relax
Chambers, J. M., Cleveland, W. S., Kleiner, B. and Tukey, P. A. (1983)
\emph{Graphical Methods for Data Analysis}.
Belmont, CA: Wadsworth.
\end{References}
%
\begin{Examples}
\begin{ExampleCode}
require(graphics)
pairs(airquality, panel = panel.smooth, main = "airquality data")
\end{ExampleCode}
\end{Examples}
\HeaderA{anscombe}{Anscombe's Quartet of `Identical' Simple Linear Regressions}{anscombe}
\keyword{datasets}{anscombe}
%
\begin{Description}\relax
Four \eqn{x}{}-\eqn{y}{} datasets which have the same traditional
statistical properties (mean, variance, correlation, regression line,
etc.), yet are quite different.
\end{Description}
%
\begin{Usage}
\begin{verbatim}
anscombe
\end{verbatim}
\end{Usage}
%
\begin{Format}
A data frame with 11 observations on 8 variables.

\Tabular{rl}{
x1 == x2 == x3 & the integers 4:14, specially arranged \\{}
x4             & values 8 and 19 \\{}
y1, y2, y3, y4 & numbers in (3, 12.5) with mean 7.5 and sdev 2.03}
\end{Format}
%
\begin{Source}\relax
Tufte, Edward R. (1989)
\emph{The Visual Display of Quantitative Information}, 13--14.
Graphics Press.
\end{Source}
%
\begin{References}\relax
Anscombe, Francis J. (1973)  Graphs in statistical analysis.
\emph{American Statistician}, \bold{27}, 17--21.
\end{References}
%
\begin{Examples}
\begin{ExampleCode}
require(stats); require(graphics)
summary(anscombe)

##-- now some "magic" to do the 4 regressions in a loop:
ff <- y ~ x
mods <- setNames(as.list(1:4), paste0("lm", 1:4))
for(i in 1:4) {
  ff[2:3] <- lapply(paste0(c("y","x"), i), as.name)
  ## or   ff[[2]] <- as.name(paste0("y", i))
  ##      ff[[3]] <- as.name(paste0("x", i))
  mods[[i]] <- lmi <- lm(ff, data= anscombe)
  print(anova(lmi))
}

## See how close they are (numerically!)
sapply(mods, coef)
lapply(mods, function(fm) coef(summary(fm)))

## Now, do what you should have done in the first place: PLOTS
op <- par(mfrow=c(2,2), mar=.1+c(4,4,1,1), oma= c(0,0,2,0))
for(i in 1:4) {
  ff[2:3] <- lapply(paste0(c("y","x"), i), as.name)
  plot(ff, data=anscombe, col="red", pch=21, bg = "orange", cex = 1.2,
       xlim=c(3,19), ylim=c(3,13))
  abline(mods[[i]], col="blue")
}
mtext("Anscombe's 4 Regression data sets", outer = TRUE, cex=1.5)
par(op)
\end{ExampleCode}
\end{Examples}
\HeaderA{attenu}{The Joyner--Boore Attenuation Data}{attenu}
\keyword{datasets}{attenu}
%
\begin{Description}\relax
This data gives peak accelerations measured at various observation
stations for 23 earthquakes in California.  The data have been used
by various workers to estimate the attenuating affect of distance
on ground acceleration.
\end{Description}
%
\begin{Usage}
\begin{verbatim}
attenu
\end{verbatim}
\end{Usage}
%
\begin{Format}
A data frame with 182 observations on 5 variables.

\Tabular{rlll}{
[,1] & event   & numeric & Event Number\\{}
[,2] & mag     & numeric & Moment Magnitude\\{}
[,3] & station & factor  & Station Number\\{}
[,4] & dist    & numeric & Station-hypocenter distance (km)\\{}
[,5] & accel   & numeric & Peak acceleration (g)}
\end{Format}
%
\begin{Source}\relax
Joyner, W.B., D.M. Boore and R.D. Porcella (1981).  Peak horizontal
acceleration and velocity from strong-motion records including
records from the 1979 Imperial Valley, California earthquake.  USGS
Open File report 81-365. Menlo Park, Ca.
\end{Source}
%
\begin{References}\relax
Boore, D. M. and Joyner, W.B.(1982)
The empirical prediction of ground motion,
\emph{Bull. Seism. Soc. Am.}, \bold{72}, S269--S268.

Bolt, B. A. and Abrahamson, N. A. (1982)
New attenuation relations for peak and expected accelerations of
strong ground motion,
\emph{Bull. Seism. Soc. Am.}, \bold{72}, 2307--2321.

Bolt B. A. and Abrahamson, N. A. (1983)
Reply to W. B. Joyner \& D. M. Boore's ``Comments on: New
attenuation relations for peak and expected accelerations for peak
and expected accelerations of strong ground motion'',
\emph{Bull. Seism. Soc. Am.}, \bold{73}, 1481--1483. 

Brillinger, D. R. and Preisler, H. K. (1984)
An exploratory analysis of the Joyner-Boore attenuation data,
\emph{Bull. Seism. Soc. Am.}, \bold{74}, 1441--1449.

Brillinger, D. R. and Preisler, H. K. (1984)
\emph{Further analysis of the Joyner-Boore attenuation data}.
Manuscript.
\end{References}
%
\begin{Examples}
\begin{ExampleCode}
require(graphics)
## check the data class of the variables
sapply(attenu, data.class)
summary(attenu)
pairs(attenu, main = "attenu data")
coplot(accel ~ dist | as.factor(event), data = attenu, show.given = FALSE)
coplot(log(accel) ~ log(dist) | as.factor(event),
       data = attenu, panel = panel.smooth, show.given = FALSE)
\end{ExampleCode}
\end{Examples}
\HeaderA{attitude}{The Chatterjee--Price Attitude Data}{attitude}
\keyword{datasets}{attitude}
%
\begin{Description}\relax
From a survey of the clerical employees of a large financial
organization, the data are aggregated from the questionnaires of the
approximately 35 employees for each of 30 (randomly selected)
departments.  The numbers give the percent proportion of favourable
responses to seven questions in each department.
\end{Description}
%
\begin{Usage}
\begin{verbatim}
attitude
\end{verbatim}
\end{Usage}
%
\begin{Format}
A dataframe with 30 observations on 7 variables. The first column are
the short names from the reference, the second one the variable names
in the data frame:

\Tabular{rlll}{
Y   & rating    & numeric  & Overall rating \\{}
X[1] & complaints& numeric & Handling of employee complaints \\{}
X[2] & privileges& numeric & Does not allow special privileges \\{}
X[3] & learning & numeric  & Opportunity to learn \\{}
X[4] & raises   & numeric  & Raises based on performance \\{}
X[5] & critical & numeric  & Too critical \\{}
X[6] & advancel & numeric  & Advancement}
\end{Format}
%
\begin{Source}\relax
Chatterjee, S. and Price, B. (1977)
\emph{Regression Analysis by Example}.
New York: Wiley.
(Section 3.7, p.68ff of 2nd ed.(1991).)
\end{Source}
%
\begin{Examples}
\begin{ExampleCode}
require(stats); require(graphics)
pairs(attitude, main = "attitude data")
summary(attitude)
summary(fm1 <- lm(rating ~ ., data = attitude))
opar <- par(mfrow = c(2, 2), oma = c(0, 0, 1.1, 0),
            mar = c(4.1, 4.1, 2.1, 1.1))
plot(fm1)
summary(fm2 <- lm(rating ~ complaints, data = attitude))
plot(fm2)
par(opar)
\end{ExampleCode}
\end{Examples}
\HeaderA{austres}{Quarterly Time Series of the Number of Australian Residents}{austres}
\keyword{datasets}{austres}
%
\begin{Description}\relax
Numbers (in thousands) of Australian residents measured quarterly from
March 1971 to March 1994.  The object is of class \code{"ts"}.
\end{Description}
%
\begin{Usage}
\begin{verbatim}
austres
\end{verbatim}
\end{Usage}
%
\begin{Source}\relax
P. J. Brockwell and R. A. Davis (1996)
\emph{Introduction to Time Series and Forecasting.}
Springer
\end{Source}
\HeaderA{beavers}{Body Temperature Series of Two Beavers}{beavers}
\aliasA{beaver1}{beavers}{beaver1}
\aliasA{beaver2}{beavers}{beaver2}
\keyword{datasets}{beavers}
%
\begin{Description}\relax
Reynolds (1994) describes a small part of a study of the long-term
temperature dynamics of beaver \emph{Castor canadensis} in
north-central Wisconsin.  Body temperature was measured by telemetry
every 10 minutes for four females, but data from a one period of
less than a day for each of two animals is used there. 
\end{Description}
%
\begin{Usage}
\begin{verbatim}
beaver1
beaver2
\end{verbatim}
\end{Usage}
%
\begin{Format}
The \code{beaver1} data frame has 114 rows and 4 columns on body
temperature measurements at 10 minute intervals.

The \code{beaver2} data frame has 100 rows and 4 columns on body
temperature measurements at 10 minute intervals.

The variables are as follows:
\begin{description}

\item[day] Day of observation (in days since the beginning of
1990), December 12--13 (\code{beaver1}) and November 3--4
(\code{beaver2}).
\item[time] Time of observation, in the form \code{0330} for
3:30am
\item[temp] Measured body temperature in degrees Celsius.
\item[activ] Indicator of activity outside the retreat.

\end{description}

\end{Format}
%
\begin{Note}\relax
The observation at 22:20 is missing in \code{beaver1}.
\end{Note}
%
\begin{Source}\relax
P. S. Reynolds (1994) Time-series analyses of beaver body
temperatures.  Chapter 11 of Lange, N., Ryan, L., Billard, L.,
Brillinger, D., Conquest, L. and Greenhouse, J. eds (1994)
\emph{Case Studies in Biometry.}
New York: John Wiley and Sons.
\end{Source}
%
\begin{Examples}
\begin{ExampleCode}
require(graphics)
(yl <- range(beaver1$temp, beaver2$temp))

beaver.plot <- function(bdat, ...) {
  nam <- deparse(substitute(bdat))
  with(bdat, {
    # Hours since start of day:
    hours <- time %/% 100 + 24*(day - day[1]) + (time %% 100)/60
    plot (hours, temp, type = "l", ...,
          main = paste(nam, "body temperature"))
    abline(h = 37.5, col = "gray", lty = 2)
    is.act <- activ == 1
    points(hours[is.act], temp[is.act], col = 2, cex = .8)
  })
}
op <- par(mfrow = c(2,1), mar = c(3,3,4,2), mgp = .9* 2:0)
 beaver.plot(beaver1, ylim = yl)
 beaver.plot(beaver2, ylim = yl)
par(op)
\end{ExampleCode}
\end{Examples}
\HeaderA{BJsales}{Sales Data with Leading Indicator}{BJsales}
\methaliasA{BJsales.lead}{BJsales}{BJsales.lead}
\keyword{datasets}{BJsales}
%
\begin{Description}\relax
The sales time series \code{BJsales} and leading indicator
\code{BJsales.lead} each contain 150 observations.
The objects are of class \code{"ts"}.
\end{Description}
%
\begin{Usage}
\begin{verbatim}
BJsales
BJsales.lead
\end{verbatim}
\end{Usage}
%
\begin{Source}\relax
The data are given in Box \& Jenkins (1976).
Obtained from the Time Series Data Library at
\url{http://www-personal.buseco.monash.edu.au/~hyndman/TSDL/}
\end{Source}
%
\begin{References}\relax
G. E. P. Box and G. M. Jenkins (1976):
\emph{Time Series Analysis, Forecasting and Control},
Holden-Day, San Francisco, p. 537.

P. J. Brockwell and R. A. Davis (1991):
\emph{Time Series: Theory and Methods},
Second edition, Springer Verlag, NY, pp. 414.
\end{References}
\HeaderA{BOD}{ Biochemical Oxygen Demand }{BOD}
\keyword{datasets}{BOD}
%
\begin{Description}\relax
The \code{BOD} data frame has 6 rows and 2 columns giving the
biochemical oxygen demand versus time in an evaluation of water
quality.
\end{Description}
%
\begin{Usage}
\begin{verbatim}
BOD
\end{verbatim}
\end{Usage}
%
\begin{Format}
This data frame contains the following columns:
\begin{description}

\item[Time] 
A numeric vector giving the time of the measurement (days).

\item[demand] 
A numeric vector giving the biochemical oxygen demand (mg/l).


\end{description}

\end{Format}
%
\begin{Source}\relax
Bates, D.M. and Watts, D.G. (1988),
\emph{Nonlinear Regression Analysis and Its Applications},
Wiley, Appendix A1.4.

Originally from Marske (1967), \emph{Biochemical
Oxygen Demand Data Interpretation Using Sum of Squares Surface}
M.Sc. Thesis, University of Wisconsin -- Madison.
\end{Source}
%
\begin{Examples}
\begin{ExampleCode}

require(stats)
# simplest form of fitting a first-order model to these data
fm1 <- nls(demand ~ A*(1-exp(-exp(lrc)*Time)), data = BOD,
   start = c(A = 20, lrc = log(.35)))
coef(fm1)
fm1
# using the plinear algorithm
fm2 <- nls(demand ~ (1-exp(-exp(lrc)*Time)), data = BOD,
   start = c(lrc = log(.35)), algorithm = "plinear", trace = TRUE)
# using a self-starting model
fm3 <- nls(demand ~ SSasympOrig(Time, A, lrc), data = BOD)
summary(fm3)

\end{ExampleCode}
\end{Examples}
\HeaderA{cars}{Speed and Stopping Distances of Cars}{cars}
\keyword{datasets}{cars}
%
\begin{Description}\relax
The data give the speed of cars and the distances taken to stop.
Note that the data were recorded in the 1920s.
\end{Description}
%
\begin{Usage}
\begin{verbatim}
cars
\end{verbatim}
\end{Usage}
%
\begin{Format}
A data frame with 50 observations on 2 variables.

\Tabular{rlll}{
[,1]  & speed  & numeric  & Speed (mph)\\{}
[,2]  & dist   & numeric  & Stopping distance (ft)
}
\end{Format}
%
\begin{Source}\relax
Ezekiel, M. (1930)
\emph{Methods of Correlation Analysis}.
Wiley.
\end{Source}
%
\begin{References}\relax
McNeil, D. R. (1977)
\emph{Interactive Data Analysis}.
Wiley.
\end{References}
%
\begin{Examples}
\begin{ExampleCode}
require(stats); require(graphics)
plot(cars, xlab = "Speed (mph)", ylab = "Stopping distance (ft)",
     las = 1)
lines(lowess(cars$speed, cars$dist, f = 2/3, iter = 3), col = "red")
title(main = "cars data")
plot(cars, xlab = "Speed (mph)", ylab = "Stopping distance (ft)",
     las = 1, log = "xy")
title(main = "cars data (logarithmic scales)")
lines(lowess(cars$speed, cars$dist, f = 2/3, iter = 3), col = "red")
summary(fm1 <- lm(log(dist) ~ log(speed), data = cars))
opar <- par(mfrow = c(2, 2), oma = c(0, 0, 1.1, 0),
            mar = c(4.1, 4.1, 2.1, 1.1))
plot(fm1)
par(opar)

## An example of polynomial regression
plot(cars, xlab = "Speed (mph)", ylab = "Stopping distance (ft)",
    las = 1, xlim = c(0, 25))
d <- seq(0, 25, length.out = 200)
for(degree in 1:4) {
  fm <- lm(dist ~ poly(speed, degree), data = cars)
  assign(paste("cars", degree, sep="."), fm)
  lines(d, predict(fm, data.frame(speed=d)), col = degree)
}
anova(cars.1, cars.2, cars.3, cars.4)
\end{ExampleCode}
\end{Examples}
\HeaderA{ChickWeight}{Weight versus age of chicks on different diets}{ChickWeight}
\keyword{datasets}{ChickWeight}
%
\begin{Description}\relax
The \code{ChickWeight} data frame has 578 rows and 4 columns from an
experiment on the effect of diet on early growth of chicks.
\end{Description}
%
\begin{Usage}
\begin{verbatim}
ChickWeight
\end{verbatim}
\end{Usage}
%
\begin{Format}
This object of class \code{c("nfnGroupedData", "nfGroupedData",
    "groupedData", "data.frame")} containing the following columns:
\begin{description}

\item[weight] 
a numeric vector giving the body weight of the chick (gm).

\item[Time] 
a numeric vector giving the number of days since birth when
the measurement was made.

\item[Chick] 
an ordered factor with levels
\code{18} < \dots < \code{48}
giving a unique identifier for the chick.  The ordering of
the levels groups chicks on the same diet together and
orders them according to their final weight (lightest to
heaviest) within diet.

\item[Diet] 
a factor with levels 1,\dots,4 indicating which
experimental diet the chick received.


\end{description}

\end{Format}
%
\begin{Details}\relax
The body weights of the chicks were measured at birth and every
second day thereafter until day 20.  They were also measured on day
21.  There were four groups on chicks on different protein diets.

This dataset was originally part of package \code{nlme}, and that has
methods (including for \code{[}, \code{as.data.frame}, \code{plot} and
\code{print}) for its grouped-data classes. 
\end{Details}
%
\begin{Source}\relax
Crowder, M. and Hand, D. (1990), \emph{Analysis of Repeated Measures},
Chapman and Hall (example 5.3)

Hand, D. and Crowder, M. (1996), \emph{Practical Longitudinal Data
Analysis}, Chapman and Hall (table A.2)

Pinheiro, J. C. and Bates, D. M. (2000) \emph{Mixed-effects Models in
S and S-PLUS}, Springer.
\end{Source}
%
\begin{SeeAlso}\relax
\code{\LinkA{SSlogis}{SSlogis}} for models fitted to this dataset.
\end{SeeAlso}
%
\begin{Examples}
\begin{ExampleCode}

require(graphics)
coplot(weight ~ Time | Chick, data = ChickWeight,
       type = "b", show.given = FALSE)
\end{ExampleCode}
\end{Examples}
\HeaderA{chickwts}{Chicken Weights by Feed Type}{chickwts}
\keyword{datasets}{chickwts}
%
\begin{Description}\relax
An experiment was conducted to measure and compare the effectiveness
of various feed supplements on the growth rate of chickens.
\end{Description}
%
\begin{Usage}
\begin{verbatim}
chickwts
\end{verbatim}
\end{Usage}
%
\begin{Format}
A data frame with 71 observations on 2 variables.
\begin{description}

\item[weight] a numeric variable giving the chick weight.
\item[feed] a factor giving the feed type.

\end{description}

\end{Format}
%
\begin{Details}\relax
Newly hatched chicks were randomly allocated into six groups, and each
group was given a different feed supplement.  Their weights in grams
after six weeks are given along with feed types.
\end{Details}
%
\begin{Source}\relax
Anonymous (1948)
\emph{Biometrika}, \bold{35}, 214.
\end{Source}
%
\begin{References}\relax
McNeil, D. R. (1977)
\emph{Interactive Data Analysis}.
New York: Wiley.
\end{References}
%
\begin{Examples}
\begin{ExampleCode}
require(stats); require(graphics)
boxplot(weight ~ feed, data = chickwts, col = "lightgray",
    varwidth = TRUE, notch = TRUE, main = "chickwt data",
    ylab = "Weight at six weeks (gm)")
anova(fm1 <- lm(weight ~ feed, data = chickwts))
opar <- par(mfrow = c(2, 2), oma = c(0, 0, 1.1, 0),
            mar = c(4.1, 4.1, 2.1, 1.1))
plot(fm1)
par(opar)
\end{ExampleCode}
\end{Examples}
\HeaderA{CO2}{Carbon Dioxide Uptake in Grass Plants}{CO2}
\keyword{datasets}{CO2}
%
\begin{Description}\relax
The \code{CO2} data frame has 84 rows and 5 columns of data from an
experiment on the cold tolerance of the grass species
\emph{Echinochloa crus-galli}.
\end{Description}
%
\begin{Usage}
\begin{verbatim}
CO2
\end{verbatim}
\end{Usage}
%
\begin{Format}
This object of class \code{c("nfnGroupedData", "nfGroupedData",
    "groupedData", "data.frame")} containing the following columns:
\begin{description}

\item[Plant] 
an ordered factor with levels
\code{Qn1} < \code{Qn2} < \code{Qn3} < \dots < \code{Mc1}
giving a unique identifier for each plant.

\item[Type] 
a factor with levels
\code{Quebec} 
\code{Mississippi}
giving the origin of the plant

\item[Treatment] 
a factor with levels
\code{nonchilled} 
\code{chilled}

\item[conc] 
a numeric vector of ambient carbon dioxide concentrations (mL/L).

\item[uptake] 
a numeric vector of carbon dioxide uptake rates
(\eqn{\mu\mbox{mol}/m^2}{} sec).


\end{description}

\end{Format}
%
\begin{Details}\relax
The \eqn{CO_2}{} uptake of six plants from Quebec and six plants
from Mississippi was measured at several levels of ambient
\eqn{CO_2}{} concentration.  Half the plants of each type were
chilled overnight before the experiment was conducted.

This dataset was originally part of package \code{nlme}, and that has
methods (including for \code{[}, \code{as.data.frame}, \code{plot} and
\code{print}) for its grouped-data classes. 
\end{Details}
%
\begin{Source}\relax
Potvin, C., Lechowicz, M. J. and Tardif, S. (1990)
``The statistical analysis of ecophysiological response curves
obtained from experiments involving repeated measures'', \emph{Ecology},
\bold{71}, 1389--1400.

Pinheiro, J. C. and Bates, D. M. (2000)
\emph{Mixed-effects Models in S and S-PLUS}, Springer.
\end{Source}
%
\begin{Examples}
\begin{ExampleCode}
require(stats); require(graphics)

coplot(uptake ~ conc | Plant, data = CO2, show.given = FALSE, type = "b")
## fit the data for the first plant
fm1 <- nls(uptake ~ SSasymp(conc, Asym, lrc, c0),
   data = CO2, subset = Plant == 'Qn1')
summary(fm1)
## fit each plant separately
fmlist <- list()
for (pp in levels(CO2$Plant)) {
  fmlist[[pp]] <- nls(uptake ~ SSasymp(conc, Asym, lrc, c0),
      data = CO2, subset = Plant == pp)
}
## check the coefficients by plant
print(sapply(fmlist, coef), digits=3)
\end{ExampleCode}
\end{Examples}
\HeaderA{co2}{Mauna Loa Atmospheric CO2 Concentration}{co2}
\keyword{datasets}{co2}
%
\begin{Description}\relax
Atmospheric concentrations of CO\eqn{_2}{} are expressed in parts per
million (ppm) and reported in the preliminary 1997 SIO manometric mole
fraction scale.
\end{Description}
%
\begin{Usage}
\begin{verbatim}
co2
\end{verbatim}
\end{Usage}
%
\begin{Format}
A time series of 468 observations; monthly from 1959 to 1997.
\end{Format}
%
\begin{Details}\relax
The values for February, March and April of 1964 were missing and have
been obtained by interpolating linearly between the values for January
and May of 1964.
\end{Details}
%
\begin{Source}\relax
Keeling, C. D. and  Whorf, T. P.,
Scripps Institution of Oceanography (SIO),
University of California,
La Jolla, California USA 92093-0220.

\url{ftp://cdiac.esd.ornl.gov/pub/maunaloa-co2/maunaloa.co2}.
\end{Source}
%
\begin{References}\relax
Cleveland, W. S. (1993)
\emph{Visualizing Data}.
New Jersey: Summit Press.
\end{References}
%
\begin{Examples}
\begin{ExampleCode}
require(graphics)
plot(co2, ylab = expression("Atmospheric concentration of CO"[2]),
     las = 1)
title(main = "co2 data set")
\end{ExampleCode}
\end{Examples}
\HeaderA{crimtab}{Student's 3000 Criminals Data}{crimtab}
\keyword{datasets}{crimtab}
%
\begin{Description}\relax
Data of 3000 male criminals over 20 years old undergoing their
sentences in the chief prisons of England and Wales.
\end{Description}
%
\begin{Usage}
\begin{verbatim}
crimtab
\end{verbatim}
\end{Usage}
%
\begin{Format}
A \code{\LinkA{table}{table}} object of  \code{\LinkA{integer}{integer}} counts, of dimension
\eqn{42 \times 22}{} with a total count, \code{sum(crimtab)} of
3000.

The 42 \code{\LinkA{rownames}{rownames}} (\code{"9.4"}, \code{"9.5"}, \dots)
correspond to midpoints of intervals of finger lengths
whereas the 22 column names (\code{\LinkA{colnames}{colnames}})
(\code{"142.24"}, \code{"144.78"}, \dots) correspond to (body) heights
of 3000 criminals, see also below.
\end{Format}
%
\begin{Details}\relax
Student is the pseudonym of William Sealy Gosset.
In his 1908 paper he wrote (on page 13) at the beginning of section VI
entitled \emph{Practical Test of the forgoing Equations}:

``Before I had succeeded in solving my problem analytically,
I had endeavoured to do so empirically.  The material used was a
correlation table containing the height and left middle finger
measurements of 3000 criminals, from a paper by W. R. MacDonell
(\emph{Biometrika}, Vol. I., p. 219).  The measurements were written
out on 3000 pieces of cardboard, which were then very thoroughly
shuffled and drawn at random.  As each card was drawn its numbers
were written down in a book, which thus contains the measurements of
3000 criminals in a random order.  Finally, each consecutive set of
4 was taken as a sample---750 in all---and the mean, standard
deviation, and correlation of each sample determined.  The
difference between the mean of each sample and the mean of the
population was then divided by the standard deviation of the sample,
giving us the \emph{z} of Section III.''

The table is in fact page 216 and not page 219 in MacDonell(1902).
In the MacDonell table, the middle finger lengths were given in mm
and the heights in feet/inches intervals, they are both converted into
cm here.  The midpoints of intervals were used, e.g., where MacDonell
has \eqn{4' 7''9/16 -- 8''9/16}{}, we have 142.24 which is 2.54*56 =
2.54*(\eqn{4' 8''}{}).

MacDonell credited the source of data (page 178) as follows:
\emph{The data on which the memoir is based were obtained, through the
kindness of Dr Garson, from the Central Metric Office, New Scotland Yard...}
He pointed out on page 179 that : \emph{The forms were drawn at random
from the mass on the office shelves; we are therefore dealing with a
random sampling.}
\end{Details}
%
\begin{Source}\relax
\url{http://pbil.univ-lyon1.fr/R/donnees/criminals1902.txt}
thanks to Jean R. Lobry and Anne-Béatrice Dufour.
\end{Source}
%
\begin{References}\relax
Garson, J.G. (1900)
The metric system of identification of criminals, as used in in Great
Britain and Ireland.
\emph{The Journal of the Anthropological Institute of Great Britain
and Ireland} \bold{30}, 161--198.

MacDonell, W.R. (1902)
On criminal anthropometry and the identification of criminals.
\emph{Biometrika} \bold{1}, 2, 177--227.

Student (1908) The probable error of a mean.
\emph{Biometrika} \bold{6}, 1--25.
\end{References}
%
\begin{Examples}
\begin{ExampleCode}
require(stats)
dim(crimtab)
utils::str(crimtab)
## for nicer printing:
local({cT <- crimtab
       colnames(cT) <- substring(colnames(cT), 2,3)
       print(cT, zero.print = " ")
})

## Repeat Student's experiment:

# 1) Reconstitute 3000 raw data for heights in inches and rounded to
#    nearest integer as in Student's paper:

(heIn <- round(as.numeric(colnames(crimtab)) / 2.54))
d.hei <- data.frame(height = rep(heIn, colSums(crimtab)))

# 2) shuffle the data:

set.seed(1)
d.hei <- d.hei[sample(1:3000), , drop = FALSE]

# 3) Make 750 samples each of size 4:

d.hei$sample <- as.factor(rep(1:750, each = 4))

# 4) Compute the means and standard deviations (n) for the 750 samples:

h.mean <- with(d.hei, tapply(height, sample, FUN = mean))
h.sd   <- with(d.hei, tapply(height, sample, FUN = sd)) * sqrt(3/4)

# 5) Compute the difference between the mean of each sample and
#    the mean of the population and then divide by the
#    standard deviation of the sample:

zobs <- (h.mean - mean(d.hei[,"height"]))/h.sd

# 6) Replace infinite values by +/- 6 as in Student's paper:

zobs[infZ <- is.infinite(zobs)] # 3 of them
zobs[infZ] <- 6 * sign(zobs[infZ])

# 7) Plot the distribution:

require(grDevices); require(graphics)
hist(x = zobs, probability = TRUE, xlab = "Student's z",
     col = grey(0.8), border = grey(0.5),
     main = "Distribution of Student's z score  for 'crimtab' data")
\end{ExampleCode}
\end{Examples}
\HeaderA{discoveries}{Yearly Numbers of Important Discoveries}{discoveries}
\keyword{datasets}{discoveries}
%
\begin{Description}\relax
The numbers of ``great'' inventions and scientific
discoveries in each year from 1860 to 1959.
\end{Description}
%
\begin{Usage}
\begin{verbatim}
discoveries
\end{verbatim}
\end{Usage}
%
\begin{Format}
A time series of 100 values.
\end{Format}
%
\begin{Source}\relax
The World Almanac and Book of Facts, 1975 Edition, pages 315--318.
\end{Source}
%
\begin{References}\relax
McNeil, D. R. (1977)
\emph{Interactive Data Analysis}.
Wiley.
\end{References}
%
\begin{Examples}
\begin{ExampleCode}
require(graphics)
plot(discoveries, ylab = "Number of important discoveries",
     las = 1)
title(main = "discoveries data set")
\end{ExampleCode}
\end{Examples}
\HeaderA{DNase}{Elisa assay of DNase}{DNase}
\keyword{datasets}{DNase}
%
\begin{Description}\relax
The \code{DNase} data frame has 176 rows and 3 columns of data
obtained during development of an ELISA assay for the recombinant
protein DNase in rat serum.
\end{Description}
%
\begin{Usage}
\begin{verbatim}
DNase
\end{verbatim}
\end{Usage}
%
\begin{Format}
This object of class \code{c("nfnGroupedData", "nfGroupedData",
    "groupedData", "data.frame")} containing the following columns:
\begin{description}

\item[Run] 
an ordered factor with levels \code{10} < \dots < \code{3}
indicating the assay run.

\item[conc] 
a numeric vector giving the known concentration of the
protein. 

\item[density] 
a numeric vector giving the measured optical density
(dimensionless) in the assay.  Duplicate optical density
measurements were obtained. 


\end{description}

\end{Format}
%
\begin{Details}\relax
This dataset was originally part of package \code{nlme}, and that has
methods (including for \code{[}, \code{as.data.frame}, \code{plot} and
\code{print}) for its grouped-data classes. 
\end{Details}
%
\begin{Source}\relax
Davidian, M. and Giltinan, D. M. (1995) \emph{Nonlinear Models for
Repeated Measurement Data}, Chapman \& Hall (section 5.2.4, p. 134)

Pinheiro, J. C. and Bates, D. M. (2000) \emph{Mixed-effects Models in
S and S-PLUS}, Springer.
\end{Source}
%
\begin{Examples}
\begin{ExampleCode}
require(stats); require(graphics)

coplot(density ~ conc | Run, data = DNase,
       show.given = FALSE, type = "b")
coplot(density ~ log(conc) | Run, data = DNase,
       show.given = FALSE, type = "b")
## fit a representative run
fm1 <- nls(density ~ SSlogis( log(conc), Asym, xmid, scal ),
    data = DNase, subset = Run == 1)
## compare with a four-parameter logistic
fm2 <- nls(density ~ SSfpl( log(conc), A, B, xmid, scal ),
    data = DNase, subset = Run == 1)
summary(fm2)
anova(fm1, fm2)
\end{ExampleCode}
\end{Examples}
\HeaderA{esoph}{Smoking, Alcohol and (O)esophageal Cancer}{esoph}
\keyword{datasets}{esoph}
%
\begin{Description}\relax
Data from a case-control study of (o)esophageal cancer in
Ille-et-Vilaine, France.
\end{Description}
%
\begin{Usage}
\begin{verbatim}
esoph
\end{verbatim}
\end{Usage}
%
\begin{Format}
A data frame with records for 88 age/alcohol/tobacco combinations.


\Tabular{rlll}{
[,1] & "agegp" & Age group & 1  25--34 years\\{}
& & & 2  35--44\\{}
& & & 3  45--54\\{}
& & & 4  55--64\\{}
& & & 5  65--74\\{}
& & & 6  75+\\{}
[,2] & "alcgp" & Alcohol consumption & 1   0--39 gm/day\\{}
& & & 2  40--79\\{}
& & & 3  80--119\\{}
& & & 4  120+\\{}
[,3] & "tobgp" & Tobacco consumption & 1   0-- 9 gm/day\\{}
& & & 2  10--19\\{}
& & & 3  20--29\\{}
& & & 4  30+\\{}
[,4] & "ncases" & Number of cases & \\{}
[,5] & "ncontrols" & Number of controls &
}
\end{Format}
%
\begin{Author}\relax
Thomas Lumley
\end{Author}
%
\begin{Source}\relax
Breslow, N. E. and Day, N. E. (1980)
\emph{Statistical Methods in Cancer Research. 1: The Analysis of
Case-Control Studies.}  IARC Lyon / Oxford University Press.
\end{Source}
%
\begin{Examples}
\begin{ExampleCode}
require(stats)
require(graphics) # for mosaicplot
summary(esoph)
## effects of alcohol, tobacco and interaction, age-adjusted
model1 <- glm(cbind(ncases, ncontrols) ~ agegp + tobgp * alcgp,
              data = esoph, family = binomial())
anova(model1)
## Try a linear effect of alcohol and tobacco
model2 <- glm(cbind(ncases, ncontrols) ~ agegp + unclass(tobgp)
                                         + unclass(alcgp),
              data = esoph, family = binomial())
summary(model2)
## Re-arrange data for a mosaic plot
ttt <- table(esoph$agegp, esoph$alcgp, esoph$tobgp)
o <- with(esoph, order(tobgp, alcgp, agegp))
ttt[ttt == 1] <- esoph$ncases[o]
tt1 <- table(esoph$agegp, esoph$alcgp, esoph$tobgp)
tt1[tt1 == 1] <- esoph$ncontrols[o]
tt <- array(c(ttt, tt1), c(dim(ttt),2),
            c(dimnames(ttt), list(c("Cancer", "control"))))
mosaicplot(tt, main = "esoph data set", color = TRUE)
\end{ExampleCode}
\end{Examples}
\HeaderA{euro}{Conversion Rates of Euro Currencies}{euro}
\methaliasA{euro.cross}{euro}{euro.cross}
\keyword{datasets}{euro}
%
\begin{Description}\relax
Conversion rates between the various Euro currencies.
\end{Description}
%
\begin{Usage}
\begin{verbatim}
euro
euro.cross
\end{verbatim}
\end{Usage}
%
\begin{Format}
\code{euro} is a named vector of length 11, \code{euro.cross} a 
matrix of size 11 by 11, with dimnames.
\end{Format}
%
\begin{Details}\relax
The data set \code{euro} contains the value of 1 Euro in all
currencies participating in the European monetary union (Austrian
Schilling ATS, Belgian Franc BEF, German Mark DEM, Spanish Peseta ESP,
Finnish Markka FIM, French Franc FRF, Irish Punt IEP, Italian Lira
ITL, Luxembourg Franc LUF, Dutch Guilder NLG and Portuguese Escudo
PTE).  These conversion rates were fixed by the European Union on
December 31, 1998.  To convert old prices to Euro prices, divide by
the respective rate and round to 2 digits.

The data set \code{euro.cross} contains conversion rates between the
various Euro currencies, i.e., the result of
\code{outer(1 / euro, euro)}.
\end{Details}
%
\begin{Examples}
\begin{ExampleCode}
cbind(euro)

## These relations hold:
euro == signif(euro,6) # [6 digit precision in Euro's definition]
all(euro.cross == outer(1/euro, euro))

## Convert 20 Euro to Belgian Franc
20 * euro["BEF"]
## Convert 20 Austrian Schilling to Euro
20 / euro["ATS"]
## Convert 20 Spanish Pesetas to Italian Lira
20 * euro.cross["ESP", "ITL"]

require(graphics)
dotchart(euro,
         main = "euro data: 1 Euro in currency unit")
dotchart(1/euro,
         main = "euro data: 1 currency unit in Euros")
dotchart(log(euro, 10),
         main = "euro data: log10(1 Euro in currency unit)")
\end{ExampleCode}
\end{Examples}
\HeaderA{eurodist}{Distances Between European Cities}{eurodist}
\keyword{datasets}{eurodist}
%
\begin{Description}\relax
The data give the road distances (in km) between 21 cities in Europe.
The data are taken from a table in \emph{The Cambridge Encyclopaedia}.
\end{Description}
%
\begin{Usage}
\begin{verbatim}
eurodist
\end{verbatim}
\end{Usage}
%
\begin{Format}
A \code{dist} object based on 21 objects.
(You must have the \pkg{stats} package loaded to have the methods for this
kind of object available).
\end{Format}
%
\begin{Source}\relax
Crystal, D. Ed. (1990)
\emph{The Cambridge Encyclopaedia}.
Cambridge: Cambridge University Press,
\end{Source}
\HeaderA{EuStockMarkets}{Daily Closing Prices of Major European Stock Indices, 1991--1998}{EuStockMarkets}
\keyword{datasets}{EuStockMarkets}
%
\begin{Description}\relax
Contains the daily closing prices of major European stock indices:
Germany DAX (Ibis), Switzerland SMI, France CAC, and UK FTSE.  The
data are sampled in business time, i.e., weekends and holidays are
omitted.
\end{Description}
%
\begin{Usage}
\begin{verbatim}
EuStockMarkets
\end{verbatim}
\end{Usage}
%
\begin{Format}
A multivariate time series with 1860 observations on 4 variables.
The object is of class \code{"mts"}.
\end{Format}
%
\begin{Source}\relax
The data were kindly provided by Erste Bank AG, Vienna, Austria.
\end{Source}
\HeaderA{faithful}{Old Faithful Geyser Data}{faithful}
\keyword{datasets}{faithful}
%
\begin{Description}\relax
Waiting time between eruptions and the duration of the eruption for
the Old Faithful geyser in Yellowstone National Park, Wyoming, USA.
\end{Description}
%
\begin{Usage}
\begin{verbatim}
faithful
\end{verbatim}
\end{Usage}
%
\begin{Format}
A data frame with 272 observations on 2 variables.

\Tabular{rlll}{
[,1]  & eruptions  & numeric  & Eruption time in mins \\{}
[,2]  & waiting    & numeric  & Waiting time to next
eruption (in mins)\\{}
}
\end{Format}
%
\begin{Details}\relax
A closer look at \code{faithful\$eruptions} reveals that these are
heavily rounded times originally in seconds, where multiples of 5 are
more frequent than expected under non-human measurement.  For a
better version of the eruption times, see the example below.

There are many versions of this dataset around: Azzalini and Bowman
(1990) use a more complete version.
\end{Details}
%
\begin{Source}\relax
W. Härdle.
\end{Source}
%
\begin{References}\relax
Härdle, W. (1991)
\emph{Smoothing Techniques with Implementation in S}.
New York: Springer.

Azzalini, A. and Bowman, A. W. (1990).
A look at some data on the Old Faithful geyser.
\emph{Applied Statistics} \bold{39}, 357--365.
\end{References}
%
\begin{SeeAlso}\relax
\code{geyser} in package \Rhref{http://CRAN.R-project.org/package=MASS}{\pkg{MASS}} for the Azzalini--Bowman version.
\end{SeeAlso}
%
\begin{Examples}
\begin{ExampleCode}
require(stats); require(graphics)
f.tit <-  "faithful data: Eruptions of Old Faithful"

ne60 <- round(e60 <- 60 * faithful$eruptions)
all.equal(e60, ne60)             # relative diff. ~ 1/10000
table(zapsmall(abs(e60 - ne60))) # 0, 0.02 or 0.04
faithful$better.eruptions <- ne60 / 60
te <- table(ne60)
te[te >= 4]                      # (too) many multiples of 5 !
plot(names(te), te, type="h", main = f.tit, xlab = "Eruption time (sec)")

plot(faithful[, -3], main = f.tit,
     xlab = "Eruption time (min)",
     ylab = "Waiting time to next eruption (min)")
lines(lowess(faithful$eruptions, faithful$waiting, f = 2/3, iter = 3),
      col = "red")
\end{ExampleCode}
\end{Examples}
\HeaderA{Formaldehyde}{Determination of Formaldehyde}{Formaldehyde}
\keyword{datasets}{Formaldehyde}
%
\begin{Description}\relax
These data are from a chemical experiment to prepare a standard curve
for the determination of formaldehyde by the addition of chromatropic
acid and concentrated sulphuric acid and the reading of the resulting
purple color on a spectrophotometer.
\end{Description}
%
\begin{Usage}
\begin{verbatim}
Formaldehyde
\end{verbatim}
\end{Usage}
%
\begin{Format}
A data frame with 6 observations on 2 variables.

\Tabular{rlll}{
[,1] & carb& numeric & Carbohydrate (ml) \\{}
[,2] & optden & numeric & Optical Density
}
\end{Format}
%
\begin{Source}\relax
Bennett, N. A. and N. L. Franklin (1954)
\emph{Statistical Analysis in Chemistry and the Chemical Industry}.
New York: Wiley.
\end{Source}
%
\begin{References}\relax
McNeil, D. R. (1977) \emph{Interactive Data Analysis.}
New York: Wiley.
\end{References}
%
\begin{Examples}
\begin{ExampleCode}
require(stats); require(graphics)
plot(optden ~ carb, data = Formaldehyde,
     xlab = "Carbohydrate (ml)", ylab = "Optical Density",
     main = "Formaldehyde data", col = 4, las = 1)
abline(fm1 <- lm(optden ~ carb, data = Formaldehyde))
summary(fm1)
opar <- par(mfrow = c(2,2), oma = c(0, 0, 1.1, 0))
plot(fm1)
par(opar)
\end{ExampleCode}
\end{Examples}
\HeaderA{freeny}{Freeny's Revenue Data}{freeny}
\methaliasA{freeny.x}{freeny}{freeny.x}
\methaliasA{freeny.y}{freeny}{freeny.y}
\keyword{datasets}{freeny}
%
\begin{Description}\relax
Freeny's data on quarterly revenue and explanatory variables.
\end{Description}
%
\begin{Usage}
\begin{verbatim}
freeny
freeny.x
freeny.y
\end{verbatim}
\end{Usage}
%
\begin{Format}
There are three `freeny' data sets.

\code{freeny.y} is a time series with 39 observations on quarterly
revenue from (1962,2Q) to (1971,4Q).

\code{freeny.x} is a matrix of explanatory variables.  The columns
are \code{freeny.y} lagged 1 quarter, price index, income level, and
market potential.

Finally, \code{freeny} is a data frame with variables \code{y},
\code{lag.quarterly.revenue}, \code{price.index}, \code{income.level},
and \code{market.potential} obtained from the above two data objects.
\end{Format}
%
\begin{Source}\relax
A. E. Freeny (1977)
\emph{A Portable Linear Regression Package with Test Programs}.
Bell Laboratories memorandum.
\end{Source}
%
\begin{References}\relax
Becker, R. A., Chambers, J. M. and Wilks, A. R. (1988)
\emph{The New S Language}.
Wadsworth \& Brooks/Cole.
\end{References}
%
\begin{Examples}
\begin{ExampleCode}
require(stats); require(graphics)
summary(freeny)
pairs(freeny, main = "freeny data")
# gives warning: freeny$y has class "ts"

summary(fm1 <- lm(y ~ ., data = freeny))
opar <- par(mfrow = c(2, 2), oma = c(0, 0, 1.1, 0),
            mar = c(4.1, 4.1, 2.1, 1.1))
plot(fm1)
par(opar)
\end{ExampleCode}
\end{Examples}
\HeaderA{HairEyeColor}{Hair and Eye Color of Statistics Students}{HairEyeColor}
\keyword{datasets}{HairEyeColor}
%
\begin{Description}\relax
Distribution of hair and eye color and sex in 592 statistics students.
\end{Description}
%
\begin{Usage}
\begin{verbatim}
HairEyeColor
\end{verbatim}
\end{Usage}
%
\begin{Format}
A 3-dimensional array resulting from cross-tabulating 592 observations
on 3 variables.  The variables and their levels are as follows:


\Tabular{rll}{
No & Name & Levels \\{}
1 & Hair & Black, Brown, Red, Blond \\{}
2 & Eye & Brown, Blue, Hazel, Green \\{}
3 & Sex & Male, Female
}
\end{Format}
%
\begin{Details}\relax
The Hair \eqn{\times}{} Eye table comes rom a survey of students at
the University of Delaware reported by Snee (1974).  The split by
\code{Sex} was added by Friendly (1992a) for didactic purposes.

This data set is useful for illustrating various techniques for the
analysis of contingency tables, such as the standard chi-squared test
or, more generally, log-linear modelling, and graphical methods such
as mosaic plots, sieve diagrams or association plots.
\end{Details}
%
\begin{Source}\relax
\url{http://euclid.psych.yorku.ca/ftp/sas/vcd/catdata/haireye.sas}

Snee (1974) gives the two-way table aggregated over \code{Sex}.  The
Sex split of the `Brown hair, Brown eye' cell was changed in
\R{} 2.6.0 to agree with that used by Friendly (2000).
\end{Source}
%
\begin{References}\relax
Snee, R. D. (1974)
Graphical display of two-way contingency tables.
\emph{The American Statistician}, \bold{28}, 9--12.

Friendly, M. (1992a)
Graphical methods for categorical data.
\emph{SAS User Group International Conference Proceedings}, \bold{17},
190--200.
\url{http://www.math.yorku.ca/SCS/sugi/sugi17-paper.html}

Friendly, M. (1992b)
Mosaic displays for loglinear models.
\emph{Proceedings of the Statistical Graphics Section},
American Statistical Association, pp. 61--68.
\url{http://www.math.yorku.ca/SCS/Papers/asa92.html}

Friendly, M. (2000)
\emph{Visualizing Categorical Data.}
SAS Institute, ISBN 1-58025-660-0.
\end{References}
%
\begin{SeeAlso}\relax
\code{\LinkA{chisq.test}{chisq.test}},
\code{\LinkA{loglin}{loglin}},
\code{\LinkA{mosaicplot}{mosaicplot}}
\end{SeeAlso}
%
\begin{Examples}
\begin{ExampleCode}
require(graphics)
## Full mosaic
mosaicplot(HairEyeColor)
## Aggregate over sex (as in Snee's original data)
x <- apply(HairEyeColor, c(1, 2), sum)
x
mosaicplot(x, main = "Relation between hair and eye color")
\end{ExampleCode}
\end{Examples}
\HeaderA{Harman23.cor}{Harman Example 2.3}{Harman23.cor}
\keyword{datasets}{Harman23.cor}
%
\begin{Description}\relax
A correlation matrix of eight physical measurements on 305 girls between
ages seven and seventeen.
\end{Description}
%
\begin{Usage}
\begin{verbatim}
Harman23.cor
\end{verbatim}
\end{Usage}
%
\begin{Source}\relax
Harman, H. H. (1976)
\emph{Modern Factor Analysis}, Third Edition Revised,
University of Chicago Press, Table 2.3.
\end{Source}
%
\begin{Examples}
\begin{ExampleCode}
require(stats)
(Harman23.FA <- factanal(factors = 1, covmat = Harman23.cor))
for(factors in 2:4) print(update(Harman23.FA, factors = factors))
\end{ExampleCode}
\end{Examples}
\HeaderA{Harman74.cor}{Harman Example 7.4}{Harman74.cor}
\keyword{datasets}{Harman74.cor}
%
\begin{Description}\relax
A correlation matrix of 24 psychological tests given to 145 seventh and
eight-grade children in a Chicago suburb by Holzinger and Swineford.
\end{Description}
%
\begin{Usage}
\begin{verbatim}
Harman74.cor
\end{verbatim}
\end{Usage}
%
\begin{Source}\relax
Harman, H. H. (1976)
\emph{Modern Factor Analysis}, Third Edition Revised,
University of Chicago Press, Table 7.4.
\end{Source}
%
\begin{Examples}
\begin{ExampleCode}
require(stats)
(Harman74.FA <- factanal(factors = 1, covmat = Harman74.cor))
for(factors in 2:5) print(update(Harman74.FA, factors = factors))
Harman74.FA <- factanal(factors = 5, covmat = Harman74.cor,
                        rotation="promax")
print(Harman74.FA$loadings, sort = TRUE)
\end{ExampleCode}
\end{Examples}
\HeaderA{Indometh}{Pharmacokinetics of Indomethacin}{Indometh}
\keyword{datasets}{Indometh}
%
\begin{Description}\relax
The \code{Indometh} data frame has 66 rows and 3 columns of data on
the pharmacokinetics of indometacin (or, older spelling,
`indomethacin').
\end{Description}
%
\begin{Usage}
\begin{verbatim}
Indometh
\end{verbatim}
\end{Usage}
%
\begin{Format}
This object of class \code{c("nfnGroupedData", "nfGroupedData",
    "groupedData", "data.frame")} containing the following columns:
\begin{description}

\item[Subject] 
an ordered factor with containing the subject codes.  The
ordering is according to increasing maximum response.

\item[time] 
a numeric vector of times at which blood samples were drawn (hr).

\item[conc] 
a numeric vector of plasma concentrations of indometacin (mcg/ml).


\end{description}

\end{Format}
%
\begin{Details}\relax
Each of the six subjects were given an intravenous injection of
indometacin.

This dataset was originally part of package \code{nlme}, and that has
methods (including for \code{[}, \code{as.data.frame}, \code{plot} and
\code{print}) for its grouped-data classes. 
\end{Details}
%
\begin{Source}\relax
Kwan, Breault, Umbenhauer, McMahon and Duggan (1976)
Kinetics of Indomethacin absorption, elimination, and
enterohepatic circulation in man.
\emph{Journal of Pharmacokinetics and Biopharmaceutics} \bold{4},
255--280.

Davidian, M. and Giltinan, D. M. (1995)
\emph{Nonlinear Models for Repeated Measurement Data},
Chapman \& Hall (section 5.2.4, p. 129)

Pinheiro, J. C. and Bates, D. M. (2000) \emph{Mixed-effects Models in
S and S-PLUS}, Springer.
\end{Source}
%
\begin{SeeAlso}\relax
\code{\LinkA{SSbiexp}{SSbiexp}} for models fitted to this dataset.
\end{SeeAlso}
\HeaderA{infert}{Infertility after Spontaneous and Induced Abortion}{infert}
\keyword{datasets}{infert}
%
\begin{Description}\relax
This is a matched case-control study dating from before the
availability of conditional logistic regression.
\end{Description}
%
\begin{Usage}
\begin{verbatim}
infert
\end{verbatim}
\end{Usage}
%
\begin{Format}

\Tabular{rll}{
1.  & Education  & 0 = 0-5  years \\{}
&            & 1 = 6-11 years \\{}
&            & 2 = 12+  years  \\{}
2.  & age        & age in years of case \\{}
3.  & parity     & count \\{}
4.  & number of prior & 0 = 0 \\{}
& induced abortions & 1 = 1 \\{}
&            & 2 = 2 or more \\{}
5.  & case status& 1 = case \\{}
&            & 0 = control \\{}
6.  & number of prior & 0 = 0 \\{}
& spontaneous abortions & 1 = 1 \\{}
&            & 2 = 2 or more \\{}
7.  & matched set number & 1-83 \\{}
8.  & stratum number & 1-63}
\end{Format}
%
\begin{Note}\relax
One case with two prior spontaneous abortions and two prior induced
abortions is omitted.
\end{Note}
%
\begin{Source}\relax
Trichopoulos et al. (1976)
\emph{Br. J. of Obst. and Gynaec.} \bold{83}, 645--650.
\end{Source}
%
\begin{Examples}
\begin{ExampleCode}
require(stats)
model1 <- glm(case ~ spontaneous+induced, data=infert,family=binomial())
summary(model1)
## adjusted for other potential confounders:
summary(model2 <- glm(case ~ age+parity+education+spontaneous+induced,
                data=infert,family=binomial()))
## Really should be analysed by conditional logistic regression
## which is in the survival package
if(require(survival)){
  model3 <- clogit(case~spontaneous+induced+strata(stratum),data=infert)
  print(summary(model3))
  detach()# survival (conflicts)
}
\end{ExampleCode}
\end{Examples}
\HeaderA{InsectSprays}{Effectiveness of Insect Sprays}{InsectSprays}
\keyword{datasets}{InsectSprays}
%
\begin{Description}\relax
The counts of insects in agricultural experimental units treated with
different insecticides.
\end{Description}
%
\begin{Usage}
\begin{verbatim}
InsectSprays
\end{verbatim}
\end{Usage}
%
\begin{Format}
A data frame with 72 observations on 2 variables.

\Tabular{rlll}{
[,1]  & count  & numeric  & Insect count\\{}
[,2]  & spray  & factor   & The type of spray
}
\end{Format}
%
\begin{Source}\relax
Beall, G., (1942)
The Transformation of data from entomological field experiments,
\emph{Biometrika}, \bold{29}, 243--262.
\end{Source}
%
\begin{References}\relax
McNeil, D. (1977) \emph{Interactive Data Analysis}.
New York: Wiley.
\end{References}
%
\begin{Examples}
\begin{ExampleCode}
require(stats); require(graphics)
boxplot(count ~ spray, data = InsectSprays,
        xlab = "Type of spray", ylab = "Insect count",
        main = "InsectSprays data", varwidth = TRUE, col = "lightgray")
fm1 <- aov(count ~ spray, data = InsectSprays)
summary(fm1)
opar <- par(mfrow = c(2,2), oma = c(0, 0, 1.1, 0))
plot(fm1)
fm2 <- aov(sqrt(count) ~ spray, data = InsectSprays)
summary(fm2)
plot(fm2)
par(opar)
\end{ExampleCode}
\end{Examples}
\HeaderA{iris}{Edgar Anderson's Iris Data}{iris}
\aliasA{iris3}{iris}{iris3}
\keyword{datasets}{iris}
%
\begin{Description}\relax
This famous (Fisher's or Anderson's) iris data set gives the
measurements in centimeters of the variables sepal length and width
and petal length and width, respectively, for 50 flowers from each
of 3 species of iris.  The species are \emph{Iris setosa},
\emph{versicolor}, and \emph{virginica}.
\end{Description}
%
\begin{Usage}
\begin{verbatim}
iris
iris3
\end{verbatim}
\end{Usage}
%
\begin{Format}
\code{iris} is a data frame with 150 cases (rows) and 5 variables
(columns) named \code{Sepal.Length}, \code{Sepal.Width},
\code{Petal.Length}, \code{Petal.Width}, and \code{Species}.

\code{iris3} gives the same data arranged as a 3-dimensional array
of size 50 by 4 by 3, as represented by S-PLUS.  The first dimension
gives the case number within the species subsample, the second the
measurements with names \code{Sepal L.}, \code{Sepal W.},
\code{Petal L.}, and \code{Petal W.}, and the third the species.
\end{Format}
%
\begin{Source}\relax
Fisher, R. A. (1936)
The use of multiple measurements in taxonomic problems.
\emph{Annals of Eugenics},
\bold{7}, Part II, 179--188.

The data were collected by
Anderson, Edgar (1935).
The irises of the Gaspe Peninsula,
\emph{Bulletin of the American Iris Society},
\bold{59}, 2--5.
\end{Source}
%
\begin{References}\relax
Becker, R. A., Chambers, J. M. and Wilks, A. R. (1988)
\emph{The New S Language}.
Wadsworth \& Brooks/Cole. (has \code{iris3} as \code{iris}.)
\end{References}
%
\begin{SeeAlso}\relax
\code{\LinkA{matplot}{matplot}} some examples of which use
\code{iris}.
\end{SeeAlso}
%
\begin{Examples}
\begin{ExampleCode}
dni3 <- dimnames(iris3)
ii <- data.frame(matrix(aperm(iris3, c(1,3,2)), ncol=4,
                        dimnames = list(NULL, sub(" L.",".Length",
                                        sub(" W.",".Width", dni3[[2]])))),
    Species = gl(3, 50, labels=sub("S", "s", sub("V", "v", dni3[[3]]))))
all.equal(ii, iris) # TRUE
\end{ExampleCode}
\end{Examples}
\HeaderA{islands}{Areas of the World's Major Landmasses}{islands}
\keyword{datasets}{islands}
%
\begin{Description}\relax
The areas in thousands of square miles of the landmasses which exceed
10,000 square miles.
\end{Description}
%
\begin{Usage}
\begin{verbatim}
islands
\end{verbatim}
\end{Usage}
%
\begin{Format}
A named vector of length 48.
\end{Format}
%
\begin{Source}\relax
The World Almanac and Book of Facts, 1975, page 406.
\end{Source}
%
\begin{References}\relax
McNeil, D. R. (1977)
\emph{Interactive Data Analysis}.
Wiley.
\end{References}
%
\begin{Examples}
\begin{ExampleCode}
require(graphics)
dotchart(log(islands, 10),
   main = "islands data: log10(area) (log10(sq. miles))")
dotchart(log(islands[order(islands)], 10),
   main = "islands data: log10(area) (log10(sq. miles))")
\end{ExampleCode}
\end{Examples}
\HeaderA{JohnsonJohnson}{Quarterly Earnings per Johnson \& Johnson Share}{JohnsonJohnson}
\keyword{datasets}{JohnsonJohnson}
%
\begin{Description}\relax
Quarterly earnings (dollars) per Johnson \& Johnson share 1960--80.
\end{Description}
%
\begin{Usage}
\begin{verbatim}
JohnsonJohnson
\end{verbatim}
\end{Usage}
%
\begin{Format}
A quarterly time series
\end{Format}
%
\begin{Source}\relax
Shumway, R. H. and Stoffer, D. S. (2000)
\emph{Time Series Analysis and its Applications}.
Second Edition.  Springer.  Example 1.1.
\end{Source}
%
\begin{Examples}
\begin{ExampleCode}

require(stats); require(graphics)
JJ <- log10(JohnsonJohnson)
plot(JJ)
## This example gives a possible-non-convergence warning on some
## platforms, but does seem to converge on x86 Linux and Windows.
(fit <- StructTS(JJ, type="BSM"))
tsdiag(fit)
sm <- tsSmooth(fit)
plot(cbind(JJ, sm[, 1], sm[, 3]-0.5), plot.type = "single",
     col = c("black", "green", "blue"))
abline(h = -0.5, col = "grey60")

monthplot(fit)
\end{ExampleCode}
\end{Examples}
\HeaderA{LakeHuron}{Level of Lake Huron 1875--1972}{LakeHuron}
\keyword{datasets}{LakeHuron}
%
\begin{Description}\relax
Annual measurements of the level, in feet, of Lake Huron 1875--1972.
\end{Description}
%
\begin{Usage}
\begin{verbatim}
LakeHuron
\end{verbatim}
\end{Usage}
%
\begin{Format}
A time series of length 98.
\end{Format}
%
\begin{Source}\relax
Brockwell, P. J. and Davis, R. A. (1991).
\emph{Time Series and Forecasting Methods}.
Second edition. Springer, New York. Series A, page 555.

Brockwell, P. J. and Davis, R. A. (1996).
\emph{Introduction to Time Series and Forecasting}.
Springer, New York.
Sections 5.1 and 7.6.
\end{Source}
\HeaderA{lh}{Luteinizing Hormone in Blood Samples}{lh}
\keyword{datasets}{lh}
%
\begin{Description}\relax
A regular time series giving the luteinizing hormone in blood
samples at 10 mins intervals from a human female, 48 samples.
\end{Description}
%
\begin{Usage}
\begin{verbatim}
lh
\end{verbatim}
\end{Usage}
%
\begin{Source}\relax
P.J. Diggle (1990)
\emph{Time Series: A Biostatistical Introduction.}
Oxford, table A.1, series 3
\end{Source}
\HeaderA{LifeCycleSavings}{Intercountry Life-Cycle Savings Data}{LifeCycleSavings}
\keyword{datasets}{LifeCycleSavings}
%
\begin{Description}\relax
Data on the savings ratio 1960--1970.
\end{Description}
%
\begin{Usage}
\begin{verbatim}
LifeCycleSavings
\end{verbatim}
\end{Usage}
%
\begin{Format}
A data frame with 50 observations on 5 variables.

\Tabular{rlll}{
[,1]  & sr    & numeric  & aggregate personal savings \\{}
[,2]  & pop15 & numeric  & \% of population under 15 \\{}
[,3]  & pop75 & numeric  & \% of population over 75 \\{}
[,4]  & dpi   & numeric  & real per-capita disposable
income \\{}
[,5]  & ddpi  & numeric  & \% growth rate of dpi
}
\end{Format}
%
\begin{Details}\relax
Under the life-cycle savings hypothesis as developed by Franco
Modigliani, the savings ratio (aggregate personal saving divided by
disposable income) is explained by per-capita disposable income, the
percentage rate of change in per-capita disposable income, and two
demographic variables: the percentage of population less than 15
years old and the percentage of the population over 75 years old.
The data are averaged over the decade 1960--1970 to remove the
business cycle or other short-term fluctuations. 
\end{Details}
%
\begin{Source}\relax
The data were obtained from Belsley, Kuh and Welsch (1980).
They in turn obtained the data from Sterling (1977).
\end{Source}
%
\begin{References}\relax
Sterling, Arnie (1977) Unpublished BS Thesis.
Massachusetts Institute of Technology.

Belsley, D. A., Kuh. E. and Welsch, R. E. (1980)
\emph{Regression Diagnostics}.
New York: Wiley.
\end{References}
%
\begin{Examples}
\begin{ExampleCode}
require(stats); require(graphics)
pairs(LifeCycleSavings, panel = panel.smooth,
      main = "LifeCycleSavings data")
fm1 <- lm(sr ~ pop15 + pop75 + dpi + ddpi, data = LifeCycleSavings)
summary(fm1)
\end{ExampleCode}
\end{Examples}
\HeaderA{Loblolly}{Growth of Loblolly pine trees}{Loblolly}
\keyword{datasets}{Loblolly}
%
\begin{Description}\relax
The \code{Loblolly} data frame has 84 rows and 3 columns of records of
the growth of Loblolly pine trees.
\end{Description}
%
\begin{Usage}
\begin{verbatim}
Loblolly
\end{verbatim}
\end{Usage}
%
\begin{Format}
This object of class \code{c("nfnGroupedData", "nfGroupedData",
    "groupedData", "data.frame")} containing the following columns:
\begin{description}

\item[height] 
a numeric vector of tree heights (ft).

\item[age] 
a numeric vector of tree ages (yr).

\item[Seed] 
an ordered factor indicating the seed source for the tree.
The ordering is according to increasing maximum height.


\end{description}

\end{Format}
%
\begin{Details}\relax
   
This dataset was originally part of package \code{nlme}, and that has
methods (including for \code{[}, \code{as.data.frame}, \code{plot} and
\code{print}) for its grouped-data classes. 
\end{Details}
%
\begin{Source}\relax
Kung, F. H. (1986),
Fitting logistic growth curve with predetermined carrying capacity,
in \emph{Proceedings of the Statistical Computing Section,
American Statistical Association}, 340--343.

Pinheiro, J. C. and Bates, D. M. (2000)
\emph{Mixed-effects Models in S and S-PLUS}, Springer.
\end{Source}
%
\begin{Examples}
\begin{ExampleCode}
require(stats); require(graphics)
plot(height ~ age, data = Loblolly, subset = Seed == 329,
     xlab = "Tree age (yr)", las = 1,
     ylab = "Tree height (ft)",
     main = "Loblolly data and fitted curve (Seed 329 only)")
fm1 <- nls(height ~ SSasymp(age, Asym, R0, lrc),
           data = Loblolly, subset = Seed == 329)
age <- seq(0, 30, length.out = 101)
lines(age, predict(fm1, list(age = age)))
\end{ExampleCode}
\end{Examples}
\HeaderA{longley}{Longley's Economic Regression Data}{longley}
\keyword{datasets}{longley}
%
\begin{Description}\relax
A macroeconomic data set which provides a well-known example for a
highly collinear regression.
\end{Description}
%
\begin{Usage}
\begin{verbatim}
longley
\end{verbatim}
\end{Usage}
%
\begin{Format}
A data frame with 7 economical variables, observed yearly from 1947 to
1962 (\eqn{n=16}{}).
\begin{description}

\item[GNP.deflator:] GNP implicit price deflator (\eqn{1954=100}{})
\item[GNP:] Gross National Product.
\item[Unemployed:] number of unemployed.
\item[Armed.Forces:] number of people in the armed forces.
\item[Population:] `noninstitutionalized' population
\eqn{\ge}{} 14 years of age.
\item[Year:] the year (time).
\item[Employed:] number of people employed.

\end{description}


The regression \code{lm(Employed \textasciitilde{} .)} is known to be highly
collinear.
\end{Format}
%
\begin{Source}\relax
J. W. Longley (1967)
An appraisal of least-squares programs from the point of view of the
user.
\emph{Journal of the American Statistical Association}, \bold{62},
819--841.
\end{Source}
%
\begin{References}\relax
Becker, R. A., Chambers, J. M. and Wilks, A. R. (1988)
\emph{The New S Language}.
Wadsworth \& Brooks/Cole.
\end{References}
%
\begin{Examples}
\begin{ExampleCode}
require(stats); require(graphics)
## give the data set in the form it is used in S-PLUS:
longley.x <- data.matrix(longley[, 1:6])
longley.y <- longley[, "Employed"]
pairs(longley, main = "longley data")
summary(fm1 <- lm(Employed ~ ., data = longley))
opar <- par(mfrow = c(2, 2), oma = c(0, 0, 1.1, 0),
            mar = c(4.1, 4.1, 2.1, 1.1))
plot(fm1)
par(opar)
\end{ExampleCode}
\end{Examples}
\HeaderA{lynx}{Annual Canadian Lynx trappings 1821--1934}{lynx}
\keyword{datasets}{lynx}
%
\begin{Description}\relax
Annual numbers of lynx trappings for 1821--1934 in Canada. Taken from
Brockwell \& Davis (1991), this appears to be the series considered
by Campbell \& Walker (1977).
\end{Description}
%
\begin{Usage}
\begin{verbatim}
lynx
\end{verbatim}
\end{Usage}
%
\begin{Source}\relax
Brockwell, P. J. and Davis, R. A. (1991) \emph{Time
Series and Forecasting Methods.}  Second edition.
Springer. Series G (page 557).
\end{Source}
%
\begin{References}\relax
Becker, R. A., Chambers, J. M. and Wilks, A. R. (1988)
\emph{The New S Language}.
Wadsworth \& Brooks/Cole.

Campbell, M. J.and A. M.  Walker (1977).  A Survey of
statistical work on the Mackenzie River series of annual
Canadian lynx trappings for the years  1821--1934 and
a new analysis.
\emph{Journal of the Royal Statistical Society series A},
\bold{140}, 411--431.
\end{References}
\HeaderA{morley}{Michelson Speed of Light Data}{morley}
\keyword{datasets}{morley}
%
\begin{Description}\relax
A classical data of Michelson (but not this one with Morley) on
measurements done in 1879 on the speed of light.  The data consists of
five experiments, each consisting of 20 consecutive `runs'.
The response is the speed of light measurement, suitably coded
(km/sec, with \code{299000} subtracted).
\end{Description}
%
\begin{Usage}
\begin{verbatim}
morley
\end{verbatim}
\end{Usage}
%
\begin{Format}
A data frame contains the following components:
\begin{description}

\item[\code{Expt}] The experiment number, from 1 to 5.
\item[\code{Run}] The run number within each experiment.
\item[\code{Speed}] Speed-of-light measurement.

\end{description}

\end{Format}
%
\begin{Details}\relax
The data is here viewed as a randomized block experiment with
`experiment' and `run' as the factors.  `run' may
also be considered a quantitative variate to account for linear (or
polynomial) changes in the measurement over the course of a single
experiment.
\end{Details}
%
\begin{Note}\relax
This is the same dataset as \code{michelson} in package \pkg{MASS}.
\end{Note}
%
\begin{Source}\relax
A. J. Weekes (1986)
\emph{A Genstat Primer}.
London: Edward Arnold.

S. M. Stigler (1977)
Do robust estimators work with real data?
\emph{Annals of Statistics} \bold{5}, 1055--1098. (See Table 6.)

A. A. Michelson (1882)
Experimental determination of the velocity of light made at the United
States Naval Academy, Annapolis.
\emph{Astronomic Papers} \bold{1} 135--8.
U.S. Nautical Almanac Office.  (See Table 24.)
\end{Source}
%
\begin{Examples}
\begin{ExampleCode}
require(stats); require(graphics)
morley$Expt <- factor(morley$Expt)
morley$Run <- factor(morley$Run)

xtabs(~ Expt + Run, data = morley)# 5 x 20 balanced (two-way)
plot(Speed ~ Expt, data = morley,
     main = "Speed of Light Data", xlab = "Experiment No.")
fm <- aov(Speed ~ Run + Expt, data = morley)
summary(fm)
fm0 <- update(fm, . ~ . - Run)
anova(fm0, fm)
\end{ExampleCode}
\end{Examples}
\HeaderA{mtcars}{Motor Trend Car Road Tests}{mtcars}
\keyword{datasets}{mtcars}
%
\begin{Description}\relax
The data was extracted from the 1974 \emph{Motor Trend} US magazine,
and comprises fuel consumption and 10 aspects of
automobile design and performance for 32 automobiles (1973--74
models).
\end{Description}
%
\begin{Usage}
\begin{verbatim}
mtcars
\end{verbatim}
\end{Usage}
%
\begin{Format}
A data frame with 32 observations on 11 variables.

\Tabular{rll}{
[, 1] & mpg  & Miles/(US) gallon \\{}
[, 2] & cyl  & Number of cylinders \\{}
[, 3] & disp & Displacement (cu.in.) \\{}
[, 4] & hp   & Gross horsepower \\{}
[, 5] & drat & Rear axle ratio \\{}
[, 6] & wt   & Weight (lb/1000) \\{}
[, 7] & qsec & 1/4 mile time \\{}
[, 8] & vs   & V/S \\{}
[, 9] & am   & Transmission (0 = automatic, 1 = manual) \\{}
[,10] & gear & Number of forward gears \\{}
[,11] & carb & Number of carburetors
}
\end{Format}
%
\begin{Source}\relax
Henderson and Velleman (1981),
Building multiple regression models interactively.
\emph{Biometrics}, \bold{37}, 391--411.
\end{Source}
%
\begin{Examples}
\begin{ExampleCode}
require(graphics)
pairs(mtcars, main = "mtcars data")
coplot(mpg ~ disp | as.factor(cyl), data = mtcars,
       panel = panel.smooth, rows = 1)
\end{ExampleCode}
\end{Examples}
\HeaderA{nhtemp}{Average Yearly Temperatures in New Haven}{nhtemp}
\keyword{datasets}{nhtemp}
%
\begin{Description}\relax
The mean annual temperature in degrees Fahrenheit in New Haven,
Connecticut, from 1912 to 1971.
\end{Description}
%
\begin{Usage}
\begin{verbatim}
nhtemp
\end{verbatim}
\end{Usage}
%
\begin{Format}
A time series of 60 observations.
\end{Format}
%
\begin{Source}\relax
Vaux, J. E. and Brinker, N. B. (1972)
\emph{Cycles}, \bold{1972}, 117--121.
\end{Source}
%
\begin{References}\relax
McNeil, D. R. (1977)
\emph{Interactive Data Analysis}.
New York: Wiley.
\end{References}
%
\begin{Examples}
\begin{ExampleCode}
require(stats); require(graphics)
plot(nhtemp, main = "nhtemp data",
  ylab = "Mean annual temperature in New Haven, CT (deg. F)")
\end{ExampleCode}
\end{Examples}
\HeaderA{Nile}{Flow of the River Nile}{Nile}
\keyword{datasets}{Nile}
%
\begin{Description}\relax
Measurements of the annual flow of the river Nile at Ashwan 1871--1970.
\end{Description}
%
\begin{Usage}
\begin{verbatim}
Nile
\end{verbatim}
\end{Usage}
%
\begin{Format}
A time series of length 100.
\end{Format}
%
\begin{Source}\relax
Durbin, J. and Koopman, S. J. (2001) \emph{Time Series Analysis by
State Space Methods.}  Oxford University Press.
\url{http://www.ssfpack.com/DKbook.html}
\end{Source}
%
\begin{References}\relax
Balke, N. S. (1993) Detecting level shifts in time series.
\emph{Journal of Business and Economic Statistics} \bold{11}, 81--92.

Cobb, G. W. (1978) The problem of the Nile: conditional solution to a
change-point problem.  \emph{Biometrika} \bold{65}, 243--51.
\end{References}
%
\begin{Examples}
\begin{ExampleCode}
require(stats); require(graphics)
par(mfrow = c(2,2))
plot(Nile)
acf(Nile)
pacf(Nile)
ar(Nile) # selects order 2
cpgram(ar(Nile)$resid)
par(mfrow = c(1,1))
arima(Nile, c(2, 0, 0))

## Now consider missing values, following Durbin & Koopman
NileNA <- Nile
NileNA[c(21:40, 61:80)] <- NA
arima(NileNA, c(2, 0, 0))
plot(NileNA)
pred <-
   predict(arima(window(NileNA, 1871, 1890), c(2,0,0)), n.ahead = 20)
lines(pred$pred, lty = 3, col = "red")
lines(pred$pred + 2*pred$se, lty=2, col="blue")
lines(pred$pred - 2*pred$se, lty=2, col="blue")
pred <-
   predict(arima(window(NileNA, 1871, 1930), c(2,0,0)), n.ahead = 20)
lines(pred$pred, lty = 3, col = "red")
lines(pred$pred + 2*pred$se, lty=2, col="blue")
lines(pred$pred - 2*pred$se, lty=2, col="blue")

## Structural time series models
par(mfrow = c(3, 1))
plot(Nile)
## local level model
(fit <- StructTS(Nile, type = "level"))
lines(fitted(fit), lty = 2)              # contemporaneous smoothing
lines(tsSmooth(fit), lty = 2, col = 4)   # fixed-interval smoothing
plot(residuals(fit)); abline(h = 0, lty = 3)
## local trend model
(fit2 <- StructTS(Nile, type = "trend")) ## constant trend fitted
pred <- predict(fit, n.ahead = 30)
## with 50% confidence interval
ts.plot(Nile, pred$pred,
        pred$pred + 0.67*pred$se, pred$pred -0.67*pred$se)

## Now consider missing values
plot(NileNA)
(fit3 <- StructTS(NileNA, type = "level"))
lines(fitted(fit3), lty = 2)
lines(tsSmooth(fit3), lty = 3)
plot(residuals(fit3)); abline(h = 0, lty = 3)
\end{ExampleCode}
\end{Examples}
\HeaderA{nottem}{Average Monthly Temperatures at Nottingham, 1920--1939}{nottem}
\keyword{datasets}{nottem}
%
\begin{Description}\relax
A time series object containing average air temperatures at
Nottingham Castle in degrees Fahrenheit for 20 years.
\end{Description}
%
\begin{Usage}
\begin{verbatim}
nottem
\end{verbatim}
\end{Usage}
%
\begin{Source}\relax
Anderson, O. D. (1976)
\emph{Time Series Analysis and Forecasting: The Box-Jenkins approach.}
Butterworths. Series R.
\end{Source}
%
\begin{Examples}
\begin{ExampleCode}
## Not run: require(stats); require(graphics)
nott <- window(nottem, end=c(1936,12))
fit <- arima(nott,order=c(1,0,0), list(order=c(2,1,0), period=12))
nott.fore <- predict(fit, n.ahead=36)
ts.plot(nott, nott.fore$pred, nott.fore$pred+2*nott.fore$se,
        nott.fore$pred-2*nott.fore$se, gpars=list(col=c(1,1,4,4)))

## End(Not run)
\end{ExampleCode}
\end{Examples}
\HeaderA{occupationalStatus}{Occupational Status of Fathers and their Sons}{occupationalStatus}
\keyword{datasets}{occupationalStatus}
%
\begin{Description}\relax
Cross-classification of a sample of British males according to each
subject's occupational status and his father's occupational status.
\end{Description}
%
\begin{Usage}
\begin{verbatim}
occupationalStatus
\end{verbatim}
\end{Usage}
%
\begin{Format}
A \code{\LinkA{table}{table}} of counts, with classifying factors
\code{origin} (father's occupational status; levels \code{1:8})
and \code{destination} (son's occupational status; levels \code{1:8}).
\end{Format}
%
\begin{Source}\relax
Goodman, L. A. (1979)
Simple Models for the Analysis of Association in Cross-Classifications
having Ordered Categories.
\emph{J. Am. Stat. Assoc.}, \bold{74} (367), 537--552.

The data set has been in package \Rhref{http://CRAN.R-project.org/package=gnm}{\pkg{gnm}} and been provided by the
package authors.
\end{Source}
%
\begin{Examples}
\begin{ExampleCode}
require(stats); require(graphics)

plot(occupationalStatus)

##  Fit a uniform association model separating diagonal effects
Diag <- as.factor(diag(1:8))
Rscore <- scale(as.numeric(row(occupationalStatus)), scale = FALSE)
Cscore <- scale(as.numeric(col(occupationalStatus)), scale = FALSE)
modUnif <- glm(Freq ~ origin + destination + Diag + Rscore:Cscore,
               family = poisson, data = occupationalStatus)

summary(modUnif)
plot(modUnif) # 4 plots, with warning about  h_ii ~= 1
\end{ExampleCode}
\end{Examples}
\HeaderA{Orange}{Growth of Orange Trees}{Orange}
\keyword{datasets}{Orange}
%
\begin{Description}\relax
The \code{Orange} data frame has 35 rows and 3 columns of records of
the growth of orange trees.
\end{Description}
%
\begin{Usage}
\begin{verbatim}
Orange
\end{verbatim}
\end{Usage}
%
\begin{Format}
This object of class \code{c("nfnGroupedData", "nfGroupedData",
    "groupedData", "data.frame")} containing the following columns:
\begin{description}

\item[Tree] 
an ordered factor indicating the tree on which the measurement is
made.  The ordering is according to increasing maximum diameter.

\item[age] 
a numeric vector giving the age of the tree (days since 1968/12/31)

\item[circumference] 
a numeric vector of trunk circumferences (mm).  This is probably
``circumference at breast height'', a standard measurement in
forestry.


\end{description}

\end{Format}
%
\begin{Details}\relax
   
This dataset was originally part of package \code{nlme}, and that has
methods (including for \code{[}, \code{as.data.frame}, \code{plot} and
\code{print}) for its grouped-data classes. 
\end{Details}
%
\begin{Source}\relax
Draper, N. R. and Smith, H. (1998), \emph{Applied Regression Analysis
(3rd ed)}, Wiley (exercise 24.N).

Pinheiro, J. C. and Bates, D. M. (2000) \emph{Mixed-effects Models
in S and S-PLUS}, Springer.
\end{Source}
%
\begin{Examples}
\begin{ExampleCode}
require(stats); require(graphics)
coplot(circumference ~ age | Tree, data = Orange, show.given = FALSE)
fm1 <- nls(circumference ~ SSlogis(age, Asym, xmid, scal),
           data = Orange, subset = Tree == 3)
plot(circumference ~ age, data = Orange, subset = Tree == 3,
     xlab = "Tree age (days since 1968/12/31)",
     ylab = "Tree circumference (mm)", las = 1,
     main = "Orange tree data and fitted model (Tree 3 only)")
age <- seq(0, 1600, length.out = 101)
lines(age, predict(fm1, list(age = age)))
\end{ExampleCode}
\end{Examples}
\HeaderA{OrchardSprays}{Potency of Orchard Sprays}{OrchardSprays}
\keyword{datasets}{OrchardSprays}
%
\begin{Description}\relax
An experiment was conducted to assess the potency of various
constituents of orchard sprays in repelling honeybees, using a
Latin square design.
\end{Description}
%
\begin{Usage}
\begin{verbatim}
OrchardSprays
\end{verbatim}
\end{Usage}
%
\begin{Format}
A data frame with 64 observations on 4 variables.

\Tabular{rlll}{
[,1]  & rowpos    & numeric & Row of the design\\{}
[,2]  & colpos    & numeric & Column of the design\\{}
[,3]  & treatment & factor  & Treatment level\\{}
[,4]  & decrease  & numeric & Response
}
\end{Format}
%
\begin{Details}\relax
Individual cells of dry comb were filled with measured amounts of lime
sulphur emulsion in sucrose solution.  Seven different concentrations
of lime sulphur ranging from a concentration of 1/100 to 1/1,562,500
in successive factors of 1/5 were used as well as a solution
containing no lime sulphur. 

The responses for the different solutions were obtained by releasing
100 bees into the chamber for two hours, and then measuring the
decrease in volume of the solutions in the various cells.

An \eqn{8 \times 8}{} Latin square design was used and the
treatments were coded as follows:

\Tabular{rl}{
A & highest level of lime sulphur\\{}
B & next highest level of lime sulphur\\{}
. & \\{}
. & \\{}
. & \\{}
G & lowest level of lime sulphur\\{}
H & no lime sulphur
}
\end{Details}
%
\begin{Source}\relax
Finney, D. J. (1947)
\emph{Probit Analysis}.
Cambridge.
\end{Source}
%
\begin{References}\relax
McNeil, D. R. (1977)
\emph{Interactive Data Analysis}.
New York: Wiley.
\end{References}
%
\begin{Examples}
\begin{ExampleCode}
require(graphics)
pairs(OrchardSprays, main = "OrchardSprays data")
\end{ExampleCode}
\end{Examples}
\HeaderA{PlantGrowth}{Results from an Experiment on Plant Growth}{PlantGrowth}
\keyword{datasets}{PlantGrowth}
%
\begin{Description}\relax
Results from an experiment to compare yields (as measured by dried
weight of plants) obtained under a control and two different treatment
conditions.
\end{Description}
%
\begin{Usage}
\begin{verbatim}
PlantGrowth
\end{verbatim}
\end{Usage}
%
\begin{Format}
A data frame of 30 cases on 2 variables.


\Tabular{rll}{
[, 1] & weight & numeric \\{}
[, 2] & group  & factor
}

The levels of \code{group} are `ctrl', `trt1', and `trt2'.
\end{Format}
%
\begin{Source}\relax
Dobson, A. J. (1983)
\emph{An Introduction to Statistical Modelling}.
London: Chapman and Hall.
\end{Source}
%
\begin{Examples}
\begin{ExampleCode}
## One factor ANOVA example from Dobson's book, cf. Table 7.4:
require(stats); require(graphics)
boxplot(weight ~ group, data = PlantGrowth, main = "PlantGrowth data",
        ylab = "Dried weight of plants", col = "lightgray",
        notch = TRUE, varwidth = TRUE)
anova(lm(weight ~ group, data = PlantGrowth))
\end{ExampleCode}
\end{Examples}
\HeaderA{precip}{Annual Precipitation in US Cities}{precip}
\keyword{datasets}{precip}
%
\begin{Description}\relax
The average amount of precipitation (rainfall) in inches for each of
70 United States (and Puerto Rico) cities.
\end{Description}
%
\begin{Usage}
\begin{verbatim}
precip
\end{verbatim}
\end{Usage}
%
\begin{Format}
A named vector of length 70.
\end{Format}
%
\begin{Source}\relax
Statistical Abstracts of the United States, 1975.
\end{Source}
%
\begin{References}\relax
McNeil, D. R. (1977)
\emph{Interactive Data Analysis}.
New York: Wiley.
\end{References}
%
\begin{Examples}
\begin{ExampleCode}
require(graphics)
dotchart(precip[order(precip)], main = "precip data")
title(sub = "Average annual precipitation (in.)")
\end{ExampleCode}
\end{Examples}
\HeaderA{presidents}{Quarterly Approval Ratings of US Presidents}{presidents}
\keyword{datasets}{presidents}
%
\begin{Description}\relax
The (approximately) quarterly approval rating for the President of the
United states from the first quarter of 1945 to the last quarter of
1974.
\end{Description}
%
\begin{Usage}
\begin{verbatim}
presidents
\end{verbatim}
\end{Usage}
%
\begin{Format}
A time series of 120 values.
\end{Format}
%
\begin{Details}\relax
The data are actually a fudged version of the approval ratings.  See
McNeil's book for details.
\end{Details}
%
\begin{Source}\relax
The Gallup Organisation.
\end{Source}
%
\begin{References}\relax
McNeil, D. R. (1977)
\emph{Interactive Data Analysis}.
New York: Wiley.
\end{References}
%
\begin{Examples}
\begin{ExampleCode}
require(stats); require(graphics)
plot(presidents, las = 1, ylab = "Approval rating (%)",
     main = "presidents data")
\end{ExampleCode}
\end{Examples}
\HeaderA{pressure}{Vapor Pressure of Mercury as a Function of Temperature}{pressure}
\keyword{datasets}{pressure}
%
\begin{Description}\relax
Data on the relation between temperature in degrees Celsius and vapor
pressure of mercury in millimeters (of mercury).
\end{Description}
%
\begin{Usage}
\begin{verbatim}
pressure
\end{verbatim}
\end{Usage}
%
\begin{Format}
A data frame with 19 observations on 2 variables.

\Tabular{rlll}{
[, 1] & temperature & numeric & temperature (deg C)\\{}
[, 2] & pressure    & numeric & pressure (mm)
}
\end{Format}
%
\begin{Source}\relax
Weast, R. C., ed. (1973)
\emph{Handbook of Chemistry and Physics}.
CRC Press.
\end{Source}
%
\begin{References}\relax
McNeil, D. R. (1977)
\emph{Interactive Data Analysis}.
New York: Wiley.
\end{References}
%
\begin{Examples}
\begin{ExampleCode}
require(graphics)
plot(pressure, xlab = "Temperature (deg C)",
     ylab = "Pressure (mm of Hg)",
     main = "pressure data: Vapor Pressure of Mercury")
plot(pressure, xlab = "Temperature (deg C)",  log = "y",
     ylab = "Pressure (mm of Hg)",
     main = "pressure data: Vapor Pressure of Mercury")
\end{ExampleCode}
\end{Examples}
\HeaderA{Puromycin}{Reaction Velocity of an Enzymatic Reaction}{Puromycin}
\keyword{datasets}{Puromycin}
%
\begin{Description}\relax
The \code{Puromycin} data frame has 23 rows and 3 columns of the
reaction velocity versus substrate concentration in an enzymatic
reaction involving untreated cells or cells treated with Puromycin.
\end{Description}
%
\begin{Usage}
\begin{verbatim}
Puromycin
\end{verbatim}
\end{Usage}
%
\begin{Format}
This data frame contains the following columns:
\begin{description}

\item[conc] 
a numeric vector of substrate concentrations (ppm)

\item[rate] 
a numeric vector of instantaneous reaction rates (counts/min/min)

\item[state] 
a factor with levels
\code{treated} 
\code{untreated} 


\end{description}

\end{Format}
%
\begin{Details}\relax
Data on the velocity of an enzymatic reaction were obtained
by Treloar (1974).  The number of counts per minute of radioactive
product from the reaction was measured as a function of substrate
concentration in parts per million (ppm) and from these counts the
initial rate (or velocity) of the reaction was calculated
(counts/min/min).  The experiment was conducted once with the enzyme
treated with Puromycin, and once with the enzyme untreated.
\end{Details}
%
\begin{Source}\relax
Bates, D.M. and Watts, D.G. (1988),
\emph{Nonlinear Regression Analysis and Its Applications},
Wiley, Appendix A1.3.

Treloar, M. A. (1974), \emph{Effects of Puromycin on
Galactosyltransferase in Golgi Membranes}, M.Sc. Thesis, U. of
Toronto. 
\end{Source}
%
\begin{SeeAlso}\relax
\code{\LinkA{SSmicmen}{SSmicmen}} for other models fitted to this dataset.
\end{SeeAlso}
%
\begin{Examples}
\begin{ExampleCode}
require(stats); require(graphics)

plot(rate ~ conc, data = Puromycin, las = 1,
     xlab = "Substrate concentration (ppm)",
     ylab = "Reaction velocity (counts/min/min)",
     pch = as.integer(Puromycin$state),
     col = as.integer(Puromycin$state),
     main = "Puromycin data and fitted Michaelis-Menten curves")
## simplest form of fitting the Michaelis-Menten model to these data
fm1 <- nls(rate ~ Vm * conc/(K + conc), data = Puromycin,
           subset = state == "treated",
           start = c(Vm = 200, K = 0.05))
fm2 <- nls(rate ~ Vm * conc/(K + conc), data = Puromycin,
           subset = state == "untreated",
           start = c(Vm = 160, K = 0.05))
summary(fm1)
summary(fm2)
## add fitted lines to the plot
conc <- seq(0, 1.2, length.out = 101)
lines(conc, predict(fm1, list(conc = conc)), lty = 1, col = 1)
lines(conc, predict(fm2, list(conc = conc)), lty = 2, col = 2)
legend(0.8, 120, levels(Puromycin$state),
       col = 1:2, lty = 1:2, pch = 1:2)

## using partial linearity
fm3 <- nls(rate ~ conc/(K + conc), data = Puromycin,
           subset = state == "treated", start = c(K = 0.05),
           algorithm = "plinear")
\end{ExampleCode}
\end{Examples}
\HeaderA{quakes}{Locations of Earthquakes off Fiji}{quakes}
\keyword{datasets}{quakes}
%
\begin{Description}\relax
The data set give the locations of 1000 seismic events of MB > 4.0.
The events occurred in a cube near Fiji since 1964.
\end{Description}
%
\begin{Usage}
\begin{verbatim}
quakes
\end{verbatim}
\end{Usage}
%
\begin{Format}
A data frame with 1000 observations on 5 variables.

\Tabular{rlll}{
[,1] & lat      & numeric & Latitude of event\\{}
[,2] & long     & numeric & Longitude\\{}
[,3] & depth    & numeric & Depth (km)\\{}
[,4] & mag      & numeric & Richter Magnitude\\{}
[,5] & stations & numeric & Number of stations reporting
}
\end{Format}
%
\begin{Details}\relax
There are two clear planes of seismic activity.  One is a major plate
junction; the other is the Tonga trench off New Zealand.  These data
constitute a subsample from a larger dataset of containing 5000
observations.
\end{Details}
%
\begin{Source}\relax
This is one of the Harvard PRIM-H project data sets.  They in turn
obtained it from Dr. John Woodhouse, Dept. of Geophysics, Harvard
University.
\end{Source}
%
\begin{Examples}
\begin{ExampleCode}
require(graphics)
pairs(quakes, main = "Fiji Earthquakes, N = 1000", cex.main=1.2, pch=".")
\end{ExampleCode}
\end{Examples}
\HeaderA{randu}{Random Numbers from Congruential Generator RANDU}{randu}
\keyword{datasets}{randu}
%
\begin{Description}\relax
400 triples of successive random numbers were taken from the VAX
FORTRAN function RANDU running under VMS 1.5.
\end{Description}
%
\begin{Usage}
\begin{verbatim}
randu
\end{verbatim}
\end{Usage}
%
\begin{Format}
A data frame with 400 observations on 3 variables named \code{x},
\code{y} and \code{z} which give the first, second and third random
number in the triple.
\end{Format}
%
\begin{Details}\relax
In three dimensional displays it is evident that the triples fall on
15 parallel planes in 3-space. This can be shown theoretically to be
true for all triples from the RANDU generator.

These particular 400 triples start 5 apart in the sequence, that is
they are ((U[5i+1], U[5i+2], U[5i+3]), i= 0, \dots, 399), and they
are rounded to 6 decimal places.

Under VMS versions 2.0 and higher, this problem has been fixed.
\end{Details}
%
\begin{Source}\relax
David Donoho
\end{Source}
%
\begin{Examples}
\begin{ExampleCode}
## Not run: ## We could re-generate the dataset by the following R code
seed <- as.double(1)
RANDU <- function() {
    seed <<- ((2^16 + 3) * seed) %% (2^31)
    seed/(2^31)
}
for(i in 1:400) {
    U <- c(RANDU(), RANDU(), RANDU(), RANDU(), RANDU())
    print(round(U[1:3], 6))
}
## End(Not run)
\end{ExampleCode}
\end{Examples}
\HeaderA{rivers}{Lengths of Major North American Rivers}{rivers}
\keyword{datasets}{rivers}
%
\begin{Description}\relax
This data set gives the lengths (in miles) of 141 ``major''
rivers in North America, as compiled by the US Geological Survey.
\end{Description}
%
\begin{Usage}
\begin{verbatim}
rivers
\end{verbatim}
\end{Usage}
%
\begin{Format}
A vector containing 141 observations.
\end{Format}
%
\begin{Source}\relax
World Almanac and Book of Facts, 1975, page 406.
\end{Source}
%
\begin{References}\relax
McNeil, D. R. (1977) \emph{Interactive Data Analysis}.
New York: Wiley.
\end{References}
\HeaderA{rock}{Measurements on Petroleum Rock Samples}{rock}
\keyword{datasets}{rock}
%
\begin{Description}\relax
Measurements on 48 rock samples from a petroleum reservoir.
\end{Description}
%
\begin{Usage}
\begin{verbatim}
rock
\end{verbatim}
\end{Usage}
%
\begin{Format}
A data frame with 48 rows and 4 numeric columns.


\Tabular{rll}{
[,1] & area  & area of pores space, in pixels
out of 256 by 256 \\{}
[,2] & peri  & perimeter in pixels \\{}
[,3] & shape & perimeter/sqrt(area) \\{}
[,4] & perm  & permeability in milli-Darcies
}
\end{Format}
%
\begin{Details}\relax
Twelve core samples from petroleum reservoirs were sampled by 4
cross-sections.  Each core sample was measured for permeability, and
each cross-section has total area of pores, total perimeter of
pores, and shape.
\end{Details}
%
\begin{Source}\relax
Data from BP Research, image analysis by Ronit Katz, U. Oxford.
\end{Source}
\HeaderA{sleep}{Student's Sleep Data}{sleep}
\keyword{datasets}{sleep}
%
\begin{Description}\relax
Data which show the effect of two soporific drugs (increase in hours
of sleep compared to control) on 10 patients.
\end{Description}
%
\begin{Usage}
\begin{verbatim}
sleep
\end{verbatim}
\end{Usage}
%
\begin{Format}
A data frame with 20 observations on 3 variables.

\Tabular{rlll}{
[, 1] & extra & numeric & increase in hours of sleep\\{}
[, 2] & group & factor  & drug given\\{}
[, 3] & ID    & factor  & patient ID
}
\end{Format}
%
\begin{Details}\relax
The \code{group} variable name may be misleading about the data:
They represent measurements on 10 persons, not in groups.

\end{Details}
%
\begin{Source}\relax
Cushny, A. R. and Peebles, A. R. (1905)
The action of optical isomers: II hyoscines.
\emph{The Journal of Physiology} \bold{32}, 501--510.

Student (1908)
The probable error of the mean.
\emph{Biometrika}, \bold{6}, 20.
\end{Source}
%
\begin{References}\relax
Scheffé, Henry (1959)
\emph{The Analysis of Variance}.
New York, NY: Wiley.
\end{References}
%
\begin{Examples}
\begin{ExampleCode}
require(stats)
## Student's paired t-test
with(sleep,
     t.test(extra[group == 1],
            extra[group == 2], paired = TRUE))
\end{ExampleCode}
\end{Examples}
\HeaderA{stackloss}{Brownlee's Stack Loss Plant Data}{stackloss}
\aliasA{stack.loss}{stackloss}{stack.loss}
\aliasA{stack.x}{stackloss}{stack.x}
\keyword{datasets}{stackloss}
%
\begin{Description}\relax
Operational data of a plant for the oxidation of ammonia to nitric
acid.
\end{Description}
%
\begin{Usage}
\begin{verbatim}
stackloss

stack.x
stack.loss
\end{verbatim}
\end{Usage}
%
\begin{Format}
\code{stackloss} is a data frame with 21 observations on 4 variables.


\Tabular{rll}{
[,1] & \code{Air Flow}   & Flow of cooling air\\{}
[,2] & \code{Water Temp} & Cooling Water Inlet
Temperature\\{}
[,3] &  \code{Acid Conc.} & Concentration of acid [per
1000, minus 500]\\{}
[,4] &  \code{stack.loss} & Stack loss\\{}
}

For compatibility with S-PLUS, the data sets \code{stack.x}, a matrix
with the first three (independent) variables of the data frame, and
\code{stack.loss}, the numeric vector giving the fourth (dependent)
variable, are provided as well.
\end{Format}
%
\begin{Details}\relax
``Obtained from 21 days of operation of a plant for the
oxidation of ammonia (NH\eqn{_3}{}) to nitric acid
(HNO\eqn{_3}{}).  The nitric oxides produced are absorbed in a
countercurrent absorption tower''.
(Brownlee, cited by Dodge, slightly reformatted by MM.)

\code{Air Flow} represents the rate of operation of the plant.
\code{Water Temp} is the temperature of cooling water circulated
through coils in the absorption tower.
\code{Acid Conc.} is the concentration of the acid circulating, minus
50, times 10: that is, 89 corresponds to 58.9 per cent acid.
\code{stack.loss} (the dependent variable) is 10 times the percentage
of the ingoing ammonia to the plant that escapes from the absorption
column unabsorbed; that is, an (inverse) measure of the over-all
efficiency of the plant.
\end{Details}
%
\begin{Source}\relax
Brownlee, K. A. (1960, 2nd ed. 1965)
\emph{Statistical Theory and Methodology in Science and Engineering}.
New York: Wiley. pp. 491--500.
\end{Source}
%
\begin{References}\relax
Becker, R. A., Chambers, J. M. and Wilks, A. R. (1988)
\emph{The New S Language}.
Wadsworth \& Brooks/Cole.

Dodge, Y. (1996)
The guinea pig of multiple regression. In:
\emph{Robust Statistics, Data Analysis, and Computer Intensive
Methods; In Honor of Peter Huber's 60th Birthday}, 1996,
\emph{Lecture Notes in Statistics} \bold{109}, Springer-Verlag, New York.
\end{References}
%
\begin{Examples}
\begin{ExampleCode}
require(stats)
summary(lm.stack <- lm(stack.loss ~ stack.x))
\end{ExampleCode}
\end{Examples}
\HeaderA{state}{US State Facts and Figures}{state}
\methaliasA{state.abb}{state}{state.abb}
\methaliasA{state.area}{state}{state.area}
\methaliasA{state.center}{state}{state.center}
\methaliasA{state.division}{state}{state.division}
\methaliasA{state.name}{state}{state.name}
\methaliasA{state.region}{state}{state.region}
\methaliasA{state.x77}{state}{state.x77}
\keyword{datasets}{state}
%
\begin{Description}\relax
Data sets related to the 50 states of the United States of
America.
\end{Description}
%
\begin{Usage}
\begin{verbatim}
state.abb
state.area
state.center
state.division
state.name
state.region
state.x77
\end{verbatim}
\end{Usage}
%
\begin{Details}\relax
\R{} currently contains the following ``state'' data sets.  Note
that all data are arranged according to alphabetical order of the
state names.
\begin{description}

\item[\code{state.abb}:] character vector of 2-letter abbreviations
for the state names.
\item[\code{state.area}:] numeric vector of state areas (in square
miles).
\item[\code{state.center}:]  list with components named \code{x} and
\code{y} giving the approximate geographic center of each state in
negative longitude and latitude.  Alaska and Hawaii are placed
just off the West Coast.
\item[\code{state.division}:] factor giving state divisions (New
England, Middle Atlantic, South Atlantic, East South Central, West
South Central, East North Central, West North Central, Mountain,
and Pacific).
\item[\code{state.name}:] character vector giving the full state
names.
\item[\code{state.region}:] factor giving the region (Northeast,
South, North Central, West) that each state belongs to.
\item[\code{state.x77}:] matrix with 50 rows and 8 columns giving
the following statistics in the respective columns.
\begin{description}

\item[\code{Population}:] population estimate as of July 1,
1975
\item[\code{Income}:] per capita income (1974)
\item[\code{Illiteracy}:] illiteracy (1970, percent of
population)
\item[\code{Life Exp}:] life expectancy in years (1969--71)
\item[\code{Murder}:] murder and non-negligent manslaughter rate
per 100,000 population (1976)
\item[\code{HS Grad}:] percent high-school graduates (1970)
\item[\code{Frost}:] mean number of days with minimum
temperature below freezing (1931--1960) in capital or large
city
\item[\code{Area}:] land area in square miles

\end{description}


\end{description}

\end{Details}
%
\begin{Source}\relax
U.S. Department of Commerce, Bureau of the Census (1977)
\emph{Statistical Abstract of the United States}.

U.S. Department of Commerce, Bureau of the Census (1977)
\emph{County and City Data Book}.
\end{Source}
%
\begin{References}\relax
Becker, R. A., Chambers, J. M. and Wilks, A. R. (1988)
\emph{The New S Language}.
Wadsworth \& Brooks/Cole.
\end{References}
\HeaderA{sunspot.month}{Monthly Sunspot Data, 1749--1997}{sunspot.month}
\keyword{datasets}{sunspot.month}
%
\begin{Description}\relax
Monthly numbers of sunspots. 
\end{Description}
%
\begin{Usage}
\begin{verbatim}
sunspot.month
\end{verbatim}
\end{Usage}
%
\begin{Format}
The univariate time series \code{sunspot.year} and
\code{sunspot.month} contain 289 and 2988 observations, respectively.
The objects are of class \code{"ts"}.
\end{Format}
%
\begin{Source}\relax
World Data Center-C1 For Sunspot Index
Royal Observatory of Belgium, Av. Circulaire, 3, B-1180 BRUSSELS
\url{http://www.oma.be/KSB-ORB/SIDC/sidc_txt.html}
\end{Source}
%
\begin{SeeAlso}\relax
\code{sunspot.month} is a longer version of \code{\LinkA{sunspots}{sunspots}}
that runs until 1988 rather than 1983.
\end{SeeAlso}
%
\begin{Examples}
\begin{ExampleCode}
require(stats); require(graphics)
## Compare the monthly series 
plot (sunspot.month, main = "sunspot.month [stats]", col = 2)
lines(sunspots) # "very barely" see something

## Now look at the difference :
all(tsp(sunspots)     [c(1,3)] ==
    tsp(sunspot.month)[c(1,3)]) ## Start & Periodicity are the same
n1 <- length(sunspots)
table(eq <- sunspots == sunspot.month[1:n1]) #>  132  are different !
i <- which(!eq) 
rug(time(eq)[i])
s1 <- sunspots[i] ; s2 <- sunspot.month[i]
cbind(i = i, sunspots = s1, ss.month = s2,
      perc.diff = round(100*2*abs(s1-s2)/(s1+s2), 1))
\end{ExampleCode}
\end{Examples}
\HeaderA{sunspot.year}{Yearly Sunspot Data, 1700--1988}{sunspot.year}
\keyword{datasets}{sunspot.year}
%
\begin{Description}\relax
Yearly numbers of sunspots. 
\end{Description}
%
\begin{Usage}
\begin{verbatim}
sunspot.year
\end{verbatim}
\end{Usage}
%
\begin{Format}
The univariate time series \code{sunspot.year} contains 289
observations, and is of class \code{"ts"}.
\end{Format}
%
\begin{Source}\relax
H. Tong (1996)
\emph{Non-Linear Time Series}. Clarendon Press, Oxford, p. 471. 
\end{Source}
\HeaderA{sunspots}{Monthly Sunspot Numbers, 1749--1983}{sunspots}
\keyword{datasets}{sunspots}
%
\begin{Description}\relax
Monthly mean relative sunspot numbers from 1749 to 1983.  Collected at
Swiss Federal Observatory, Zurich until 1960, then Tokyo Astronomical
Observatory.
\end{Description}
%
\begin{Usage}
\begin{verbatim}
sunspots
\end{verbatim}
\end{Usage}
%
\begin{Format}
A time series of monthly data from 1749 to 1983.
\end{Format}
%
\begin{Source}\relax
Andrews, D. F. and Herzberg, A. M. (1985)
\emph{Data: A Collection of Problems from Many Fields for the
Student and Research Worker}.
New York: Springer-Verlag.
\end{Source}
%
\begin{SeeAlso}\relax
\code{\LinkA{sunspot.month}{sunspot.month}} has a longer (and a bit different) series.
\end{SeeAlso}
%
\begin{Examples}
\begin{ExampleCode}
require(graphics)
plot(sunspots, main = "sunspots data", xlab = "Year",
     ylab = "Monthly sunspot numbers")
\end{ExampleCode}
\end{Examples}
\HeaderA{swiss}{Swiss Fertility and Socioeconomic Indicators (1888) Data}{swiss}
\keyword{datasets}{swiss}
%
\begin{Description}\relax
Standardized fertility measure and socio-economic indicators for each
of 47 French-speaking provinces of Switzerland at about 1888.
\end{Description}
%
\begin{Usage}
\begin{verbatim}
swiss
\end{verbatim}
\end{Usage}
%
\begin{Format}
A data frame with 47 observations on 6 variables, \emph{each} of which
is in percent, i.e., in \eqn{[0, 100]}{}.


\Tabular{rll}{
[,1] & Fertility & \eqn{I_g}{}, `common standardized
fertility measure'\\{}
[,2] & Agriculture& \% of males involved in agriculture
as occupation\\{}
[,3] & Examination& \% draftees receiving highest mark
on army examination\\{}
[,4] & Education & \% education beyond primary school for draftees.\\{}
[,5] & Catholic & \% `catholic' (as opposed to `protestant').\\{}
[,6] & Infant.Mortality& live births who live less than 1
year.
}

All variables but `Fertility' give proportions of the
population.
\end{Format}
%
\begin{Details}\relax
(paraphrasing Mosteller and Tukey):

Switzerland, in 1888, was entering a period known as the
\emph{demographic transition}; i.e., its fertility was beginning to
fall from the high level typical of underdeveloped countries.

The data collected are for 47 French-speaking ``provinces'' at
about 1888.

Here, all variables are scaled to \eqn{[0, 100]}{}, where in the
original, all but \code{"Catholic"} were scaled to \eqn{[0, 1]}{}.
\end{Details}
%
\begin{Note}\relax
Files for all 182 districts in 1888 and other years have been available at
\url{http://opr.princeton.edu/archive/eufert/switz.html} or
\url{http://opr.princeton.edu/archive/pefp/switz.asp}.

They state that variables \code{Examination} and \code{Education}
are averages for 1887, 1888 and 1889.
\end{Note}
%
\begin{Source}\relax
Project ``16P5'', pages 549--551 in

Mosteller, F. and Tukey, J. W. (1977)
\emph{Data Analysis and Regression: A Second Course in Statistics}.
Addison-Wesley, Reading Mass.

indicating their source as
``Data used by permission of Franice van de Walle. Office of
Population Research, Princeton University, 1976.  Unpublished data
assembled under NICHD contract number No 1-HD-O-2077.''
\end{Source}
%
\begin{References}\relax
Becker, R. A., Chambers, J. M. and Wilks, A. R. (1988)
\emph{The New S Language}.
Wadsworth \& Brooks/Cole.
\end{References}
%
\begin{Examples}
\begin{ExampleCode}
require(stats); require(graphics)
pairs(swiss, panel = panel.smooth, main = "swiss data",
      col = 3 + (swiss$Catholic > 50))
summary(lm(Fertility ~ . , data = swiss))
\end{ExampleCode}
\end{Examples}
\HeaderA{Theoph}{Pharmacokinetics of Theophylline}{Theoph}
\keyword{datasets}{Theoph}
%
\begin{Description}\relax
The \code{Theoph} data frame has 132 rows and 5 columns of data from
an experiment on the pharmacokinetics of theophylline.
\end{Description}
%
\begin{Usage}
\begin{verbatim}
Theoph
\end{verbatim}
\end{Usage}
%
\begin{Format}
This object of class \code{c("nfnGroupedData", "nfGroupedData",
    "groupedData", "data.frame")} containing the following columns:
\begin{description}

\item[Subject] 
an ordered factor with levels \code{1}, \dots, \code{12}
identifying the subject on whom the observation was made.  The
ordering is by increasing maximum concentration of theophylline
observed.

\item[Wt] 
weight of the subject (kg).

\item[Dose] 
dose of theophylline administered orally to the subject (mg/kg).

\item[Time] 
time since drug administration when the sample was drawn (hr).

\item[conc] 
theophylline concentration in the sample (mg/L).


\end{description}

\end{Format}
%
\begin{Details}\relax
Boeckmann, Sheiner and Beal (1994) report data from a study by
Dr. Robert Upton of the kinetics of the anti-asthmatic drug
theophylline.  Twelve subjects were given oral doses of theophylline
then serum concentrations were measured at 11 time points over the
next 25 hours.

These data are analyzed in Davidian and Giltinan (1995) and Pinheiro
and Bates (2000) using a two-compartment open pharmacokinetic model,
for which a self-starting model function, \code{SSfol}, is available.

This dataset was originally part of package \code{nlme}, and that has
methods (including for \code{[}, \code{as.data.frame}, \code{plot} and
\code{print}) for its grouped-data classes. 
\end{Details}
%
\begin{Source}\relax
Boeckmann, A. J., Sheiner, L. B. and Beal, S. L. (1994), \emph{NONMEM
Users Guide: Part V}, NONMEM Project Group, University of
California, San Francisco.

Davidian, M. and Giltinan, D. M. (1995) \emph{Nonlinear Models for
Repeated Measurement Data}, Chapman \& Hall (section 5.5, p. 145 and
section 6.6, p. 176)

Pinheiro, J. C. and Bates, D. M. (2000) \emph{Mixed-effects Models in
S and S-PLUS}, Springer (Appendix A.29)
\end{Source}
%
\begin{SeeAlso}\relax
\code{\LinkA{SSfol}{SSfol}}
\end{SeeAlso}
%
\begin{Examples}
\begin{ExampleCode}
require(stats); require(graphics)

coplot(conc ~ Time | Subject, data = Theoph, show.given = FALSE)
Theoph.4 <- subset(Theoph, Subject == 4)
fm1 <- nls(conc ~ SSfol(Dose, Time, lKe, lKa, lCl),
           data = Theoph.4)
summary(fm1)
plot(conc ~ Time, data = Theoph.4,
     xlab = "Time since drug administration (hr)",
     ylab = "Theophylline concentration (mg/L)",
     main = "Observed concentrations and fitted model",
     sub  = "Theophylline data - Subject 4 only",
     las = 1, col = 4)
xvals <- seq(0, par("usr")[2], length.out = 55)
lines(xvals, predict(fm1, newdata = list(Time = xvals)),
      col = 4)
\end{ExampleCode}
\end{Examples}
\HeaderA{Titanic}{Survival of passengers on the Titanic}{Titanic}
\keyword{datasets}{Titanic}
%
\begin{Description}\relax
This data set provides information on the fate of passengers on the
fatal maiden voyage of the ocean liner `Titanic', summarized according
to economic status (class), sex, age and survival.
\end{Description}
%
\begin{Usage}
\begin{verbatim}
Titanic
\end{verbatim}
\end{Usage}
%
\begin{Format}
A 4-dimensional array resulting from cross-tabulating 2201
observations on 4 variables.  The variables and their levels are as
follows:


\Tabular{rll}{
No & Name & Levels\\{}
1 & Class & 1st, 2nd, 3rd, Crew\\{}
2 & Sex & Male, Female\\{}
3 & Age & Child, Adult\\{}
4 & Survived & No, Yes
}
\end{Format}
%
\begin{Details}\relax
The sinking of the Titanic is a famous event, and new books are still
being published about it.  Many well-known facts---from the
proportions of first-class passengers to the `women and
children first' policy, and the fact that that policy was not
entirely successful in saving the women and children in the third
class---are reflected in the survival rates for various classes of
passenger.

These data were originally collected by the British Board of Trade in
their investigation of the sinking.  Note that there is not complete
agreement among primary sources as to the exact numbers on board,
rescued, or lost.

Due in particular to the very successful film `Titanic', the last
years saw a rise in public interest in the Titanic.  Very detailed
data about the passengers is now available on the Internet, at sites
such as \emph{Encyclopedia Titanica}
(\url{http://www.rmplc.co.uk/eduweb/sites/phind}).
\end{Details}
%
\begin{Source}\relax
Dawson, Robert J. MacG. (1995),
The `Unusual Episode' Data Revisited.
\emph{Journal of Statistics Education}, \bold{3}.
\url{http://www.amstat.org/publications/jse/v3n3/datasets.dawson.html}

The source provides a data set recording class, sex, age, and survival
status for each person on board of the Titanic, and is based on data
originally collected by the British Board of Trade and reprinted in:

British Board of Trade (1990),
\emph{Report on the Loss of the `Titanic' (S.S.)}.
British Board of Trade Inquiry Report (reprint).
Gloucester, UK: Allan Sutton Publishing.
\end{Source}
%
\begin{Examples}
\begin{ExampleCode}
require(graphics)
mosaicplot(Titanic, main = "Survival on the Titanic")
## Higher survival rates in children?
apply(Titanic, c(3, 4), sum)
## Higher survival rates in females?
apply(Titanic, c(2, 4), sum)
## Use loglm() in package 'MASS' for further analysis ...
\end{ExampleCode}
\end{Examples}
\HeaderA{ToothGrowth}{The Effect of Vitamin C on Tooth Growth in Guinea Pigs}{ToothGrowth}
\keyword{datasets}{ToothGrowth}
%
\begin{Description}\relax
The response is the length of odontoblasts (teeth) in each of 10
guinea pigs at each of three dose levels of Vitamin C (0.5, 1, and 2
mg) with each of two delivery methods (orange juice or ascorbic
acid).
\end{Description}
%
\begin{Usage}
\begin{verbatim}
ToothGrowth
\end{verbatim}
\end{Usage}
%
\begin{Format}
A data frame with 60 observations on 3 variables.

\Tabular{rlll}{
[,1]  & len   & numeric  & Tooth length\\{}
[,2]  & supp  & factor   & Supplement type (VC or OJ).\\{}
[,3]  & dose  & numeric  & Dose in milligrams.
}
\end{Format}
%
\begin{Source}\relax
C. I. Bliss (1952)
\emph{The Statistics of Bioassay}.
Academic Press.
\end{Source}
%
\begin{References}\relax
McNeil, D. R. (1977)
\emph{Interactive Data Analysis}.
New York: Wiley.
\end{References}
%
\begin{Examples}
\begin{ExampleCode}
require(graphics)
coplot(len ~ dose | supp, data = ToothGrowth, panel = panel.smooth,
       xlab = "ToothGrowth data: length vs dose, given type of supplement")
\end{ExampleCode}
\end{Examples}
\HeaderA{treering}{Yearly Treering Data, -6000--1979}{treering}
\keyword{datasets}{treering}
%
\begin{Description}\relax
Contains normalized tree-ring widths in dimensionless units.
\end{Description}
%
\begin{Usage}
\begin{verbatim}
treering
\end{verbatim}
\end{Usage}
%
\begin{Format}
A univariate time series with 7981 observations. The object is of
class \code{"ts"}.

Each tree ring corresponds to one year.
\end{Format}
%
\begin{Details}\relax
The data were recorded by Donald A. Graybill, 1980, from 
Gt Basin Bristlecone Pine 2805M, 3726-11810 in Methuselah Walk, California.
\end{Details}
%
\begin{Source}\relax
Time Series Data Library:
\url{http://www-personal.buseco.monash.edu.au/~hyndman/TSDL/},
series \file{CA535.DAT}
\end{Source}
%
\begin{References}\relax
For background on Bristlecone pines and Methuselah Walk, see
\url{http://www.sonic.net/bristlecone/}; for some photos see
\url{http://www.ltrr.arizona.edu/~hallman/sitephotos/meth.html}
\end{References}
\HeaderA{trees}{Girth, Height and Volume for Black Cherry Trees}{trees}
\keyword{datasets}{trees}
%
\begin{Description}\relax
This data set provides measurements of the girth, height and volume
of timber in 31 felled black cherry trees.  Note that girth is the
diameter of the tree (in inches) measured at 4 ft 6 in above the
ground.
\end{Description}
%
\begin{Usage}
\begin{verbatim}
trees
\end{verbatim}
\end{Usage}
%
\begin{Format}
A data frame with 31 observations on 3 variables.

\Tabular{rlll}{
\code{[,1]} & \code{Girth} & numeric
& Tree diameter in inches\\{}
\code{[,2]} & \code{Height}& numeric
& Height in ft\\{}
\code{[,3]} & \code{Volume}& numeric
& Volume of timber in cubic ft\\{}}
\end{Format}
%
\begin{Source}\relax
Ryan, T. A., Joiner, B. L. and Ryan, B. F. (1976)
\emph{The Minitab Student Handbook}.
Duxbury Press.
\end{Source}
%
\begin{References}\relax
Atkinson, A. C. (1985)
\emph{Plots, Transformations and Regression}.
Oxford University Press.
\end{References}
%
\begin{Examples}
\begin{ExampleCode}
require(stats); require(graphics)
pairs(trees, panel = panel.smooth, main = "trees data")
plot(Volume ~ Girth, data = trees, log = "xy")
coplot(log(Volume) ~ log(Girth) | Height, data = trees,
       panel = panel.smooth)
summary(fm1 <- lm(log(Volume) ~ log(Girth), data = trees))
summary(fm2 <- update(fm1, ~ . + log(Height), data = trees))
step(fm2)
## i.e., Volume ~= c * Height * Girth^2  seems reasonable
\end{ExampleCode}
\end{Examples}
\HeaderA{UCBAdmissions}{Student Admissions at UC Berkeley}{UCBAdmissions}
\keyword{datasets}{UCBAdmissions}
%
\begin{Description}\relax
Aggregate data on applicants to graduate school at Berkeley for the
six largest departments in 1973 classified by admission and sex.
\end{Description}
%
\begin{Usage}
\begin{verbatim}
UCBAdmissions
\end{verbatim}
\end{Usage}
%
\begin{Format}
A 3-dimensional array resulting from cross-tabulating 4526
observations on 3 variables.  The variables and their levels are as
follows:


\Tabular{rll}{
No & Name & Levels\\{}
1 & Admit & Admitted, Rejected\\{}
2 & Gender & Male, Female\\{}
3 & Dept & A, B, C, D, E, F
}
\end{Format}
%
\begin{Details}\relax
This data set is frequently used for illustrating Simpson's paradox,
see Bickel et al.\bsl{} (1975).  At issue is whether the data show evidence
of sex bias in admission practices.  There were 2691 male applicants,
of whom 1198 (44.5\%) were admitted, compared with 1835 female
applicants of whom 557 (30.4\%) were admitted.  This gives a sample
odds ratio of 1.83, indicating that males were almost twice as likely
to be admitted.  In fact, graphical methods (as in the example below)
or log-linear modelling show that the apparent association between
admission and sex stems from differences in the tendency of males and
females to apply to the individual departments (females used to apply
\emph{more} to departments with higher rejection rates).

This data set can also be used for illustrating methods for graphical
display of categorical data, such as the general-purpose mosaic plot
or the fourfold display for 2-by-2-by-\eqn{k}{} tables.  See the
home page of Michael Friendly
(\url{http://www.math.yorku.ca/SCS/friendly.html}) for further
information.
\end{Details}
%
\begin{References}\relax
Bickel, P. J., Hammel, E. A., and O'Connell, J. W. (1975)
Sex bias in graduate admissions: Data from Berkeley.
\emph{Science}, \bold{187}, 398--403.
\end{References}
%
\begin{Examples}
\begin{ExampleCode}
require(graphics)
## Data aggregated over departments
apply(UCBAdmissions, c(1, 2), sum)
mosaicplot(apply(UCBAdmissions, c(1, 2), sum),
           main = "Student admissions at UC Berkeley")
## Data for individual departments
opar <- par(mfrow = c(2, 3), oma = c(0, 0, 2, 0))
for(i in 1:6)
  mosaicplot(UCBAdmissions[,,i],
    xlab = "Admit", ylab = "Sex",
    main = paste("Department", LETTERS[i]))
mtext(expression(bold("Student admissions at UC Berkeley")),
      outer = TRUE, cex = 1.5)
par(opar)
\end{ExampleCode}
\end{Examples}
\HeaderA{UKDriverDeaths}{Road Casualties in Great Britain 1969--84}{UKDriverDeaths}
\aliasA{Seatbelts}{UKDriverDeaths}{Seatbelts}
\keyword{datasets}{UKDriverDeaths}
%
\begin{Description}\relax
\code{UKDriverDeaths} is a time series giving the monthly totals
of car drivers in
Great Britain killed or seriously injured Jan 1969 to Dec 1984.
Compulsory wearing of seat belts was introduced on 31 Jan 1983.

\code{Seatbelts} is more information on the same problem.
\end{Description}
%
\begin{Usage}
\begin{verbatim}
UKDriverDeaths
Seatbelts
\end{verbatim}
\end{Usage}
%
\begin{Format}
\code{Seatbelts} is a multiple time series, with columns
\begin{description}

\item[\code{DriversKilled}] car drivers killed.
\item[\code{drivers}] same as \code{UKDriverDeaths}.
\item[\code{front}] front-seat passengers killed or seriously injured.
\item[\code{rear}] rear-seat passengers killed or seriously injured.
\item[\code{kms}] distance driven.
\item[\code{PetrolPrice}] petrol price.
\item[\code{VanKilled}] number of van (`light goods vehicle')
drivers.
\item[\code{law}] 0/1: was the law in effect that month?

\end{description}

\end{Format}
%
\begin{Source}\relax
Harvey, A.C. (1989)
\emph{Forecasting, Structural Time Series Models and the Kalman Filter.}
Cambridge University Press, pp. 519--523.

Durbin, J. and Koopman, S. J. (2001) \emph{Time Series Analysis by
State Space Methods.}  Oxford University Press.
\url{http://www.ssfpack.com/dkbook/}
\end{Source}
%
\begin{References}\relax
Harvey, A. C. and Durbin, J. (1986) The effects of seat belt
legislation on British road casualties: A case study in structural
time series modelling. \emph{Journal of the Royal Statistical Society}
series B, \bold{149}, 187--227.
\end{References}
%
\begin{Examples}
\begin{ExampleCode}
require(stats); require(graphics)
## work with pre-seatbelt period to identify a model, use logs
work <- window(log10(UKDriverDeaths), end = 1982+11/12)
par(mfrow = c(3,1))
plot(work); acf(work); pacf(work)
par(mfrow = c(1,1))
(fit <- arima(work, c(1,0,0), seasonal = list(order= c(1,0,0))))
z <- predict(fit, n.ahead = 24)
ts.plot(log10(UKDriverDeaths), z$pred, z$pred+2*z$se, z$pred-2*z$se,
        lty = c(1,3,2,2), col = c("black", "red", "blue", "blue"))

## now see the effect of the explanatory variables
X <- Seatbelts[, c("kms", "PetrolPrice", "law")]
X[, 1] <- log10(X[, 1]) - 4
arima(log10(Seatbelts[, "drivers"]), c(1,0,0),
      seasonal = list(order= c(1,0,0)), xreg = X)
\end{ExampleCode}
\end{Examples}
\HeaderA{UKgas}{UK Quarterly Gas Consumption}{UKgas}
\keyword{datasets}{UKgas}
%
\begin{Description}\relax
Quarterly UK gas consumption from 1960Q1 to 1986Q4, in millions of therms.
\end{Description}
%
\begin{Usage}
\begin{verbatim}
UKgas
\end{verbatim}
\end{Usage}
%
\begin{Format}
A quarterly time series of length 108.
\end{Format}
%
\begin{Source}\relax
Durbin, J. and Koopman, S. J. (2001) \emph{Time Series Analysis by
State Space Methods.}  Oxford University Press.
\url{http://www.ssfpack.com/dkbook/}
\end{Source}
%
\begin{Examples}
\begin{ExampleCode}
## maybe str(UKgas) ; plot(UKgas) ...
\end{ExampleCode}
\end{Examples}
\HeaderA{UKLungDeaths}{Monthly Deaths from Lung Diseases in the UK}{UKLungDeaths}
\aliasA{fdeaths}{UKLungDeaths}{fdeaths}
\aliasA{ldeaths}{UKLungDeaths}{ldeaths}
\aliasA{mdeaths}{UKLungDeaths}{mdeaths}
\keyword{datasets}{UKLungDeaths}
%
\begin{Description}\relax
Three time series giving the monthly deaths from bronchitis,
emphysema and asthma in the UK, 1974--1979,
both sexes (\code{ldeaths}), males (\code{mdeaths}) and
females (\code{fdeaths}).
\end{Description}
%
\begin{Usage}
\begin{verbatim}
ldeaths
fdeaths
mdeaths
\end{verbatim}
\end{Usage}
%
\begin{Source}\relax
P. J. Diggle (1990)
\emph{Time Series: A Biostatistical Introduction.}
Oxford, table A.3
\end{Source}
%
\begin{Examples}
\begin{ExampleCode}
require(stats); require(graphics) # for time
plot(ldeaths) 
plot(mdeaths, fdeaths) 
## Better labels:
yr <- floor(tt <- time(mdeaths))
plot(mdeaths, fdeaths,
     xy.labels = paste(month.abb[12*(tt - yr)], yr-1900, sep="'"))
\end{ExampleCode}
\end{Examples}
\HeaderA{USAccDeaths}{Accidental Deaths in the US 1973--1978}{USAccDeaths}
\keyword{datasets}{USAccDeaths}
%
\begin{Description}\relax
A time series giving the monthly totals of accidental deaths in the
USA.  The values for the first six months of 1979 are 7798 7406 8363
8460 9217 9316.
\end{Description}
%
\begin{Usage}
\begin{verbatim}
USAccDeaths
\end{verbatim}
\end{Usage}
%
\begin{Source}\relax
P. J. Brockwell and R. A. Davis (1991)
\emph{Time Series: Theory and Methods.}
Springer, New York.
\end{Source}
\HeaderA{USArrests}{Violent Crime Rates by US State}{USArrests}
\keyword{datasets}{USArrests}
%
\begin{Description}\relax
This data set contains statistics, in arrests per 100,000 residents
for assault, murder, and rape in each of the 50 US states in 1973.
Also given is the percent of the population living in urban areas.
\end{Description}
%
\begin{Usage}
\begin{verbatim}
USArrests
\end{verbatim}
\end{Usage}
%
\begin{Format}
A data frame with 50 observations on 4 variables.


\Tabular{rlll}{
[,1]  & Murder    & numeric
& Murder arrests (per 100,000)\\{}
[,2]  & Assault   & numeric
& Assault arrests (per 100,000)\\{}
[,3]  & UrbanPop  & numeric
& Percent urban population\\{}
[,4]  & Rape      & numeric
& Rape arrests (per 100,000)
}
\end{Format}
%
\begin{Source}\relax
World Almanac and Book of facts 1975.  (Crime rates).

Statistical Abstracts of the United States 1975.  (Urban rates).
\end{Source}
%
\begin{References}\relax
McNeil, D. R. (1977)
\emph{Interactive Data Analysis}.
New York: Wiley.
\end{References}
%
\begin{SeeAlso}\relax
The \code{\LinkA{state}{state}} data sets.
\end{SeeAlso}
%
\begin{Examples}
\begin{ExampleCode}
require(graphics)
pairs(USArrests, panel = panel.smooth, main = "USArrests data")
\end{ExampleCode}
\end{Examples}
\HeaderA{USJudgeRatings}{Lawyers' Ratings of State Judges in the US Superior Court}{USJudgeRatings}
\keyword{datasets}{USJudgeRatings}
%
\begin{Description}\relax
Lawyers' ratings of state judges in the US Superior Court.
\end{Description}
%
\begin{Usage}
\begin{verbatim}
USJudgeRatings
\end{verbatim}
\end{Usage}
%
\begin{Format}
A data frame containing 43 observations on 12 numeric variables.

\Tabular{rll}{
[,1] & CONT & Number of contacts of lawyer with judge.\\{}
[,2] & INTG & Judicial integrity.\\{}
[,3] & DMNR & Demeanor.\\{}
[,4] & DILG & Diligence.\\{}
[,5] & CFMG & Case flow managing.\\{}
[,6] & DECI & Prompt decisions.\\{}
[,7] & PREP & Preparation for trial.\\{}
[,8] & FAMI & Familiarity with law.\\{}
[,9] & ORAL & Sound oral rulings.\\{}
[,10] & WRIT & Sound written rulings.\\{}
[,11] & PHYS & Physical ability.\\{}
[,12] & RTEN & Worthy of retention.
}
\end{Format}
%
\begin{Source}\relax
New Haven Register, 14 January, 1977 (from John Hartigan).
\end{Source}
%
\begin{Examples}
\begin{ExampleCode}
require(graphics)
pairs(USJudgeRatings, main = "USJudgeRatings data")
\end{ExampleCode}
\end{Examples}
\HeaderA{USPersonalExpenditure}{Personal Expenditure Data}{USPersonalExpenditure}
\keyword{datasets}{USPersonalExpenditure}
%
\begin{Description}\relax
This data set consists of United States personal expenditures (in
billions of dollars) in the categories; food and tobacco, household
operation, medical and health, personal care, and private education
for the years 1940, 1945, 1950, 1955 and 1960.
\end{Description}
%
\begin{Usage}
\begin{verbatim}
USPersonalExpenditure
\end{verbatim}
\end{Usage}
%
\begin{Format}
A matrix with 5 rows and 5 columns.
\end{Format}
%
\begin{Source}\relax
The World Almanac and Book of Facts, 1962, page 756.
\end{Source}
%
\begin{References}\relax
Tukey, J. W. (1977)
\emph{Exploratory Data Analysis}.
Addison-Wesley.

McNeil, D. R. (1977)
\emph{Interactive Data Analysis}.
Wiley.
\end{References}
%
\begin{Examples}
\begin{ExampleCode}
require(stats) # for medpolish
USPersonalExpenditure
medpolish(log10(USPersonalExpenditure))
\end{ExampleCode}
\end{Examples}
\HeaderA{uspop}{Populations Recorded by the US Census}{uspop}
\keyword{datasets}{uspop}
%
\begin{Description}\relax
This data set gives the population of the United States (in millions)
as recorded by the decennial census for the period 1790--1970.
\end{Description}
%
\begin{Usage}
\begin{verbatim}
uspop
\end{verbatim}
\end{Usage}
%
\begin{Format}
A time series of 19 values.
\end{Format}
%
\begin{Source}\relax
McNeil, D. R. (1977)
\emph{Interactive Data Analysis}.
New York: Wiley.
\end{Source}
%
\begin{Examples}
\begin{ExampleCode}
require(graphics)
plot(uspop, log = "y", main = "uspop data", xlab = "Year",
     ylab = "U.S. Population (millions)")
\end{ExampleCode}
\end{Examples}
\HeaderA{VADeaths}{Death Rates in Virginia (1940)}{VADeaths}
\keyword{datasets}{VADeaths}
%
\begin{Description}\relax
Death rates per 1000 in Virginia in 1940.
\end{Description}
%
\begin{Usage}
\begin{verbatim}
VADeaths
\end{verbatim}
\end{Usage}
%
\begin{Format}
A matrix with 5 rows and 4 columns.
\end{Format}
%
\begin{Details}\relax
The death rates are measured per 1000 population per year.  They
are cross-classified by age group (rows) and
population group (columns).  The age groups are: 50--54, 55--59,
60--64, 65--69, 70--74 and the population groups are Rural/Male,
Rural/Female, Urban/Male and Urban/Female.

This provides a rather nice 3-way analysis of variance example.
\end{Details}
%
\begin{Source}\relax
Molyneaux, L.,  Gilliam, S. K., and  Florant, L. C.(1947)
Differences in Virginia death rates by color, sex, age,
and rural or urban residence.
\emph{American Sociological Review}, \bold{12}, 525--535.
\end{Source}
%
\begin{References}\relax
McNeil, D. R. (1977)
\emph{Interactive Data Analysis}.
Wiley.
\end{References}
%
\begin{Examples}
\begin{ExampleCode}
require(stats); require(graphics)
n <- length(dr <- c(VADeaths))
nam <- names(VADeaths)
d.VAD <- data.frame(
 Drate = dr,
 age = rep(ordered(rownames(VADeaths)),length.out=n),
 gender= gl(2,5,n, labels= c("M", "F")),
 site =  gl(2,10,  labels= c("rural", "urban")))
coplot(Drate ~ as.numeric(age) | gender * site, data = d.VAD,
       panel = panel.smooth, xlab = "VADeaths data - Given: gender")
summary(aov.VAD <- aov(Drate ~ .^2, data = d.VAD))
opar <- par(mfrow = c(2,2), oma = c(0, 0, 1.1, 0))
plot(aov.VAD)
par(opar)
\end{ExampleCode}
\end{Examples}
\HeaderA{volcano}{Topographic Information on Auckland's Maunga Whau Volcano}{volcano}
\keyword{datasets}{volcano}
%
\begin{Description}\relax
Maunga Whau (Mt Eden) is one of about 50 volcanos in the Auckland
volcanic field.  This data set gives topographic information for
Maunga Whau on a 10m by 10m grid.
\end{Description}
%
\begin{Usage}
\begin{verbatim}
volcano
\end{verbatim}
\end{Usage}
%
\begin{Format}
A matrix with 87 rows and 61 columns, rows corresponding to grid lines
running east to west and columns to grid lines running south to
north.
\end{Format}
%
\begin{Source}\relax
Digitized from a topographic map by Ross Ihaka.
These data should not be regarded as accurate.
\end{Source}
%
\begin{SeeAlso}\relax
\code{\LinkA{filled.contour}{filled.contour}} for a nice plot.
\end{SeeAlso}
%
\begin{Examples}
\begin{ExampleCode}
require(grDevices); require(graphics)
filled.contour(volcano, color.palette = terrain.colors, asp = 1)
title(main = "volcano data: filled contour map")
\end{ExampleCode}
\end{Examples}
\HeaderA{warpbreaks}{The Number of Breaks in Yarn during Weaving}{warpbreaks}
\keyword{datasets}{warpbreaks}
%
\begin{Description}\relax
This data set gives the number of warp breaks per loom, where a loom
corresponds to a fixed length of yarn.
\end{Description}
%
\begin{Usage}
\begin{verbatim}
warpbreaks
\end{verbatim}
\end{Usage}
%
\begin{Format}
A data frame with 54 observations on 3 variables.

\Tabular{rlll}{
\code{[,1]} & \code{breaks}  & numeric & The number of breaks\\{}
\code{[,2]} & \code{wool}    & factor  & The type of wool (A or B)\\{}
\code{[,3]} & \code{tension} & factor  & The level of tension (L, M, H)
}
There are measurements on 9 looms for each of the six types of warp
(\code{AL}, \code{AM}, \code{AH}, \code{BL}, \code{BM}, \code{BH}).
\end{Format}
%
\begin{Source}\relax
Tippett, L. H. C. (1950)
\emph{Technological Applications of Statistics}.
Wiley.  Page 106.

\end{Source}
%
\begin{References}\relax
Tukey, J. W. (1977)
\emph{Exploratory Data Analysis}.
Addison-Wesley.

McNeil, D. R. (1977)
\emph{Interactive Data Analysis}.
Wiley.
\end{References}
%
\begin{SeeAlso}\relax
\code{\LinkA{xtabs}{xtabs}} for ways to display these data as a table.
\end{SeeAlso}
%
\begin{Examples}
\begin{ExampleCode}
require(stats); require(graphics)
summary(warpbreaks)
opar <- par(mfrow = c(1,2), oma = c(0, 0, 1.1, 0))
plot(breaks ~ tension, data = warpbreaks, col = "lightgray",
     varwidth = TRUE, subset = wool == "A", main = "Wool A")
plot(breaks ~ tension, data = warpbreaks, col = "lightgray",
     varwidth = TRUE, subset = wool == "B", main = "Wool B")
mtext("warpbreaks data", side = 3, outer = TRUE)
par(opar)
summary(fm1 <- lm(breaks ~ wool*tension, data = warpbreaks))
anova(fm1)
\end{ExampleCode}
\end{Examples}
\HeaderA{women}{Average Heights and Weights for American Women}{women}
\keyword{datasets}{women}
%
\begin{Description}\relax
This data set gives the average heights and weights for American women
aged 30--39.
\end{Description}
%
\begin{Usage}
\begin{verbatim}
women
\end{verbatim}
\end{Usage}
%
\begin{Format}
A data frame with 15 observations on 2 variables.

\Tabular{rlll}{
\code{[,1]}  & \code{height}  & numeric  & Height (in)\\{}
\code{[,2]}  & \code{weight}  & numeric  & Weight (lbs)
}
\end{Format}
%
\begin{Details}\relax
The data set appears to have been taken from the American Society of
Actuaries \emph{Build and Blood Pressure Study} for some (unknown to
us) earlier year.

The World Almanac notes: ``The figures represent weights in
ordinary indoor clothing and shoes, and heights with shoes''.
\end{Details}
%
\begin{Source}\relax
The World Almanac and Book of Facts, 1975.
\end{Source}
%
\begin{References}\relax
McNeil, D. R. (1977)
\emph{Interactive Data Analysis}.
Wiley.
\end{References}
%
\begin{Examples}
\begin{ExampleCode}
require(graphics)
plot(women, xlab = "Height (in)", ylab = "Weight (lb)",
     main = "women data: American women aged 30-39")
\end{ExampleCode}
\end{Examples}
\HeaderA{WorldPhones}{The World's Telephones}{WorldPhones}
\keyword{datasets}{WorldPhones}
%
\begin{Description}\relax
The number of telephones in various regions of the world (in
thousands).
\end{Description}
%
\begin{Usage}
\begin{verbatim}
WorldPhones
\end{verbatim}
\end{Usage}
%
\begin{Format}
A matrix with 7 rows and 8 columns.  The columns of the matrix give
the figures for a given region, and the rows the figures for a year.

The regions are: North America, Europe, Asia, South America, Oceania,
Africa, Central America.

The years are: 1951, 1956, 1957, 1958, 1959, 1960, 1961.
\end{Format}
%
\begin{Source}\relax
AT\&T (1961) \emph{The World's Telephones}.
\end{Source}
%
\begin{References}\relax
McNeil, D. R. (1977)
\emph{Interactive Data Analysis}.
New York: Wiley.
\end{References}
%
\begin{Examples}
\begin{ExampleCode}
require(graphics)
matplot(rownames(WorldPhones), WorldPhones, type = "b", log = "y",
        xlab = "Year", ylab = "Number of telephones (1000's)")
legend(1951.5, 80000, colnames(WorldPhones), col = 1:6, lty = 1:5, 
       pch = rep(21, 7))
title(main = "World phones data: log scale for response")
\end{ExampleCode}
\end{Examples}
\HeaderA{WWWusage}{Internet Usage per Minute}{WWWusage}
\keyword{datasets}{WWWusage}
%
\begin{Description}\relax
A time series of the numbers of users connected to the Internet
through a server every minute.
\end{Description}
%
\begin{Usage}
\begin{verbatim}
WWWusage
\end{verbatim}
\end{Usage}
%
\begin{Format}
A time series of length 100.
\end{Format}
%
\begin{Source}\relax
Durbin, J. and Koopman, S. J. (2001) \emph{Time Series Analysis by
State Space Methods.}  Oxford University Press.
\url{http://www.ssfpack.com/dkbook/} 
\end{Source}
%
\begin{References}\relax
Makridakis, S., Wheelwright, S. C. and Hyndman, R. J. (1998)
\emph{Forecasting: Methods and Applications.} Wiley.
\end{References}
%
\begin{Examples}
\begin{ExampleCode}
require(graphics)
work <- diff(WWWusage)
par(mfrow = c(2,1)); plot(WWWusage); plot(work)
## Not run: 
require(stats)
aics <- matrix(, 6, 6, dimnames=list(p=0:5, q=0:5))
for(q in 1:5) aics[1, 1+q] <- arima(WWWusage, c(0,1,q),
    optim.control = list(maxit = 500))$aic
for(p in 1:5)
   for(q in 0:5) aics[1+p, 1+q] <- arima(WWWusage, c(p,1,q),
       optim.control = list(maxit = 500))$aic
round(aics - min(aics, na.rm=TRUE), 2)

## End(Not run)
\end{ExampleCode}
\end{Examples}
\clearpage
